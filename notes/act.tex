\subsection{\textit{Act} - Preguntar}
\begin{itemize}
    \item {Definición del desafío/objetivo empresarial, junto a los lineamientos para lo que se considera exitoso}
    \item {Asegurarse de comprender las expectativas de los Stakeholders}
    \item {Colaboración y comunicación abierta con Stakeholders, Expertos en la Materia y otros profesionales}
    \item {Concentrarse en el problema real, evitar distracciones}
    \item {Dar un paso atrás para ver la situación en contexto}
    \item {Aplicación del Pensamiento Analítico y Estructural}
    \item {\textbf{Preguntas efectivas : }
    \begin{itemize}
        \item {Qué problemas dicen tener los stakeholders?}
        \item {Una vez identificado el problema, qué puedo hacer para resolverlo?}
        \item {Cuál es la principal causa del problema?
        \begin{itemize}
            \item {\textbf{Five Why's : }Preguntar ''por qué?'' cinco veces para llegar a la raíz del problema}
        \end{itemize}}
        \item {Cuáles son los déficits de nuestro proyecto?
        \begin{itemize}
            \item {\textbf{Análisis de Déficit - Gap Analysis : }Método de evaluación y examinación de un proceso y cómo funciona actualmente, para determinar cómo llegar a la meta a futuro}
        \end{itemize}}
        \item {Qué no consideramos previamente?
        \begin{itemize}
            \item {Observar el estado actual, e idenfiticar qué lo diferencia del estado ideal}
            \item {Hay datos previos para revisar?}
            \item {Hay preconcepciones? Cuáles?}
        \end{itemize}}
        \item{Qué intentamos lograr?}
    \end{itemize}}
    \item{Aspectos útiles para manejar una conversación:
    \begin{itemize}
        \item {\textit{Hechos : }Anotar datos concretos}
        \item {\textit{Contexto : }Cualquier cosa que facilite la comprensión}
        \item {\textit{Incógnitas : }Hay algo que se pase por alto? Hay conceptos desconocidos?}
        \item {\textit{Temas comunes para crear preguntas : }Objetivos, audiencia, tiempo, recursos, seguridad. Aspectos útiles para manejar una conversación}
    \end{itemize}}
\end{itemize}


\subsubsection{Preguntas \textit{SMART}}
Preguntas altamente efectivas, generalmente abiertas, que incitan a respuestas útiles. Es importante que estas preguntas sean justas, y no creen o mantengan sesgos.

\paragraph{\textit{Specific : }Simples, enfocadas en una cantidad pequeña de ideas relacionadas entre si}
\begin{itemize}
    \item {La pregunta tiene el contexto necesario?}
    \item {Aborda el problema a resolver?}
    \item {Entrega la respuesta que necesito?}
\end{itemize}

\paragraph{\textit{Measurable : }Cuantificables y evaluables}
\begin{itemize}
    \item {Esta pregunta me da respuestas que puedo medir?}
\end{itemize}

\paragraph{\textit{Action-oriented : }Promueven el cambio y la acción}
\begin{itemize}
    \item {La pregunta entregará información que ayude a diseñar un plan de acción o cambio?}
\end{itemize}

\paragraph{\textit{Relevant : }Tienen importancia en el problema a resolver}
\begin{itemize}
    \item {Esta pregunta se relaciona con un problema en específico?}
\end{itemize}

\paragraph{\textit{Time-bound : }Son específicas al periodo de tiempo a estudiar}
\begin{itemize}
    \item {La respuesta es relevante para el periodo establecido?}
\end{itemize}

\paragraph{Preguntas a evitar : }
\begin{itemize}
    \item {\textbf{Sugestivas}, con una respuesta en particular}
    \item {\textbf{Cerradas}, con una respuesta breve o de una sola palabra}
    \item {\textbf{Imprecisas}, no específicas o que no ofrecen contexto}
\end{itemize}

\newpage