\section{\textit{Prepare} - Preparar}
Una vez se define el \gls{objemp}, se \textbf{planifican} los datos para resolverlo. Se determinan \textbf{métricas, formatos y estructuras}, junto a \textbf{medidas de seguridad} como protección, administración y acceso. También se toman medidas contra \textbf{sesgos estadísticos}

\paragraph{Beneficios de Modificar Formatos y Estructuras}
\begin{itemize}
    \item {\textbf{Organización} : más fácil de usar}
    \item {\textbf{Compatibilidad} : el mismo set de datos puede usarse por aplicaciones o sistemas externos y diferentes}
    \item {\textbf{Migración} : los datos pueden transladarse entre sistemas con formatos similares}
    \item {\textbf{Fusión} : unión de datos en una misma organización}
    \item {\textbf{Mejora} : los datos pueden mostrarse en campos más detallados}
    \item {\textbf{Comparación} : entre datos o sets}
\end{itemize}

\subsection{Ordenar vs. Filtrar}
\begin{itemize}
    \item {\textbf{Ordenado} : Búsqueda de orden significativo para facilitar el entender, analizar y visualizar datos}
    \item {\textbf{Filtrar} : Ocultar datos que no cumplan con criterios definidos, para destacar los que si lo hacen}
\end{itemize}

\subsection{Organización de Datos}
Al organizar los datos, estos son más fáciles de encontrar, usar, mantener seguros y también previenen problemas a futuro. Se recomienda establecer las siguientes prácticas con el equipo de trabajo al empezar el proyecto, e implementar convenciones de metadatos
\begin{itemize}
    \item {\textbf{Convenciones de Nomenclatura} : Crear nombres significativos y cortos para archivos que lo describan. Eligiendo nombres descriptivos y lógicos, son más fáciles de utilizar y encontrar. Algunos elementos a usar son fechas (formato según país de trabajo), versiones (v02), nombre del proyecto. Por último, evitar espacios o caracteres extraños, y preferir guiones (bajos) o mayúsculas}
    \item {\textbf{Jerarquía Lógica de Carpetas} : Utilizar carpetas y subcarpetas en jerarquía lógica para agrupar archivos relacionados a algún proyecto o entre si}
    \item {\textbf{Archivado} : Mover proyectos viejos o en desuso a otra ubicación, ya sean carpetas o unidades de almacenamiento externo. Esto permite crear un archivo, además de eliminar ruido visual. También es una buena oportunidad para realizar copias automáticas, o agendarlas de forma manual}
\end{itemize}

\subsection{Niveles y Técnicas del Modelado de Datos}
Un \textbf{modelo} de datos consiste en diagramas o visualizaciones que representen la organización y conexión entre datos. \textbf{Algunas técnicas de modelado comunes son \textit{ERD, UML, Data Dictionary}}. Los tipos más comunes de modelado son 
\begin{itemize}
    \item {\textbf{Modelado Conceptual} : Ofrece una visión de alto nivel de la estructura de datos sin incluir detalles técnicos. \textbf{Forma en que los datos interactúan en una organización}}
    \item {\textbf{Modelado Lógico} : Se centra en los detalles técnicos de una \textit{DB} como relaciones, atributos y entidades, inlcuyendo cómo identificar registros individuales}
    \item {\textbf{Modelado Físico} : Describe el funcionamiento de una \textit{DB} junto con todas las entidades y atributos utilizados, como \textit{ nombres de columnas y tablas, tipo de DB}} 
\end{itemize}

\subsection{Ética de los Datos}
Extendiendo el concepto de \gls{ethcs}, la \textbf{ética de los datos} consiste en estándares para la recolección, uso y distribución de los datos, justificados por el \textit{\textbf{GDPR} - General Data Protection Regulation of the European Union}. Sus aspectos consisten en
\begin{itemize}
    \item {\textbf{\textit{Ownership} - Propiedad} : Los datos sin procesar son propiedad de los individuos que la entregan, y tienen el control principal sobre su uso, proceso y distribución}
    \item {\textbf{\textit{Transaction Transparency} - Transparencia de Transacción} : Las actividades y algoritmos utilizados para procesar datos, debieran ser transparentes y entendibles para el individuo que los entrega}
    \item {\textbf{\textit{Consent} - Consentimiento} : El individuo tiene derecho a conocer las formas y razones detrás del uso de sus datos antes de que estén de acuerdo con entregarlos}
    \item {\textbf{\textit{Currency} - Transacciones} : Los individuos debieran ser conscientes de las transacciones monetarias y su escala, que resulten del uso de los datos}
    \item {\textbf{\textit{Privacy} - Privacidad} : Ante toda transacción de datos, la información y actividad personal del individuo debe asegurarse. Esto puede traducirse en
    \begin{itemize}
        \item {Protección contra acceso no autorizado a datos privados}
        \item {Libertad ante uso inapropiado de los datos}
        \item {Derecho a inspeccionar, actualizar o corregir datos}
        \item {Consentimiento libre para el uso o restricción de datos}
        \item {Derecho legal al acceso de los datos}
    \end{itemize}}
    \item {\textbf{\textit{Openness} - Apertura} : Los datos procesados debieran ser de libre acceso, uso y distribución. Los estándares de \textbf{\textit{Open Data}} son
    \begin{itemize}
        \item {\textbf{Disponibilidad y Acceso}}
        \item {\textbf{Reutilización y Redistribución}}
        \item {\textbf{Participación Universal}}
        \item {\textbf{Interoperabilidad} : capacidad de conectar sistemas y servicios para la conexión y distribución abierta de datos}
    \end{itemize}}
\end{itemize}

\subsubsection{Anonimización de Datos}
Protección de los \textbf{datos privados o confidenciales} de las personas. \textit{Ej : patentes de autos, direcciones IP, registros médicos, cuentas bancarias, número de seguro social}
\begin{itemize}
    \item {\textbf{\textit{PII} - Información de Identificación Personal} : Información que puede usarse por si misma o junto a otros datos, para rastrear la identidad de una persona}
    \item {\textbf{Desidentificación} : proceso para eliminar toda la información de identificación personal. \textit{Ejemplo : dejar espacios en blanco, ejecutar comando hash, enmascarar con códigos de longitud fija, ocultar utilizando valores alterados}}
\end{itemize}

\subsection{Seguridad de los Datos}
Protección de los datos ante posibles accesos no autorizados, o corrupción de distintos grados. Algunas herramientas como hojas de cálculo tienen medidas de seguridad incorporadas, como permisos de edición, visualización o control. Pero otras medidas generales consisten en
\begin{itemize}
    \item {\textbf{Cifrado} : Un algoritmo único altera los datos y funciona como clave para que otros individuos o sistemas puedan usarlos}
    \item {\textbf{Tokenización} : Reemplazo de los datos con otros generados aleatoriamente. Para acceder a los datos originales y completos, el usuario o aplicación debe tener permiso para utilizar los datos tokenizados, junto con la asignación del token. Si estos datos tokenizados se piratean, los originales estarán seguros en otra ubicación}
\end{itemize}



\newpage