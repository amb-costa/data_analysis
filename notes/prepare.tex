% nta: eventualmente se añade esto: cosas para aprender en la unidad
% cómo se generan los datos
% diferentes formatos, tipos y estructuras de datos
% analizar sesgo y credibilidad
% qué significa "datos limpios"?
% familiaridad con db
% extracción de datos con sql
% introducción a organización de datos
% protección de datos


\subsection{\textit{Prepare} - Preparar}
Una vez se define el proyecto, se \textbf{planifican} los datos necesarios para resolver el desafío. Se determinan \textbf{métricas, formatos y estructuras}, junto a \textbf{medidas de seguridad} como protección, administración y acceso. También se toman medidas contra \textbf{sesgo y parcialidad}
\begin{itemize}
    \item {\textbf{Preguntas Efectivas} :
    \begin{itemize}
        \item {Qué tipo de datos son adecuados y correctos para resolver el problema?}
        \item {Qué necesito para descubrir cómo resolver este problema? qué investigación previa necesito?}
        \item {Qué habilidades prácticas son necesarias para extraer, utilizar, organizar y proteger los datos?}
    \end{itemize}}
\end{itemize}

\subsubsection{Conceptos Previos}
\begin{itemize}
    \item {\textbf{Población} : Todos los valores posibles en un set de datos}
    \item{\textbf{Muestra} : Parte de una población, y que es representativa del mismo}
    \item {\textbf{Modelo de Datos} : modelo utilizado para organizar y relacionar los \textit{Elementos de Datos : piezas de información como nombres de personas, números de cuenta y direcciones}}
\end{itemize}

\subsubsection{Tipos de Datos}
\paragraph{Por Objetividad y Capacidad de Medición}
\begin{itemize}
    \item{\textbf{Cuantitativos} : Cuentan con hechos numéricos, medidas específicas y objetivas. \textit{Qué cosas, cuántas, qué tan seguido?} Se subdividen en dos tipos:
    \begin{itemize}
        \item{\textbf{Discretos} : Son cuantificables y tienen un número limitado de valores}
        \item{\textbf{Contínuos} : Se miden y pueden tener casi cualquier valor numérico}
    \end{itemize}}
    \item{\textbf{Cualitativos} : Medidas subjetivas o explanatorias para cualidades y características. \textit{Por qué? Cómo?} Estos datos también se dividen en
    \begin{itemize}
        \item{\textbf{Nominal} : No presentan estructura o secuencia aparente}
        \item{\textbf{Ordinal} : Pueden ordenarse en base a características}
    \end{itemize}}
\end{itemize}

\paragraph{Por Cercanía a sus Fuentes}
\begin{itemize}
    \item {\textbf{\textit{$1^{\text{st}}$ party data} - de Primera Fuente} : Datos recolectados por un individuo o grupo, utilizando recursos propios}
    \item {\textbf{\textit{$2^{\text{nd}}$ party data} - de Segunda Fuente} : Datos recolectados por un grupo, directamente desde su audiencia, y que luego se venden}
    \item {\textbf{\textit{$3^{\text{rd}}$ party data} - de Tercera Fuente} : Datos recolectados desde fuentes externas, que no los capturaron directamente}
\end{itemize}

\paragraph{Por Ubicación}
\begin{itemize}
    \item {\textbf{Internos} : Datos que viven dentro de los sistemas de la compañía o empresa}
    \item {\textbf{Externos} : Datos que viven o son generados fuera de la organización}
\end{itemize}

\subsubsection{Estructuras de Datos}
\begin{itemize}
    \item {\textbf{Estructurados} : Datos organizados en formatos específicos, como filas y columnas. \textit{Ejemplos : Spreadsheets, DB}}
    \item {\textbf{No Estructurados} : Datos que no evidencian ninguna manera fácil de organizar. \textit{Ejemplos : audios, videos, archivos, correos, fotos, redes sociales}}
\end{itemize}


\subsubsection{Recolección de Datos}
Al momento de recolectar, se consideran los métodos de captura, fuentes de datos, la cantidad y periodo deseado. Una vez se tienen los datos, se decide y filtra qué datos utilizar

\begin{itemize}
    \item {Entrevistas}
    \item {Observaciones}
    \item {Formularios, cuestionarios, encuestas}
    \item {\textit{Cookies}}
\end{itemize}