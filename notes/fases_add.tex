\section{Ciclo de Vida del Análisis de Datos}

% nta: ahora el ciclo de vida pasa a secciones, con tal de mantener orden en secciones y subsecciones

\section{\textit{Ask} - Preguntar}
Se establece un \gls{probemp}. Es necesario comprender las expectativas de los \gls{stkhldrs}, junto a sus desafíos o preconcepciones anteriores. Algunas técnicas o métodos a aplicar incluyen
\begin{itemize}
    \item {\gls{strctrdthnkng}}
    \item {\gls{5w}, para determinar la causa principal de un problema}
    \item {\gls{gpnlss}, para encontrar déficits en el proyecto}
\end{itemize}

\subsection{Tipos de Objetivos}
\begin{description}
    \item [Realizar Predicciones]{ : Decisiones informadas para visualizar las cosas a futuro}
    \item [Categorizar Cosas]{ : Asignar información a dos grupos de datos distintos con cualidades comunes}
    \item [Encontrar Algo Inusual]{ : Identificar datos que se salen de la norma}
    \item [Identificar Temas]{ : Agrupar información categorizada en conceptos más amplios}
    \item [Descubrir Conexiones]{ : Encontrar desafíos similares encontrados por otros, para combinar datos y conclusiones}
    \item [Encontrar Patrones]{ : Utilizar datos históricos para entender qué ocurrió en el pasado, y que podría repetirse}
\end{description}


\subsection{Preguntas \textit{SMART}}
Preguntas abiertas (por lo general), que incitan a respuestas útiles y efectivas. \textit{Esto contrasta con las preguntas \textbf{sugestivas} con respuestas en particular, \textbf{imprecisas} que no ofrecen \gls{cntxt}, y \textbf{cerradas} con respuestas breves o de una sola palabra}
\begin{description}
    \item [\textit{Specific}]{ : Simples, enfocadas en una cantidad pequeña de ideas relacionadas entre si. \textbf{Esta pregunta aborda el problema, entrega la respuesta o \gls{cntxt} que necesito?}}
    \item [\textit{Measurable}]{ : Cuantificables y evaluables. \textbf{Esta pregunta da respuestas medibles?}}
    \item [\textit{Action-Oriented}]{ : Promueven el cambio y la acción. \textbf{Esta pregunta entregará información que ayude a diseñar un plan de acción o cambio?}}
    \item [\textit{Relevant}]{ : Tienen importancia en el problema a resolver. \textbf{Esta pregunta se relaciona con un problema en específico?}}
    \item [\textit{Time-Bound}]{ : Son específicas al periodo de tiempo estudiado. \textbf{La pregunta es relevante para el periodo establecido?}}
\end{description}

\newpage

\subsection{\textit{Prepare} - Preparar}
Una vez definido el proyecto, se \textbf{planifican} los datos y sus características
\begin{itemize}
   \item {Determinar métricas, formatos y estructuras de datos}
   \item {Establecer medidas de seguridad y protección}
   \item {Acordar almacenamiento, administración y acceso}
   \item {Tomar medidas contra sesgos y parcialidad}
    \item {\textbf{Preguntas Efectivas} :
    \begin{itemize}
        \item {Qué necesito para descubrir cómo resolver este problema?}
        \item {Qué investigación previa debo realizar?}
    \end{itemize}}
\end{itemize}

\subsection{\textit{Process} - Procesar}
Ya teniendo los datos, estos pasan por una \textbf{limpieza} para corregir limitaciones
\begin{itemize}
    \item {Buscar datos incorrectos, repetidos o espacios vacíos en \textit{Spreadsheets} y \textit{DB}}
    \item {Manejar imprecisiones, casos extremos y datos desalineados sin métricas estandarizadas}
    \item {Comprobar presencia de sesgos e inexactitudes}
    \item {\textbf{Preguntas Efectivas} :
    \begin{itemize}
        \item {Cómo puedo limpiar datos para que mis datos sean más coherentes?}
        \item {Qué errores podrían interponerse en mi camino?}
    \end{itemize}}
\end{itemize}

\subsection{\textit{Analyze} - Analizar}
\begin{itemize}
    \item {Exploración de los datos ya preparados :
    \begin{itemize}
        \item {Formatear, transformar, ordenar y filtrar datos}
        \item {Combinar datos de varias fuentes, teniendo en mente su perspectiva o \gls{cntxt} : 
        \begin{description}
            \item[Quién]{ : Creó, recopiló, financió la recopilación de datos}
            \item[Qué]{ : Circunstancias que generan impacto en cualquier parte del mundo}
            \item[Dónde]{ : Origen de los datos}
            \item[Cuándo]{ : Momento de creación o recopilación} 
            \item[Cómo]{ : Método con el que se creó o recopiló} 
        \end{description}}
    \end{itemize}}
    \item {Identificar patrones y conclusiones}
    \item {Realizar predicciones y recomendaciones}
    \item {\textbf{Preguntas Efectivas:}
    \begin{itemize}
        \item {Qué historia me cuentan los datos?}
        \item {Como ayudan estos datos a resolver el problema?}
        \item {Quién necesita el producto o servicio, qué tipo de cliente es más probable que lo use?}
    \end{itemize}}
\end{itemize}

\subsection{\textit{Share} - Compartir}
\begin{itemize}
    \item {Interpretar resultados a través de \textbf{Visualización}}
    \item {Comunicar resultados : 
    \begin{itemize}
        \item {Exclusivo para Stakeholders y personas con acceso autorizado}
        \item {Conversaciones para ayudar a comprender resultados}
    \end{itemize}}
    \item {\textbf{Preguntas efectivas:}
    \begin{itemize}
        \item {Cómo puedo visualizar los datos para que sean atractivos y fáciles de entender?}
        \item {Cómo puedo ayudar al oyente a entender la información?}
        \item {Cuáles son las decisiones más informadas?}
        \item {Qué decisiones pueden tomarse en base a los resultados?}
    \end{itemize}}
\end{itemize}

\subsection{\textit{Act} - Actuar}
\begin{itemize}
    \item {Transformar resultados en acciones de cambio}
    \item {Resolver el desafío empresarial definido al principio del ciclo}
    \item {Crear algo nuevo: otro proceso, otro problema, modelos nuevos}
    \item {\textbf{Preguntas efectivas: }
    \begin{itemize}
        \item {Cómo puedo utilizar el feedback de los stakeholders para satisfacer sus necesidades y expectativas?}
    \end{itemize}}
\end{itemize}

\newpage