\section{Ciclo de Vida del Análisis de Datos}
Segun Google Data Analytics

\subsection{Act - Preguntar}
\begin{itemize}
    \item {Definición del desafío/objetivo empresarial, junto a los lineamientos para lo que se considera exitoso}
    \item {Comprender las expectativas de los Stakeholders}
    \item {Colaboración con Expertos en la Materia y otros profesionales}
    \item {Aplicación del Pensamiento Analítico/Estructural}
    \item {\textbf{Preguntas efectivas : }
    \begin{itemize}
        \item {Cuál es la principal causa del problema?}
        \begin{itemize}
            \item {\textbf{Five Why's : }Preguntar ''por qué?'' cinco veces para llegar a la raíz del problema}
        \end{itemize}
        \item {Cuáles son los déficits de nuestro proyecto?}
        \begin{itemize}
            \item {\textbf{Análisis de Déficit - Gap Analysis : }Método de evaluación y examinación de un proceso y cómo funciona actualmente, para determinar cómo llegar a la meta a futuro}
        \end{itemize}
        \item {Qué no consideramos previamente?}
        \begin{itemize}
            \item {Observar el estado actual, e idenfiticar qué lo diferencia del estado ideal}
        \end{itemize}
    \end{itemize}}
\end{itemize}

\subsection{Prepare - Preparar}
\begin{itemize}
    \item {Recolectar y capturar datos}
    \item {Seleccionar tipo de dato de acuerdo al problema: formatos, tipos y estructuras}
    \item {Tomar medidas para evitar sesgos y mantener imparcialidad}
    \item {Almacenamiento y administración de dichos datos}
\end{itemize}

\subsection{Process - Procesar}
\begin{itemize}
    \item {\textbf{Limpieza de datos : }
    \begin{itemize}
        \item {Completar datos usando otras fuentes externas}
        \item {Eliminar y manejar errores, imprecisiones y casos extremos}
        \item {Transformación a formatos distintos}
    \end{itemize}}
    \item {Control de calidad previo al análisis}
\end{itemize}

\subsection{Analyze - Analizar}
\begin{itemize}
    \item {Exploración de los datos ya preparados :
    \begin{itemize}
        \item {Usar herramientascomo hojas de cálculo, bases de datos y queries}
        \item {Formatear, tansformar, ordenar y filtrar datos}
    \end{itemize}
    \item {Identificar patrones y conclusiones}
    \item {Realizar predicciones y recomendaciones}}
\end{itemize}

\subsection{Share - Compartir}
\begin{itemize}
    \item {Interpretar resultados a través de \textbf{Visualización}}
    \item {Comunicar resultados y ayudar a Stakeholders para comprenderlos}
    \item {Evaluación y toma de decisiones}
\end{itemize}

\subsection{Act - Actuar}
\begin{itemize}
    \item {Transformar resultados en acciones de cambio}
    \item {Resolver el desafío empresarial definido al principio del ciclo}
    \item {Crear algo nuevo: otro proceso, otro problema, modelos nuevos}
\end{itemize}

\newpage