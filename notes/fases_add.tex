\section{Ciclo de Vida del Análisis de Datos}



\subsection{\textit{Analyze} - Analizar}
\begin{itemize}
    \item {Exploración de los datos ya preparados :
    \begin{itemize}
        \item {Formatear, transformar, ordenar y filtrar datos}
        \item {Combinar datos de varias fuentes, teniendo en mente su perspectiva o \gls{cntxt} : 
        \begin{description}
            \item[Quién]{ : Creó, recopiló, financió la recopilación de datos}
            \item[Qué]{ : Circunstancias que generan impacto en cualquier parte del mundo}
            \item[Dónde]{ : Origen de los datos}
            \item[Cuándo]{ : Momento de creación o recopilación} 
            \item[Cómo]{ : Método con el que se creó o recopiló} 
        \end{description}}
    \end{itemize}}
    \item {Identificar patrones y conclusiones}
    \item {Realizar predicciones y recomendaciones}
    \item {\textbf{Preguntas Efectivas:}
    \begin{itemize}
        \item {Qué historia me cuentan los datos?}
        \item {Como ayudan estos datos a resolver el problema?}
        \item {Quién necesita el producto o servicio, qué tipo de cliente es más probable que lo use?}
    \end{itemize}}
\end{itemize}

\subsection{\textit{Share} - Compartir}
\begin{itemize}
    \item {Interpretar resultados a través de \textbf{Visualización}}
    \item {Comunicar resultados : 
    \begin{itemize}
        \item {Exclusivo para Stakeholders y personas con acceso autorizado}
        \item {Conversaciones para ayudar a comprender resultados}
    \end{itemize}}
    \item {\textbf{Preguntas efectivas:}
    \begin{itemize}
        \item {Cómo puedo visualizar los datos para que sean atractivos y fáciles de entender?}
        \item {Cómo puedo ayudar al oyente a entender la información?}
        \item {Cuáles son las decisiones más informadas?}
        \item {Qué decisiones pueden tomarse en base a los resultados?}
    \end{itemize}}
\end{itemize}

\subsection{\textit{Act} - Actuar}
\begin{itemize}
    \item {Transformar resultados en acciones de cambio}
    \item {Resolver el desafío empresarial definido al principio del ciclo}
    \item {Crear algo nuevo: otro proceso, otro problema, modelos nuevos}
    \item {\textbf{Preguntas efectivas: }
    \begin{itemize}
        \item {Cómo puedo utilizar el feedback de los stakeholders para satisfacer sus necesidades y expectativas?}
    \end{itemize}}
\end{itemize}

\newpage