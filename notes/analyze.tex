% nta: ahora el ciclo de vida pasa a secciones, con tal de mantener orden en secciones y subsecciones

\section{\textit{Analyze} - Analizar}
Una vez \textbf{organizados, agregados, formateados y ajustados} los datos, es posible realizar el proceso de \textbf{análisis} en si. La meta es identificar tendencias y relaciones entre datos, que puedan resolver el \gls{probemp}

Al momento de combinar fuentes, se debe considerar el \gls{cntxt} : \textbf{quién} los creó, recopiló o financió, \textbf{dónde} se originan, \textbf{cuándo} se crearon o recopilaron, y \textbf{cómo} lo hicieron

\subsection{Formateo y Ajuste de Datos}
Los \textbf{Datos Atípicos} son puntos de datos que difieren mucho de aquellos que se recogieron de forma similar, y podrían no ser fiables o inducir a errores. Lo más probable es que se trabaje con datos que necesiten correcciones para ser considerados útiles. 
\begin{itemize}
    \item {\textbf{Ordenación} : Organización de datos en un orden significativo, para que sea más fácil entenderlos} 
    \item {\textbf{Filtrado} : Proceso para ocultar datos que no cumplen con criterios definidos previamente, para destacar los que si lo hacen}
\end{itemize}

\subsection{\textit{Data Aggregation} - Agregación de Datos}
En el contexto de análisis, la agregación de datos consiste en el proceso de recolección de datos, utilizando múltiples fuentes, para combinarlo todo en un solo grupo. A través de esto se logra 
\begin{itemize}
    \item {Identificar Tendencias}
    \item {Realizar comparaciones}
    \item {Obtener información}
    \item {Cuando se mide a través de un periodo de tiempo, se pueden obtener estadísticas como \textbf{promedios, mínimos, máximos y sumas}}
\end{itemize}


\newpage