\section{\textit{Analyze} - Analizar}
Una vez \textbf{organizados, agregados, formateados y ajustados} los datos, es posible realizar el proceso de \textbf{análisis} en si. La meta es identificar tendencias y relaciones entre datos, que puedan resolver el \gls{probemp}

Al momento de combinar fuentes, se debe considerar el \gls{cntxt} : \textbf{quién} los creó, recopiló o financió, \textbf{dónde} se originan, \textbf{cuándo} se crearon o recopilaron, y \textbf{cómo} lo hicieron

\subsection{Formateo y Ajuste de Datos}
Los \textbf{Datos Atípicos} son puntos de datos que difieren mucho de aquellos que se recogieron de forma similar, y podrían no ser fiables o inducir a errores. Lo más probable es que se trabaje con datos que necesiten correcciones para ser considerados útiles. 
\begin{itemize}
    \item {\textbf{Ordenación} : Organización de datos en un orden significativo, para que sea más fácil entenderlos} 
    \item {\textbf{Filtrado} : Proceso para ocultar datos que no cumplen con criterios definidos previamente, para destacar los que si lo hacen}
\end{itemize}

\subsection{\textit{Data Aggregation} - Agregación de Datos}
En el contexto de análisis, la agregación de datos consiste en el proceso de recolección de datos, utilizando múltiples fuentes, para combinarlo todo en un solo grupo. A través de esto se logra 
\begin{itemize}
    \item {Identificar Tendencias}
    \item {Realizar comparaciones}
    \item {Obtener información}
    \item {Cuando se mide a través de un periodo de tiempo, se pueden obtener estadísticas como \textbf{promedios, mínimos, máximos y sumas}}
\end{itemize}

\subsection{Validación de Datos}
La validación es una herramienta de revisión de exactitud y calidad de los datos antes de añadirlos o importarlos. Se utilizan criterios para los datos, y la validez se determina cuando los valores cumplen (o no) con las restricciones establecidas. \textit{Es una forma de limpieza de datos}

\subsubsection{Tipos de Validación}
\begin{itemize}
    \item {\textbf{Tipo de Datos} : los datos coinciden con los tipos de datos definidos para un campo}
    \item {\textbf{Rango de Datos} : los datos se ubican dentro de un rango de valores aceptable definido para el campo}
    \item {\textbf{Limitaciones de Datos} : los datos cumplen con ciertas condiciones/criterios para un campo (número entero? cabtudad de carácteres?)}
    \item {\textbf{Coherencia de Datos} : los datos tienen sentido en el contexto de otros datos}
    \item {\textbf{Estructura de Datos} : los datos siguen o se ajustan a una estructura establecida}
    \item {\textbf{Validación de Código} : los códigos de aplicación usados realizan sistemáticamente cualquiera de las validaciones antes mencionadas}
\end{itemize}

\subsubsection{Condiciones de Validación}
\begin{description}
    \item [\gls{datatype}]{ : formato correcto para los valores}
    \item [Rango]{ : los valores se encuentran dentro de un rango de máximos y mínimos predefinidos}
    \item [Obligatoriedad]{ : los valores no pueden ser nulos, blancos o vacíos}
    \item [Unicidad]{ : ningún valor está duplicado}
    \item [Largo de Campo]{ : el largo de un campo está dentro del rango de caracteres esperado}
    \item [\gls{rgx}]{ : los valores coinciden con patrones preestablecidos}
    \item [Campos Cruzados]{ : los valores cumplen simultáneamente con restricciones múltiples}
    \item [Membresía]{ : los valores en una columna provienen de un conjunto de valores específicos} 
    \item [Exactitud]{ : se alcanza un grado de conformidad establecido, respecto a la entidad real medida}
    \item [Exhaustividad]{ : grado en que los datos contienen todos los componentes deseados}
    \item [Coherencia]{ : grado de repetibilidad de los datos desde diferentes puntos de entrada o recopilación}
\end{description}

\subsubsection{Proceso de Verificación}
Al verificar el proceso de limpieza, se confirma su calidad: si fue bien ejecutada, si los datos resultantes son precisos y confiables. Si es necesario, se realiza otra limpieza manual. Algunos pasos de verificación son
\begin{itemize}
    \item {Comparar datos limpios vs. sucios}
    \item {Ver el proyecto desde una perspectiva general, revisar cómo los datos podrían resolver las metas y problemas del proyecto. \textit{Compatibilidad con la lógica de negocios}}
    \item {Poner atención a cualquier detalle sospechoso o problemático, como números que no hagan sentido}
\end{itemize}

\paragraph{Lista de checkeo para Verificación}
\begin{itemize}
    \item {Fuentes de errores}
    \item {Faltas de tipografía en palabras y cifras}
    \item {Espacios y caracteres extra}
    \item {Duplicados}
    \item {Formatos no coincidentes}
    \item {\gls{str} y fechas desordenadas o incoherentes}
    \item {Etiquetas engañosas de columnas}
    \item {Datos truncados}
\end{itemize}


\newpage