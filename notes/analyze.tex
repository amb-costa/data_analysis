% nta: ahora el ciclo de vida pasa a secciones, con tal de mantener orden en secciones y subsecciones

\section{\textit{Analyze} - Analizar}
Una vez \textbf{organizados, agregados, formateados y ajustados} los datos, es posible realizar el proceso de \textbf{análisis} en si. La meta es identificar tendencias y relaciones entre datos, que puedan resolver el \gls{probemp}

\begin{itemize}
    \item {Exploración de los datos ya preparados :
    \begin{itemize}
        \item {Formatear, transformar, ordenar y filtrar datos}
        \item {Combinar datos de varias fuentes, teniendo en mente su perspectiva o \gls{cntxt} : 
        \begin{description}
            \item[Quién]{ : Creó, recopiló, financió la recopilación de datos}
            \item[Qué]{ : Circunstancias que generan impacto en cualquier parte del mundo}
            \item[Dónde]{ : Origen de los datos}
            \item[Cuándo]{ : Momento de creación o recopilación} 
            \item[Cómo]{ : Método con el que se creó o recopiló} 
        \end{description}}
    \end{itemize}}
    \item {\textbf{Preguntas Efectivas:}
    \begin{itemize}
        \item {Qué historia me cuentan los datos?}
        \item {Como ayudan estos datos a resolver el problema?}
        \item {Quién necesita el producto o servicio, qué tipo de cliente es más probable que lo use?}
    \end{itemize}}
\end{itemize}

\subsection{Formateo y Ajuste de Datos}
Los \textbf{Datos Atípicos} son puntos de datos que difieren mucho de aquellos que se recogieron de forma similar, y podrían no ser fiables o inducir a errores. Lo más probable es que se trabaje con datos que necesiten correcciones para ser considerados útiles. 

\begin{itemize}
    \item {\textbf{Ordenación} : Organización de datos en un orden significativo, para que sea más fácil entenderlos} 
    \item {\textbf{Filtrado} : Proceso para ocultar datos que no cumplen con criterios definidos previamente, para destacar los que si lo hacen}
\end{itemize}