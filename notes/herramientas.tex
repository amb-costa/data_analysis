\section {Kit de Herramientas del Analista}

\subsection{\textit{Spreadsheets} - Hojas de Cálculo}
\begin{description}
    \item[Celda]{ : Posición descrita usando filas \textbf{o registros/\textit{records}} y columnas \textbf{o campos/\textit{fields}}. \textit{Ejemplo : columna A fila 3 = A3}. 
    \begin{description}
        \item[Referencia Relativa]{ : El valor de la celda cambia junto con la fórmula donde está presente, al arrastrarla o copiarla en otro lugar}
        \item[Referencia Absoluta]{ : El valor de la celda es fijo y no cambia con movimientos en fórmulas o celdas.\textit{Para realizar cambio entre absoluto y relativo, presionar F4}}
        \begin{table}
            \centering
            \begin{tabular}{|p{5cm}|p{4.5cm}|p{4.5cm}|}
                \hline
                \multicolumn{3}{|c|}{Diferencias entre Referencias de Celdas} \\
                \hline
                & \textbf{Referencia Absoluta} & \textbf{Referencia Relativa} \\
                \hline
                \textbf{Ejemplos} & A1, C3, G2:G9 & \$A\$10, C\$2, \$D3 \\
                \hline
                \textbf{Si la celda se arrastra una columna a la derecha, una fila hacia abajo} & A10 $\rightarrow$ B11 & \$A10 $\rightarrow$ \$A11 \\
                \hline
            \end{tabular}
        \end{table}
        \item[Formato Condicional]{ : Es posible resaltar celdas con colores diferentes en función del contexto, como errores o valores importantes}
    \end{description}}
    \item[Fórmulas y Funciones]{ : 
    \begin{description}
        \item {Para empezar fórmulas, se empieza escribiendo '='. Al hacerlo, se despliega un menú que incluye fórmulas, nombres y \textit{strings}}
        \item {No todas las funciones requieren operadores. \textit{Ejemplos : SUM, AVERAGE, COUNT, MIN, MAX}}
        \item {Es posible combinar ambos, la fórmula entrega los criterios bajo los que una función se ejecuta. \textit{Ejemplo : =CONTAR:SI()}}
        \item {En la esquina inferior derecha se muestra un cuadrado de \textbf{Llenado Automático} : al hacer clic y arrastrar, se rellenan otras celdas con el mismo término}
        \item[Errores Comunes]{ : 
        \begin{itemize}
            \item {\textbf{\#DIV/0!} : Intento de dividir por cero o por una celda vacía}
            \item {\textbf{\#ERROR()} : Fórmula no puede ser interpretada como input. \textit{Parsing error}}
            \item {\textbf{\#N/A} : Los datos a usar en la fórmula no se encuentran en la hoja}
            \item {\textbf{\#NAME?} : No se entiende el nombre de la fórmula o función}
            \item {\textbf{\#NUM!} : Los cálculos de la fórmula o función no pueden realizarse}
            \item {\textbf{\#VALUE!} : Error general para problemas con fórmulas o referencias de celda}
            \item {\textbf{\#REF!} : Una fórmula referencia una celda no válida o eliminada}
        \end{itemize}} 
    \end{description}}
\end{description}




\subsection{\textit{DB + Queries} - Bases de Datos y Consultas}
\begin{itemize}
    \item {Aislar información específica de una base de datos}
    \item {Facilitar aprendizaje y comprensión de las solicitudes/consultas}
    \item {Seleccionar, crear, agregar o descargar datos desde la Base de Datos para Análisis}
    \item {\textbf{\textit{Query} - Consulta} : Petición a una Base de Datos para obtener y manipular datos}
    \item {\textbf{Tipos de Base de Datos : }
    \begin{itemize}
        \item {\textbf{SQL : }\textit{Structured Query Language.} MySQL, PostgreSQL, Big Query}
        \item {\textbf{NOSQL}}
    \end{itemize}}
\end{itemize}

\subsection{\textit{Spreadsheets vs Databases}}
\begin{table}
    \centering
    \begin{tabular}{|p{3.5cm}|p{5.5cm}|p{5.5cm}|}
        \hline
        & \textbf{Spreadsheet} & \textbf{Database} \\
        \hline
        \textbf{Ubicación} & Apps de Software & Almacenes de datos accesibles mediante \textit{queries} \\
        \hline
        \textbf{Estructura\break de Datos} & Formato de filas y columnas & Uso de reglas y relaciones \\
        \hline
        \textbf{Organización} & En celdas & En colecciones completas \\
        \hline
        \textbf{Acceso de Datos} & Cantidad limitada & Grandes cantidades \\
        \hline
        \textbf{Ingreso de Datos} & Manual & Escrito y Coherente \\
        \hline
        \textbf{Personas\break trabajando} & Generalmente un usuario a la vez & Múltiples usuarios \\
        \hline
        \textbf{Controlado por} & Usuario & Sistema de gestión de base de datos \\
        \hline
    \end{tabular}
\end{table}

\subsection{\textit{Data Viz} - Herramientas de Visualización}
\begin{itemize}
    \item {Permiten crear gráficos, tablas, mapas y cuadros}
    \item {Convierten números complejos en historias que las personas puedan entender}
    \item {Ayudan a interesados a sacar conclusiones que permiten tomar decisiones informadas y elaborar estrategias empresariales eficaces}
    \item {\textbf{Métrica} : Tipo de dato singular y cuantificable que puede utilizarse para medir algo
    \begin{itemize}
        \item {\textbf{Meta Métrica} : Meta propuesta por una compañía que puede ser medida y evaluada de acuerdo a métricas}
        \item {\textit{Ejemplo} : Retorno de Inversión = $\frac{Ganancia Neta}{Costo de Inversión}$}
    \end{itemize}}
    \item {\textbf{\textit{Pivot Table} - Tabla Pivote} : Utilizado en el procesamiento de datos, permite resumir, ordenar y reorganizar datos en un DB, para realizar cálculos como promedios o cantidades totales}
\end{itemize}

\subsubsection{\textit{Report} - Informes}
Recuento estático de datos, entregados de forma periódica a \Gls{stkhldrs}

\subsubsection{\textit{Dashboard} - Panel}
Monitoreo constante y en vivo de datos entrantes, con una ubicación central que los agrupa. El ingreso de datos es automático \textit{solo si la estructura de datos es la misma}
\paragraph{Proceso para crear un panel}
    \begin{itemize}
        \item {Identificar a \Gls{stkhldrs} que necesiten ver los datos y cómo los usarán, a través de preguntas efectivas}
        \item {Diseño del panel : encabezados claros para etiquetar información ; descripciones de texto concisas para cada visualización ; mostrar la información más importante en la parte superior}
        \item {Crear prototipos}
        \item {Seleccionar visualizaciones para utilizar en el panel : cambio de valores a lo largo del tiempo con gráficos de línea y barras ; contribución de cada parte al resultado final con gráficos circulares y de torta}
        \item {Crear filtros de ser necesario, identificar patrones mientras los datos originales se mantienen intactos}
    \end{itemize}
\paragraph{Tres categorías comunes de paneles}
    \begin{itemize}
        \item {\textbf{Paneles Estratégicos} : Toma de decisiones para la empresa en periodos más largos}
        \item {\textbf{Paneles Operativos} : Escala temporal de días, semanas o meses, prácticamente a tiempo real}
        \item {\textbf{Paneles Analíticos} : Uso, análisis y predicciones como categoría más técnica, realizadas por \textit{data scientists}}
    \end{itemize}

\begin{table}
    \centering
    \begin{tabular}{|p{2.9cm}|p{5.4cm}|p{5.5cm}|}
        \hline
        \multicolumn{3}{|c|}{Beneficios de paneles} \\
        \hline
        & \textbf{Para analistas} & \textbf{Para \Gls{stkhldrs}} \\
        \hline
        \small{\textbf{Centralización}} & Compartir una única fuente de datos con los interesados & Trabajar con visión integral de datos, iniciativas y procesos \\
        \hline
        \small{\textbf{Visualización}} & Mostrar y actualizar datos \break entrantes en tiempo real & Detectar patrones y tendencias cambiante más rápidamente \\
        \hline
         \small{\textbf{Percepción}} & Extraer información desde \break diferentes conjuntos de datos & Comprender la historia detrás de los números \\
        \hline
        \small{\textbf{Personalización}} & Crear vistas personalizadas para los interesados & Profundizar en preocupaciones o intereses específicos \\
        \hline
    \end{tabular}
\end{table}

\begin{table}
    \centering
    \begin{tabular}{|p{1.8cm}|p{5.9cm}|p{5.9cm}|}
        \hline
        & \textbf{\textit{Reports}} & \textbf{\textit{Dashboard}} \\
        \hline
        \textbf{Pros} & \begin{description}
            \item {Datos de alto nivel histórico}
            \item {Fácil de diseñar y usar}
            \item {Datos limpios y ordenados}
        \end{description} & \begin{description}
            \item {Automático e interactivo}
            \item {Mejor acceso para \Gls{stkhldrs}}
            \item {Bajo mantenimiento}
        \end{description} \\
        \hline
        \textbf{Contras} & \begin{description}
            \item {Mantenimiento contínuo}
            \item {Estático}
            \item {Menos atractivo visual}
        \end{description} & \begin{description}
            \item {Diseño requiere más trabajo}
            \item {Puede ser confuso}
            \item {Datos sin procesar}
        \end{description} \\
        \hline
    \end{tabular}
\end{table}
    

\subsection{Ejemplos y Otras Herramientas}
Algunos ejemplos consisten en :
\begin{itemize}
        \item {\textbf{Tableau} : \textit{arrastrar + soltar} para crear gráficos interactivos en paneles y hojas de trabajo}
        \item {\textbf{Looker} : Conexión directa con bases de datos}
\end{itemize}
Pueden usarse otras herramientas para análisis estadístico, visualizaciones y otras tareas
\begin{itemize}
    \item {R}
    \item {Python}
\end{itemize}

\newpage