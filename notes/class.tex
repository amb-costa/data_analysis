\section{Clasificación de Datos}
\subsection{Por Objetividad y Capacidad de Medición}
\begin{itemize}
    \item{\textbf{Cuantitativos} : Cuentan con hechos numéricos, medidas específicas y objetivas. \textit{Qué cosas, cuántas, qué tan seguido?} Se subdividen en dos tipos:
    \begin{itemize}
        \item{\textbf{Discretos} : Son cuantificables y tienen un número limitado de valores}
        \item{\textbf{Contínuos} : Se miden y pueden tener casi cualquier valor numérico}
    \end{itemize}}
    \item{\textbf{Cualitativos} : Medidas subjetivas o explanatorias para cualidades y características. \textit{Por qué? Cómo?} Estos datos también se dividen en
    \begin{itemize}
        \item{\textbf{Nominal} : No presentan estructura o secuencia aparente}
        \item{\textbf{Ordinal} : Pueden ordenarse en base a características}
    \end{itemize}}
\end{itemize}

\subsection{Por Cercanía a sus Fuentes}
\begin{itemize}
    \item {\textbf{\textit{$1^{\text{st}}$ party data} - de Primera Fuente} : Datos recolectados por un individuo o grupo, utilizando recursos propios}
    \item {\textbf{\textit{$2^{\text{nd}}$ party data} - de Segunda Fuente} : Datos recolectados por un grupo, directamente desde su audiencia, y que luego se venden}
    \item {\textbf{\textit{$3^{\text{rd}}$ party data} - de Tercera Fuente} : Datos recolectados desde fuentes externas, que no los capturaron directamente}
\end{itemize}

\subsection{Por Ubicación}
\begin{itemize}
    \item {\textbf{Internos} : Datos que viven dentro de los sistemas de la compañía o empresa}
    \item {\textbf{Externos} : Datos que viven o son generados fuera de la organización}
\end{itemize}

\subsection{Por Nivel de Organización}
\begin{itemize}
    \item {\textbf{Estructurados} : Corresponden a datos definidos, a menudo \textbf{cuantitativos}, organizados en formatos específicos como filas y columnas. Son fáciles de ordenar, buscar y analizar. \textit{Ejemplos : registros telefónicos, Excel, DB SQL}}
    \item {\textbf{No Estructurados} : Datos variados, a menudo \textbf{cualitativos}, que no muestran alguna manera fácil de organizar. Proporcionan libertad de análisis, pero son más difíciles de buscar. \textbf{Representan la mayoría de los datos en el mundo.} \textit{Ejemplos : DB NOSQL, SMS, fotos y videos, archivos}}
\end{itemize}

\subsection{Por Formato}
\begin{itemize}
    \item {\textbf{\textit{Wide Data} - Formato Ancho} : Cada elemento de dato tiene un solo registro o fila, con múltiples campos que contienen los valores para sus atributos. Preferir cuando
    \begin{itemize}
        \item {Se crean tablas o gráficos con pocas variables sobre cada tema}
        \item {Se comparan gráficos lineales sencillos}
    \end{itemize}}
    \item {\textbf{\textit{Long Data} - Formato Corto} : Cada registro se relaciona con un valor por atributo, así que cada elemento tendrá datos a través de múltiples registros. Preferir cuando
    \begin{itemize}
        \item {Se almacenan muchas variables sobre cada tema}
        \item {Se realizan análisis estadísticos avanzados o gráficos}
    \end{itemize}}
\end{itemize}

\subsection{Por Tamaño}
\begin{itemize}
    \item {\textbf{\textit{Small Data} - Microdatos} : Cantidad de datos pequeños y específicos, generalmente obtenidos en periodos cortos. Útiles para decisiones diarias}
    \item {\textbf{\textit{Big Data} - Macrodatos} : Cantidad de datos grandes y menos específicos, generalmente obtenidos a través de periodos largos
    \begin{itemize}
        \item {\paragraph{Desafíos : }
        \begin{itemize}
            \item {Sobrecarga de datos, información sin relevancia o importancia}
            \item {Datos importantes ocultos por datos no relevantes. Esto puede provocar demoras e ineficiencias en la toma de decisiones}
            \item {Los datos no siempre son fácilmente accesibles}
            \item {Debido al tamaño, no todas las soluciones tecnológicas pueden trabajar con macrodatos. Esto puede llevar a sesgos algorítmicos}
        \end{itemize}}
        \item {\paragraph{Beneficios}
        \begin{itemize}
            \item {Al almacenar y analizar grandes cantidades de datos, se pueden identificar formas más eficientes para llevar a cabo negocios y proyectos} 
            \item {Permiten detectar tendencias y patrones debido al periodo de tiempo en que son generados}
            \item {Presentan una comprensión más amplia del problema, lo que puede usarse para mantenerse por delante de la competencia}
        \end{itemize}}
        \item {\paragraph{Cuatro 'V' para resumir Beneficios y Desafíos}
        \begin{description}
            \item[Volumen] : Cantidad de datos
            \item[Variedad] : Diferentes tipos de datos
            \item[Velocidad] : Qué tan rápido se pueden procesar los datos
            \item[Veracidad] : Calidad y fiabilidad de los datos
        \end{description}}
    \end{itemize}}
\end{itemize}

\begin{table}
    \centering
    \begin{tabular}{|p{7cm}|p{7cm}|}
        \hline
        \multicolumn{2}{|c|}{Diferencias} \\
        \hline
        \textbf{\textit{Small Data}} & \textbf{\textit{Big Data}} \\
        \hline
        \begin{description}
            \item {Conjunto de datos de métricas específicas, generados durante un periodo de tiempo corto y definido}
            \item {Generalmente se trabajan en hojas de cálculo}
            \item {Es probable que los utilicen pequeñas y medianas empresas}
            \item {Fáciles de recopilar, almacenar, trabajar y representar visualmente}
            \item {Tamaño manejable para análisis}
        \end{description} & \begin{description}
            \item {Conjunto de datos más amplio, menos específico, que cubre periodos de tiempo más extensos}
            \item {Generalmente se mantienen en DB}
            \item {Es probable que lo utilicen grandes organizaciones}
            \item {Requiere mucho esfuerzo para recopilar, almacenar, administrar, clasificar y representar visualmente}
            \item {Se deben dividir en porciones más pequeñas antes de organizar y analizarlos}
        \end{description} \\
        \hline
    \end{tabular}
\end{table}
\newpage

