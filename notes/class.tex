\section{Clasificación de Datos}
\subsection{Por Objetividad y Capacidad de Medición}
\begin{itemize}
    \item{\textbf{Cuantitativos} : Cuentan con hechos numéricos, medidas específicas y objetivas. \textit{Qué cosas, cuántas, qué tan seguido?} Se subdividen en dos tipos:
    \begin{itemize}
        \item{\textbf{Discretos} : Son cuantificables y tienen un número limitado de valores}
        \item{\textbf{Contínuos} : Se miden y pueden tener casi cualquier valor numérico}
    \end{itemize}}
    \item{\textbf{Cualitativos} : Medidas subjetivas o explanatorias para cualidades y características. \textit{Por qué? Cómo?} Estos datos también se dividen en
    \begin{itemize}
        \item{\textbf{Nominal} : No presentan estructura o secuencia aparente}
        \item{\textbf{Ordinal} : Pueden ordenarse en base a características}
    \end{itemize}}
\end{itemize}

\subsection{Por Cercanía a sus Fuentes}
\begin{itemize}
    \item {\textbf{\textit{$1^{\text{st}}$ party data} - de Primera Fuente} : Datos recolectados por un individuo o grupo, utilizando recursos propios}
    \item {\textbf{\textit{$2^{\text{nd}}$ party data} - de Segunda Fuente} : Datos recolectados por un grupo, directamente desde su audiencia, y que luego se venden}
    \item {\textbf{\textit{$3^{\text{rd}}$ party data} - de Tercera Fuente} : Datos recolectados desde fuentes externas, que no los capturaron directamente}
\end{itemize}

\subsection{Por Ubicación}
\begin{itemize}
    \item {\textbf{Internos} : Datos que viven dentro de los sistemas de la compañía o empresa}
    \item {\textbf{Externos} : Datos que viven o son generados fuera de la organización}
\end{itemize}

\subsection{Por Nivel de Organización}
\begin{itemize}
    \item {\textbf{Estructurados} : Corresponden a datos definidos, a menudo \textbf{cuantitativos}, organizados en formatos específicos como filas y columnas. Son fáciles de ordenar, buscar y analizar. \textit{Ejemplos : registros telefónicos, Excel, DB SQL}}
    \item {\textbf{No Estructurados} : Datos variados, a menudo \textbf{cualitativos}, que no muestran alguna manera fácil de organizar. Proporcionan libertad de análisis, pero son más difíciles de buscar. \textbf{Representan la mayoría de los datos en el mundo.} \textit{Ejemplos : DB NOSQL, SMS, fotos y videos, archivos}}
\end{itemize}

\subsection{Por Formato}
\begin{itemize}
    \item {\textbf{\textit{Wide Data} - Formato Ancho} : Cada elemento de dato tiene un solo registro o fila, con múltiples campos que contienen los valores para sus atributos. Preferir cuando
    \begin{itemize}
        \item {Se crean tablas o gráficos con pocas variables sobre cada tema}
        \item {Se comparan gráficos lineales sencillos}
    \end{itemize}}
    \item {\textbf{\textit{Long Data} - Formato Corto} : Cada registro se relaciona con un valor por atributo, así que cada elemento tendrá datos a través de múltiples registros. Preferir cuando
    \begin{itemize}
        \item {Se almacenan muchas variables sobre cada tema}
        \item {Se realizan análisis estadísticos avanzados o gráficos}
    \end{itemize}}
\end{itemize}

\subsection{Por Tamaño}
\begin{itemize}
    \item {\paragraph{\textit{Small Data} - Microdatos}
    \begin{itemize}
        \item {Específicos}
        \item {Periodos cortos}
        \item {Útiles para decisiones diarias}
    \end{itemize}}
    \item {\paragraph{\textit{Big Data} - Macrodatos}
    \begin{itemize}
        \item {Grandes y menos específicos}
        \item {Periodos largos}
        \item {Útiles para decisiones diarias}
        \item {\paragraph{Desafíos del \textit{Big Data}}
        \begin{description}
            \item {Muchas organizaciones se enfrentan a la sobrecarga de muchos datos e información sin importancia o irrelevante}
            \item {Los datos importantes se pueden ocultar en el fondo con todos los datos no relevantes, lo que hace más difícil encontrarlos y usarlos. Esto puede provocar plazos de toma de decisiones más lentos e ineficientes}
            \item {Los datos no siempre son fácilmente accesibles}
            \item {Las herramientas y soluciones tecnológicas actuales todavía luchanb por proporcionar datos medibles que se puedan informar. Esto puede conducir a un sesgo algorítmico injusto}
            \item {Hay brechas en muchas soluciones empresariales de \textit{Big Data}}
        \end{description}}
        \item {\paragraph{Beneficios}
        \begin{description}
            \item {Cuando se pueden almacenar y analizar grandes cantidades de datos, esto puede ayudar a las empresas a identificar formas más eficientes de hacer negocios y ahorrar tiempo y dinero} 
            \item {Los macrodatos ayudan a las organizaciones a detectar las tendencias de los patrones de compra de los clientes y los niveles de satisfacción, lo cual puede ayudarlas a crear nuevos productos y soluciones que harán felices a los clientes}
            \item {Al analizar los macrodatos, las empresas obtienen una comprensión mucho mejor de las condiciones actuales del mercado, lo cual puede ayudarlas a mantenerse por delante de la competencia}
            \item {Ayudan a las empresas a realizar un seguimiento de su presencia en línea, en especial de los comentarios, tanto buenos como malos, de los clientes. Esto les da información que necesitan para mejorar y proteger su marca}
        \end{description}}
        \item {\paragraph{Cuatro 'V' para resumir Beneficios y Desafíos}
        \begin{description}
            \item[Volumen] : Cantidad de datos
            \item[Variedad] : Diferentes tipos de datos
            \item[Velocidad] : Qué tan rápido se pueden procesar los datos
            \item[Veracidad] : Calidad y fiabilidad de los datos
        \end{description}}
    \end{itemize}}
\end{itemize}
\begin{table}
    \centering
    \begin{tabular}{|p{7cm}|p{7cm}|}
        \hline
        \multicolumn{2}{|c|}{Diferencias} \\
        \hline
        \textbf{\textit{Small Data}} & \textbf{\textit{Big Data}} \\
        \hline
        \begin{description}
            \item {Describe un conjunto de datos compuesto por métricas específicas durante un periodo de tiempo corto y bien definido}
            \item {Por lo general se organizan y analizan en hojas de cálculo}
            \item {Es probable que los utilicen pequeñas y medianas empresas}
            \item {Fáciles de recopilar, almacenar, administrar, ordenar y representar visualmente}
            \item {Por lo general ya tienen un tamaño manejable para análisis}
        \end{description} & \begin{description}
            \item {Describe un conjunto de datos más amplio y menos específico que cubre un periodo extenso de tiempo}
            \item {Por lo general se mantienen en una DB y se consultan}
            \item {Es probable que lo utilicen grandes organizaciones}
            \item {Requiere mucho esfuerzo para recopilar, almacenar, administrar, clasificar y representar visualmente}
            \item {Por lo general, deben dividirse en porciones más pequeñas para organizarlos y analizarlos de manera efectiva para tomar decisiones}
        \end{description} \\
        \hline
    \end{tabular}
\end{table}
\newpage

