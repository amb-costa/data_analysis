\section{Ciclo de Vida de los Datos}

\textit{No confundir con análisis, solamente datos}

\begin{itemize}
    \item {\textbf{\textit{Planning} - Planificación} :
    \begin{itemize}
        \item {Características de los datos a tomar: qué tipo, cómo obtenerlos, quién los administra, resultados óptimos}
        \item {Qué tipos de datos se necesitan? Cómo se gestionarán? Quién será el responsable de ellos}
    \end{itemize}}
    \item {\textbf{\textit{Capture} - Captura} :
    \begin{itemize}
        \item {Recolección de datos desde distintas fuentes como DB, Archivos, Encuestas...}
    \end{itemize}}
    \item {\textbf{\textit{Manage} - Mantención} :
    \begin{itemize}
        \item {Cuidado de los datos, almacenamiento y seguridad}
        \item {Cómo y dónde se almacenan? Qué herramientas se usan para hacerlo?}
        \item {\textit{No corresponde limpieza, ya que es una tarea del analista y no una parte del ciclo de vida}}
    \end{itemize}}
    \item {\textbf{\textit{Analyze} - Análisis} :
    \begin{itemize}
        \item {Resolución del problema utilizando los datos}
        \item {Uso de los datos para respaldar objetivos empresariales y tomar decisiones}
    \end{itemize}}
    \item {\textbf{\textit{Archive} - Archivar} :
    \begin{itemize}
        \item {Almacenar los datos en un lugar donde estén disponible para referencias futuras y a largo plazo}
        \item {Pueden o no ser usados de nuevo}
    \end{itemize}}
    \item {\textbf{\textit{Destroy} - Destrucción} :
    \begin{itemize}
       \item {Eliminar los datos y las copias compartidas}
       \item {Se destruye la documentación en papel, o se realiza limpieza de discos duros, servidores y bases de datos}
    \end{itemize}}
\end{itemize}

% Creando tabla de ejemplos para otros tipos de ciclo de vida de datos
% 3 columnas: nombre de institución, lista para ciclo de vida, info extra
% 5 filas: una de atributos, cuatro ejemplos
\begin{table}
\centering
\scriptsize
\begin{tabular}{|p{0.05\textwidth}|p{0.15\textwidth}|p{0.17\textwidth}|p{0.15\textwidth}|p{0.18\textwidth}|} 
\hline
 & {Servicio de Pesca y \break Vida Silvestre (USA)} & \href{https://www.usgs.gov/data-management/data-lifecycle}{Servicio Geológico de EEUU (USGS)} & \href{https://www.sfmagazine.com/articles/2018/july/the-data-life-cycle/}{Strategic Finance \break Magazine} & \href{https://online.hbs.edu/blog/post/data-life-cycle}{Escuela de Negocios de Harvard (HBS)} \\ 
\hline
Ciclo\break de\break Vida & 
\begin{description}
    \item Planificar
    \item Adquirir
    \item Mantener
    \item Acceder
    \item Evaluar
    \item Archivar
\end{description} & \begin{description}
    \item Planificar
    \item Adquirir
    \item Procesar
    \item Analizar
    \item Preservar
    \item Publicar/Compartir
\end{description} & \begin{description}
    \item Capturar
    \item Calificar
    \item Transformar
    \item Utilizar
    \item Informar
    \item Archivar
    \item Depurar
\end{description} & \begin{description}
    \item Generación
    \item Recolección
    \item Procesamiento
    \item Almacenamiento
    \item Gestión
    \item Análisis
    \item Visualización
    \item Interpretación
\end{description} \\ 
\bottomrule
\end{tabular}
\end{table}

{\footnotesize {USGS realiza otras actividades transversales, como Describir (metadatos y documentación), y Gestionar calidad y copias de seguridad}}

\newpage