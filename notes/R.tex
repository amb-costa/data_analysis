% nta: 
% introducción a los lenguajes de programación
% exploración de las características y funciones principales
% conceptos básicos de programación en R
% cómo trabajar con información en R
% limpiar, transformar, visualizar, reportar datos en R

\section{Programación en R}
R es un lenguaje de programación utilizado con frecuencia para \textbf{Reproducción de Análisis, Procesamiento de grandes cantidades de datos, y Creación de visualizaciones}, basado en S. Es \textit{Case-Sensitive}. Es \textbf{Accesible, Centrado en Datos, \gls{opnsrc}, y tiene una comunidad de usuarios}. Para usar R, se deben instalar paquetes, los que se invocan como una librería
\begin{itemize}
    \item {\textit{install.package("package\_name")}}
    \item {\textit{library("package\_name")}}
\end{itemize}
Los objetos fundamentales de R son 
\begin{itemize}
    \item {\textbf{Funciones}: secciones de código reutilizable, usados para realizar tareas en específico}
    \item {\textbf{Comentarios - \#}}
    \item {\textbf{Variables}: representación de un valor que se puede guardar para usos futuros. Se asignan con el \textbf{operador de asignación <-}. \textit{siempre debieran empezar con una letra}}
    \item {\textbf{\gls{datatype}}}
    \item {\textbf{Vectores}: grupo de elementos de datos del mismo tipo, guardados en secuencia \textit{c(v1,v2,...)}}
    \item {\textbf{Pipas}: herramienta para expresar una secuencia de operaciones, representado como \textit{\%>\%}}
\end{itemize}
Las estructuras de datos más comunes son
\begin{itemize}
    \item {\textbf{Vectores}: Grupo de elementos de datos del mismo tipo, guardados en secuencia. \textit{c(v1,v2,...)} crea un vector, con sus valores separados por comas
    \begin{itemize}
        \item {Atómicos: no se puede tener un vector de valores numéricos y lógicos. \textit{Seis tipos primarios: lógicos (bool), enteros (integer, positivos y negativos), dobles (decimales), caracter(strings), complejos y sin formato}}
        \item {Listas}
    \end{itemize}}
    \item {Marcos de datos}
    \item {Matrices}
    \item {Rangos}
\end{itemize}
Algunas de las funciones más comunes para vectores, y que se comparten en varios lenguajes, son 
\begin{itemize}
    \item {typeof()}
    \item {length()}
    \item {is.logical()}
    \item {is.double()}
    \item{is.integer()}
    \item{is.character()}
\end{itemize}

\subsection{Conceptos Básicos}
La \textbf{Programación Computacional} consiste en entregar instrucciones a una computadora para llevar una acción (o un grupo de ellas) a cabo. Un \textbf{Lenguaje de Programación} corresponde a las palabras y símbolos que utilizamos para escribir instrucciones, de forma que un computador las siga. Esto permite \textbf{clarificar pasos del análisis, ahorrar tiempo, y reproducir y compartir el trabajo.}\textit{ej: R, Python, Javascript}
\begin{itemize}
    \item {\textbf{IDE}: aplicación ligera que reúne varias herramientas a usar, en un lugar central}
    \item {\textbf{Consola} : área donde se entregan comandos al lenguaje. Se borra cuando el lenguaje se cierra, por lo que se necesita usar un editor para guardar comandos necesarios}
\end{itemize}

\newpage
