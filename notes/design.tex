% nta : esto debiera ir en share pero el capítulo se vuelve muy extenso
% mejor separar lo estético en un capítulo y lo técnico otro

\section{Diseño}
\subsection{Atributos Preatencionales}
Elementos de una visualización cualquiera que las personas reconocen automáticamente, sin realizar un esfuerzo consciente. \textit{No se limita a visualización de datos}
\begin{itemize}
    \item {\textbf{Marcas}: objetos visuales básicos como puntos, líneas o formas, donde cada uno se descompone en \textbf{tamaño, forma, color y posición} (del objeto en el espacio, en relación a una escala u otro objeto)}
    \item {\textbf{Canales}: aspectos o variables que caracterizan a los datos, la marca utilizada para visualizarlo. La efectividad de la marca depende de
    \begin{description}
        \item {\textbf{Exactitud}: utilidad para estimar con exactitud los valores representados}
        \item {\textbf{Destaque}: qué tan fácil se distingue un dato del otro}
        \item {\textbf{Agrupamiento}: comunicar sobre grupos que puedan existir en los datos}
    \end{description}}
\end{itemize}

\subsection{Principios básicos de Diseño}
Además de elegir el elemento visual correcto, se debe optimizar la proporción dato-trazo: \textit{enfocarse en la parte del elemento visual que es esencial para comprender el sentido del gráfico}. Se debe utilizar la orientación de manera eficaz, al igual que la cantidad de cosas y los colores a elegir
\begin{itemize}
    \item {\textbf{Equilibrio}: distribución pareja de la información}
    \item {\textbf{Énfasis}: punto focal en los datos más importantes, para la concentración del público}
    \item {\textbf{Movimiento}: imitar la manera en que las personas leen}
    \item {\textbf{Patrón}: crear o romper}
    \item {\textbf{Repetición}: añade eficacia}
    \item {\textbf{Proporción}: de acuerdo a la importancia}
    \item {\textbf{Ritmo}: sensación de movimiento o flujo}
    \item {\textbf{Variedad}: mantiene el interés}
    \item {\textbf{Unidad}: el \textit{DataViz} debe ser cohesivo}
\end{itemize}

\subsection{\textit{Design Thinking}}
Proceso utilizado para resolver problemas complejos desde el foco del usuario. Las cinco fases representan una descripción general, no es necesario el seguir un orden
\begin{itemize}
    \item {\textbf{Empatizar} con las emociones y necesidades de la audiencia}
    \item {\textbf{Definir} las necesidades de la audiencia (de la mano con empatizar)}
    \item {\textbf{Idear}, generar ideas y soluciones} 
    \item {\textbf{Prototipar} paneles, diagramas y otras visualizaciones}
    \item {\textbf{Testear} y obtener comentarios de parte de los \gls{stkhldrs}}
\end{itemize}

\subsection{Tableau}
Tableau es una plataforma de análisis y de inteligencia de negocios (BI), que permite visualizar datos, principalmente a través de \textbf{Visualizaciones Dinámicas}: visualizaciones interactivas o que cambian con el tiempo

Además de los elementos de \textit{DataViz} más comunes, Tableau ofrece \textbf{tablas resaltadas (con formato condicional), mapas térmicos, mapas de densidad, diagramas gantt}, entre otros. Además, similar a SQL, permite combinar datos a través de JOINs

\newpage