\section{Tipos de Datos}

\subsection{Por Objetividad y Capacidad de Medición}
\begin{itemize}
    \item {\textbf{Cuantitativos} : Medidas específicas y objetivas con hechos numéricos
    \begin{itemize}
        \item {Qué cosas? Cuántas? Qué tan seguido?}
        \item {Medibles a través de tablas, gráficos y visualizaciones}
    \end{itemize}}
    \item {\textbf{Cualitativos} : Medidas subjetivas o explanatorias para cualidades y características
    \begin{itemize}
        \item {Por qué? Cómo?}
        \item {Añade contexto}
    \end{itemize}}
\end{itemize}

\subsection{Por Tamaño}
\begin{itemize}
    \item {\paragraph{\textit{Small Data} - Microdatos}
    \begin{itemize}
        \item {Específicos}
        \item {Periodos cortos}
        \item {Útiles para decisiones diarias}
    \end{itemize}}
    \item {\paragraph{\textit{Big Data} - Macrodatos}
    \begin{itemize}
        \item {Grandes y menos específicos}
        \item {Periodos largos}
        \item {Útiles para decisiones diarias}
        \item {\paragraph{Desafíos del \textit{Big Data}}
        \begin{description}
            \item {Muchas organizaciones se enfrentan a la sobrecarga de muchos datos e información sin importancia o irrelevante}
            \item {Los datos importantes se pueden ocultar en el fondo con todos los datos no relevantes, lo que hace más difícil encontrarlos y usarlos. Esto puede provocar plazos de toma de decisiones más lentos e ineficientes}
            \item {Los datos no siempre son fácilmente accesibles}
            \item {Las herramientas y soluciones tecnológicas actuales todavía luchanb por proporcionar datos medibles que se puedan informar. Esto puede conducir a un sesgo algorítmico injusto}
            \item {Hay brechas en muchas soluciones empresariales de \textit{Big Data}}
        \end{description}}
        \item {\paragraph{Beneficios}
        \begin{description}
            \item {Cuando se pueden almacenar y analizar grandes cantidades de datos, esto puede ayudar a las empresas a identificar formas más eficientes de hacer negocios y ahorrar tiempo y dinero} 
            \item {Los macrodatos ayudan a las organizaciones a detectar las tendencias de los patrones de compra de los clientes y los niveles de satisfacción, lo cual puede ayudarlas a crear nuevos productos y soluciones que harán felices a los clientes}
            \item {Al analizar los macrodatos, las empresas obtienen una comprensión mucho mejor de las condiciones actuales del mercado, lo cual puede ayudarlas a mantenerse por delante de la competencia}
            \item {Ayudan a las empresas a realizar un seguimiento de su presencia en línea, en especial de los comentarios, tanto buenos como malos, de los clientes. Esto les da información que necesitan para mejorar y proteger su marca}
        \end{description}}
        \item {\paragraph{Cuatro 'V' para resumir Beneficios y Desafíos}
        \begin{itemize}
            \item{\textbf{Volumen} : Cantidad de datos}
            \item{\textbf{Variedad} : Diferentes tipos de datos}
            \item{\textbf{Velocidad} : Qué tan rápido se pueden procesar los datos}
            \item{\textbf{Veracidad} : Calidad y fiabilidad de los datos}
        \end{itemize}}
    \end{itemize}}
\end{itemize}
\begin{table}
    \centering
    \begin{tabular}{|p{7cm}|p{7cm}|}
        \hline
        \multicolumn{2}{|c|}{Diferencias} \\
        \hline
        \textbf{\textit{Small Data}} & \textbf{\textit{Big Data}} \\
        \hline
        \begin{itemize}
            \item {Describe un conjunto de datos compuesto por métricas específicas durante un periodo de tiempo corto y bien definido}
            \item {Por lo general se organizan y analizan en hojas de cálculo}
            \item {Es probable que los utilicen pequeñas y medianas empresas}
            \item {Fáciles de recopilar, almacenar, administrar, ordenar y representar visualmente}
            \item {Por lo general ya tienen un tamaño manejable para análisis}
        \end{itemize} & \begin{itemize}
            \item {Describe un conjunto de datos más amplio y menos específico que cubre un periodo extenso de tiempo}
            \item {Por lo general se mantienen en una DB y se consultan}
            \item {Es probable que lo utilicen grandes organizaciones}
            \item {Requiere mucho esfuerzo para recopilar, almacenar, administrar, clasificar y representar visualmente}
            \item {Por lo general, deben dividirse en porciones más pequeñas para organizarlos y analizarlos de manera efectiva para tomar decisiones}
        \end{itemize} \\
        \hline
    \end{tabular}
\end{table}
\newpage

