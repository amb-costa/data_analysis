% separando subsecciones por motivos de orden
\subsection{\textit{DB + Queries} - Bases de Datos y Consultas}
\begin{itemize}
    \item {Aislar información específica de una base de datos}
    \item {Facilitar aprendizaje y comprensión de las solicitudes/consultas}
    \item {Seleccionar, crear, agregar o descargar datos desde la Base de Datos para Análisis}
    \item {\textbf{\textit{Primary Key} - Clave Primaria} : Identificador que referencia una columna donde cada uno de sus valores es único
    \begin{itemize}
        \item {Se utiliza para asegurar que los datos en una columna en específico son únicos}
        \item {No admite valores nulos o blancos}
        \item {Solo se permite una llave primaria en una tabla. Además, las tablas no siempre requieren tener una clave primaria}
        \item {Es posible construir una clave primaria a partir de varias columnas de una tabla: \textbf{Clave Compuesta}}
    \end{itemize}}
    \item {\textbf{\textit{Foreign Key} - Clave Externa} : Campo dentro de una tabla que es una llave primaria en otra.
    \begin{itemize}
        \item {Columna (o grupo de) que permite conectar los datos entre dos tablas}
        \item {Como referencia a la clave primaria de otra tabla, permite conectarlas}
        \item {Puede existir más de una clave externa en una tabla}
    \end{itemize}}
    \item {\textbf{Tipos de Base de Datos : }
    \begin{itemize}
        \item {\textbf{Relacional} : Bases de datos que contienen un grupo de tablas que pueden conectarse entre si utilizando sus relaciones o lo que tengan en común. Generalmente se relacionan con \textit{SQL - Structured Query Language} \textbf{Si un campo se usa en dos tablas, esto puede usarse para conectarlas.} \textit{Ejemplos de dialectos SQL: MySQL, PostgreSQL, Big Query}}
        \item {\textbf{No Relacional} : Bases de datos donde todas las variables posibles a analizar, se agrupan conjuntamente. Esto las hace difícil de clasificar. Generalmente relacionadas con dialectos \textit{NOSQL}}
    \end{itemize}}
\end{itemize}


