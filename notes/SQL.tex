% nta : separado para desglosar el capítulo de herramientas
% se hace necesario tener más secciones 

\section{SQL - \textit{Structured Query Language}}
SQL origina desde \textit{System R/IBM}, y está directamente relacionado con DB relacionales. Permite trabajar con conjuntos de datos grandes al acceder a tablas de DB, en tareas como preparación para análisis posteriores en otro software. Es ideal para el trabajo colaborativo y el seguimiento de \gls{qry} por todos los usuarios.

Tomar en cuenta que existen múltiples dialectos SQL y es recomendable consultar la documentación. Sin embargo, la gran mayoría de las instrucciones son comunes, y se promueven buenas prácticas de comandos. \textit{Ejemplos de Dialectos SQL : MySQL, BigQuery, PostgreSQL, Microsoft SQL, SQLite}

También considerar que SQL no guarda los resultados de los \gls{qry} como tablas, por lo que debe crearse una si es que eso es necesario.

\paragraph{Términos comunes}
\begin{description}
    \item {\textbf{*} : selector universal para todas las columnas}
    \item {\textbf{Null} : Indicador de valores no existentes en un set de datos. \textit{null ${\neq}$ 0}} 
\end{description}

\subsection{Cláusulas, Comandos y Funciones}
\subsubsection{Data Definition Language}
Definen y modifican la estructura de una DB, su esquema u objetos como índices y funciones
\begin{itemize}
    \item {\textbf{CREATE} [structure] strct\_name : crea una estructura a definir junto con sus características. \textit{Ej : CREATE TABLE tableName (col1 dt1, col2 dt2....) : define una tabla junto a pares de columnas y su \gls{datatype} correspondiente}}
    \item {\textbf{ALTER} [structure] strct\_name {action}: modifica la estructura de TABLE, DATABASE, entre otras estructuras. \textit{Ej : ALTER TABLE tableName ADD COLUMN col\_name data\_type}}
    \item {\textbf{DROP} [structure] strct\_name :  elimina la estructura de DB. \textit{Ej : DROP TABLE tableName}}
    \item {\textbf{TRUNCATE}}
    \item {\textbf{COMMENT}}
    \item {\textbf{RENAME} [structure] name1 \textbf{TO} name2 : renombra un objeto de DB}
\end{itemize}

\subsubsection{Data Manipulation Language}
Controlan el acceso y manejo a los datos del DB.
\begin{itemize}
    \item {\textbf{INSERT INTO} tableName (col1,col2,...) \textbf{VALUES} (val1,val2,...) : los valores deben seguir el orden definido en el insert, separados por comas y encerrados en paréntesis}
    \item {\textbf{UPDATE} tableName \textbf{SET} value = ''new value'' \textbf{WHERE} {conditions}}
    \item {\textbf{DELETE}}
    \item {\textbf{LOCK}}
\end{itemize}

\subsubsection{Data Control Language}
Controlan los derechos, permisos y restricciones dentro de la DB
\begin{itemize}
    \item {\textbf{GRANT}}
    \item {\textbf{REVOKE}}
\end{itemize}

\subsubsection{Transaction Control Language}
Definen las tareas para llevar a cabo una transacción en específico, que solo puede tener dos resultados : exitosa o no. 
\begin{itemize}
    \item {\textbf{BEGIN TRANSACTION}}
    \item {\textbf{COMMIT}}
    \item {\textbf{ROLLOUT}}
    \item {\textbf{SAVEPOINT}}
\end{itemize}

\subsubsection{Comandos}
Instrucciones que se añaden a las cláusulas previamente mencionadas
\begin{itemize}
    \item {\textbf{SELECT} : \textit{Data Query Language}, realiza la consulta en si}
    \item {\textbf{FROM} tableName}
    \item {\textbf{WHERE} : filtro para incorporar condiciones tipo \textit{WHEN}, etc}
    \item {object \textbf{AS} alias : asigna un alias al objeto descrito. También se utiliza en otras funciones para asignar características}
\end{itemize}

\subsubsection{Funciones}
Acciones a llevar a cabo, generalmente se añaden en otras cláusulas para realizar filtros
\begin{itemize}
    \item {\textbf{LENGTH} (col) : devuelve el tamaño del \gls{str}}
    \item {\textbf{SUBSTR} (col,start,amount) : crea un substring en base a un punto de partida y una cantidad de caracteres}
    \item {\textbf{TRIM} (col) : remueve espacios extra}
    \item {\textbf{CAST} (col AS data\_type) : realiza \gls{typecast}}
    \item {\textbf{CONCAT} (substr1, substr2, ...) AS full\_str : une dos o más substrings en una columna}
    \item {SELECT \textbf{DISTINCT} : se realiza el query eliminando duplicados en la columna}
    \item {SELECT \textbf{COALESCE} (col1, col2,...) \textbf{AS} new\_col : entrega resultados no \textit{null} y los asigna a un registro nuevo. Si registro es \textit{null}, devuelve el siguiente campo que no lo sea}
    \item {SELECT col \textbf{CASE} \textbf{WHEN} cond1 \textbf{THEN} action1 \textbf{ELSE} cond2 \textbf{END AS} newcol : recorre múltiples condiciones a definir, y entrega un valor cuando una condición se cumple. \textit{Similar a if/else en otros lenguajes de programación}}
\end{itemize}


\subsection{Buenas Prácticas de Sintáxis}
\begin{itemize}
    \item {MAYÚSCULAS para iniciadores/cláusulas y funciones}
    \item {minúsculas para nombres de columnas. evitar mezclas}
    \item {'comillas simples' para strings}
    \item {${"}$comillas dobles${"}$ si el string contiene comillas simples/'apóstrofos'}
    \item {${--}$realizar comentarios para revisiones a futuro}
    \item{/* Para comentarios muy extensos, considerar este formato */}
    \item {\textbf{\textit{snake\_case}} como reemplazo de los espacios en los nombres\_de\_columnas}
    \item {\textbf{\textit{CamelCase}} para nombrar tablas. laPrimeraMayúscula no es obligatoria}
    \item {Uso de sangrías para mantener cada línea en menos de 100 caracteres}
    \item {Considerar el uso de editores de texto SQL, que mejoran la legibilidad a través de ediciones como colores y sangrías, además de aceptar \gls{rgx}}
\end{itemize}

\paragraph{Excepciones}
\begin{itemize}
    \item {Generalmente el uso de mayúsculas/minúsculas no afecta al código. BigQuery si distingue entre mayúsculas/minúsculas, a diferencia de otros dialectos}
    \item {Lo mismo para comillas simples o dobles, pero las buenas prácticas garantizan buen funcionamiento a través de todos los dialectos}
    \item {MySQL acepta \#etiqueta para comentarios}
\end{itemize}