% estos son los temas a tocar
% data integrity
% sample size
% random sampling
% testing data
% data cleaning techniques for spreadsheets + DB
% verifying + reporting (documenting) cleaning results

\section{\textit{Process} - Procesar}
Después de ser recopilados, los datos pasan por una \textbf{limpieza} para arreglar cualquier limitación como errores, duplicados o espacios vacíos, y manejar imprecisiones, casos extremos y valores desalineados. También se comprueba la ausencia de sesgos
\begin{itemize}
    \item {\textbf{Datos Duplicados} : el mismo dato aparece en más de una fila}
    \item {\textbf{Datos Indirectos} : datos con características similares a los datos reales}
    \item {\textbf{Fusión de Datos} : proceso de combinación de dos o más sets de datos en uno. Es importante que los sets sean \textbf{compatibles}, es decir, que puedan trabajar entre si}
    \item {\textbf{Mapeo de Datos} : trazabilidad de campos desde una fuente de datos a otra, para fines de compatibilidad}
\end{itemize}

\subsection{Datos Sucios}
Los datos sucios son incompletos, incorrectos y/o irrelevantes para el problema formulado
\paragraph{Tipos de datos sucios más comunes}
\begin{itemize}
    \item {\textbf{Desactualizados} : Existen valores más actualizados y exactos}
    \item {\textbf{Duplicados} : Existen registros repetidos}
    \item {\textbf{Incompletos} : Carecen de campos importantes}
    \item {\textbf{Incorrectos, Inexactos}}
    \item {\textbf{Desordenados} : Diferentes formatos para el mismo valor}
    \item {\textbf{Provenientes de una sola fuente}}
    \item {\textbf{Actualizados constantemente}}
    \item {\textbf{Limitados por geografía}}
\end{itemize}

\paragraph{Situaciones de posible Vulneración de Integridad}
\begin{itemize}
    \item {\textbf{Replicación/Duplicación} : los datos se guardan en diferentes ubicaciones a la vez}
    \item {\textbf{Transferencia} : los datos son copiados desde un dispositivo de almacenamiento a una memoria, o desde un computador a otro}
    \item {\textbf{Manipulación} : los datos se modifican para que sean más fáciles de organizar y leer} 
    \item {\textbf{Amenazas Informáticas} : virus, \textit{malware}, hacking, fallas de sistemas, entre otras}
    \item {\textbf{Error Humano} : causa principal de mala calidad de datos}
\end{itemize}

\subsection{Limpieza e Integridad de los Datos}
Cuando los datos están íntegros, se asegura que son exactos, completos, consistentes y confiables a lo largo de su ciclo de vida: un análisis exitoso depende de esta integridad. Se recomienda realizar una \textbf{copia de los datos} antes de iniciar cualquier proceso de limpieza, orden o filtrado. El proceso de limpiado incluye pero no se limita a 
\begin{itemize}
    \item {Remover duplicados}
    \item {Remover espacios extra o en blanco}
    \item {Arreglar faltas de ortografía, puntuación incorrecta, mayúsculas inconsistentes, entre otras fallas tipográficas}
    \item {Remover formatos}
\end{itemize}
\paragraph{Cómo trabajar con Datos Incompletos}
\begin{itemize}
    \item {Identificar patrones en los datos ya disponibles}
    \item {Si el tiempo lo permite, esperar para recolectar más datos}
    \item {Conversar con \gls{stkhldrs} y ajustar los objetivos}
    \item {Buscar otros sets de datos que puedan funcionar. \textit{Posiblemente realizar Fusión de Datos}}
\end{itemize}

\subsubsection{Errores Comunes de Limpieza}
\begin{itemize}
    \item {No verificar errores ortográficos}
    \item {No documentar errores}
    \item {No verificar valores de campos erróneos}
    \item {Pasar por alto valores faltantes}
    \item {Mirar solo un subconjunto de datos}
    \item {Perder de vista los objetivos empresariales}
    \item {No corregir la fuente de los errores}
    \item {No analizar el sistema antes de limpiar los datos}
    \item {No realizar copias de seguridad antes de empezar la limpieza}
    \item {No contemplar la limpieza en plazos y procesos}
\end{itemize}


\subsection{Validación de Datos}
La validación es una herramienta de revisión de exactitud y calidad de los datos antes de añadirlos o importarlos. Se utilizan criterios para los datos, y la validez se determina cuando los valores cumplen (o no) con las restricciones establecidas. 

\subsubsection{Condiciones de Validación}
\begin{description}
    \item [\gls{datatype}]{ : formato correcto para los valores}
    \item [Rango]{ : los valores se encuentran dentro de un rango de máximos y mínimos predefinidos}
    \item [Obligatoriedad]{ : los valores no pueden ser nulos, blancos o vacíos}
    \item [Unicidad]{ : ningún valor está duplicado}
    \item [Largo de Campo]{ : el largo de un campo está dentro del rango de caracteres esperado}
    \item [\gls{rgx}]{ : los valores coinciden con patrones preestablecidos}
    \item [Campos Cruzados]{ : los valores cumplen simultáneamente con restricciones múltiples}
    \item [Membresía]{ : los valores en una columna provienen de un conjunto de valores específicos} 
    \item [Exactitud]{ : se alcanza un grado de conformidad establecido, respecto a la entidad real medida}
    \item [Exhaustividad]{ : grado en que los datos contienen todos los componentes deseados}
    \item [Coherencia]{ : grado de repetibilidad de los datos desde diferentes puntos de entrada o recopilación}
\end{description}

\subsubsection{Proceso de Verificación}
Al verificar el proceso de limpieza, se confirma su calidad: si fue bien ejecutada, si los datos resultantes son precisos y confiables. Si es necesario, se realiza otra limpieza manual. Algunos pasos de verificación son
\begin{itemize}
    \item {Comparar datos limpios vs. sucios}
    \item {Ver el proyecto desde una perspectiva general, revisar cómo los datos podrían resolver las metas y problemas del proyecto. \textit{Compatibilidad con la lógica de negocios}}
    \item {Poner atención a cualquier detalle sospechoso o problemático, como números que no hagan sentido}
\end{itemize}

\paragraph{Lista de checkeo para Verificación}
\begin{itemize}
    \item {Fuentes de errores}
    \item {Faltas de tipografía en palabras y cifras}
    \item {Espacios y caracteres extra}
    \item {Duplicados}
    \item {Formatos no coincidentes}
    \item {\gls{str} y fechas desordenadas o incoherentes}
    \item {Etiquetas engañosas de columnas}
    \item {Datos truncados}
\end{itemize}

\subsection{Documentación}
Proceso en que se mantienen archivos de texto, para realizar rastreos de cambios, adiciones, eliminaciones y errores que hayan aparecido durante un proyecto (no solo limpieza de datos). Mantener documentación permite recuperar errores que se hayan producido en la limpieza, informar a otros usuarios ante cualquier cambio, y evaluar la calidad de los datos limpios. 

Tanto hojas de cálculo como SQL poseen herramientas de \gls{chnglg} incorporadas, que funcionan como un \textit{Historial de Versiones} automático. Estos documentos no tienen formatos preestablecidos por lo que deben acordarse con el resto del equipo, pero por lo general incluyen
\begin{itemize}
    \item {Dato, archivo, fórmula, componente que se haya modificado}
    \item {Descripción del cambio}
    \item {Fecha, persona que realizó/aprobó la modificación}
    \item {Número de versión}
    \item {Motivos detrás del cambio}
\end{itemize}

\newpage