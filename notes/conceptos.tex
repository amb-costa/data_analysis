\section{Conceptos Clave}
\begin{itemize}
    \item {\textbf{Datos} : 
    \begin{itemize}
        \item {Conjunto de hechos}
        \item {Centro de la toma de decisiones, por lo que se deben reducir sesgos, lagunas y errores}
    \end{itemize}}
    \item {\textbf{Ecosistema de Datos} : Todos los elementos que interactúan entre si para producir, administrar, almacenar, organizar, analizar y compartir datos. \textit{Esto incluye las herramientas de Hardware y Software}}
    \item {\textbf{\textit{Data-driven decision making}} : Uso de datos para guiar la Estrategia de Negocios 
    \begin{itemize}
        \item {Los datos nunca serán más poderosos que su combinación con experiencia humana, observación e intuición}
        \item {\textit{\textbf{Data-driven ${\neq}$ Data inspired}} : Explorar diversas fuentes de datos para encontrar similitudes entre ellas}
    \end{itemize}}
    \item {\textbf{\textit{Scope of Work} - Alcance del Trabajo} : Un esquema previamente acordado sobre el trabajo a realizar en un proyecto. Algunos contenidos básicos a cualquier alcance son entregables, hitos, cronogramas e informes. \textit{Declaración de Trabajo ${\neq}$ Alcance de Trabajo}:
    \begin{description}
        \item[Declaración]{ : Documento que identifica claramente los productos y servicios que un proveedor o contratista proporcionará a una organización. Incluye obbjetivos, directrices, programas y costos}
        \item[Alcance]{ : Se basa en proyectos y establece las expectativas y límites de un proyecto. Puede incluirse en la Declaración previa para ayudar a definir los resultados del proyecto}
    \end{description}}
\end{itemize}

\newpage