\section{Conceptos Clave}
\begin{itemize}
    \item {\textbf{Datos} : 
    \begin{itemize}
        \item {Conjunto de hechos}
        \item {Centro de la toma de decisiones, por lo que se deben reducir sesgos, lagunas y errores}
    \end{itemize}}
    \item {\textbf{Ecosistema de Datos} : Todos los elementos que interactúan entre si para producir, administrar, almacenar, organizar, analizar y compartir datos. \textit{Esto incluye las herramientas de Hardware y Software}}
    \item {\textbf{\textit{Data-driven decision making}} : Uso de datos para guiar la Estrategia de Negocios 
    \begin{itemize}
        \item {Los datos nunca serán más poderosos que su combinación con experiencia humana, observación e intuición}
        \item {\textit{\textbf{Data-driven ${\neq}$ Data inspired}} : Explorar diversas fuentes de datos para encontrar similitudes entre ellas}
    \end{itemize}}
    \item {\textbf{\textit{Scope of Work} - Alcance del Trabajo} : Un esquema previamente acordado sobre el trabajo a realizar en un proyecto. Algunos contenidos básicos a cualquier alcance son entregables, hitos, cronogramas e informes. \textit{Declaración de Trabajo ${\neq}$ Alcance de Trabajo}:
    \begin{description}
        \item[Declaración]{ : Documento que identifica claramente los productos y servicios que un proveedor o contratista proporcionará a una organización. Incluye obbjetivos, directrices, programas y costos}
        \item[Alcance]{ : Se basa en proyectos y establece las expectativas y límites de un proyecto. Puede incluirse en la Declaración previa para ayudar a definir los resultados del proyecto}
    \end{description}}
\end{itemize}

\subsection{Diferencias entre Análisis y Ciencia de Datos}
\begin{itemize}
    \item 
\end{itemize}
\begin{itemize}
    \item {Diferencias entre Análisis y Ciencia de Datos}
    \item {\textbf{Análisis de Datos} : 
    \begin{itemize}
        \item {Colecta, transformación, organización de datos, con el fin de sacar conclusiones, realizar predicciones y tomar decisiones informadas basadas en hechos}
        \item {Responder preguntas ya existentes, al proponer nuevas perspectivas en base a datos y sus fuentes}
        \item {\textbf{Analista de Datos} : Persona que recolecta, transforma, organiza datos para ayudar a realizar decisiones informadas}
    \end{itemize}}
    \item {\textbf{Análisis Computacional de Datos} : 
    \begin{itemize}
        \item {\textit{Ciencia de los Datos}}
        \item {Creación de modelos, exploración de lo desconocido usando datos sin editar}
        \item {Creación de preguntas y problemas usando datos}
        \item {\textbf{Ciencia de Datos} contiene a los \textit{Ecosistemas de Datos} y al \textit{Análisis de Datos}}
    \end{itemize}}
\end{itemize}

\newpage