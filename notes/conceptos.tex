\section{Conceptos Clave}
\begin{itemize}
    \item {\textbf{Datos : }
    \begin{itemize}
        \item {Conjunto de hechos}
        \item {Centro de la toma de decisiones, por lo que se deben reducir sesgos, lagunas y errores}
    \end{itemize}}
    \item {\textbf{Ecosistema de Datos : }Todos los elementos que interactúan entre si para producir, administrar, almacenar, organizar, analizar y compartir datos
    \begin{itemize}
        \item {Esto incluye las herramientas de \textit{Hardware} y \textit{Software}}
    \end{itemize}}
    \item {\textbf{\textit{Data-driven decision making : }}Utilización de datos para guiar la Estrategia de Negocios 
    \begin{itemize}
        \item {Los datos nunca serán más poderosos que su combinación con experiencia humana, observación e intuición}
    \end{itemize}}
    \item {\textbf{Expertos en la Materia : }Profesionales del área de estudio que pueden aportar conocimiento experto al análisis}
    \item {\textbf{\textit{Stakeholder} : }Persona que invierte tiempo y recurso en el proyecto, que le interesa el resultado del estudio}
    
    \item {\textbf{Análisis de Personas : }Análisis de datos enfocado en los empleados y su vida laboral, con tal de definir y crear un lugar de trabajo más productivo y empoderador}
    
    \item {\textbf{La Nube : }Ubicación virtual donde pueden almacenarse datos, en vez de utilizar un harddrive de computador}
\end{itemize}

\subsection{Diferencias entre Análisis y Ciencia de Datos}
\begin{itemize}
    \item {\textbf{Análisis de Datos : }
    \begin{itemize}
        \item {Colecta, transformación, organización de datos, con el fin de sacar conclusiones, realizar predicciones y realizar decisiones informadas basadas en hechos}
        \item {Responder preguntas ya existentes, al proponer nuevas perspectivas en base a datos y sus fuentes}
        \item {\textbf{Analista de Datos : }Persona que recolecta, transforma, organiza datos para ayudar a realizar decisiones informadas}
    \end{itemize}}
    \item {\textbf{Análisis Computacional de Datos : }
    \begin{itemize}
        \item {\textit{Ciencia de los Datos }}
        \item {Creación de modelos, exploración de lo desconocido usando datos sin editar}
        \item {Creación de preguntas y problemas usando datos}
        \item {\textbf{Ciencia de Datos} contiene a los \textit{Ecosistemas de Datos} y al \textit{Análisis de Datos}}
    \end{itemize}}
\end{itemize}

\newpage