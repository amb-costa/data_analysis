% nta : si bien presentaciones debieran ir en share, se entiende mejor en el contexto de habilidades del analista
% la idea es recolectar destrezas o habilidades blandas de otras secciones

\section{Destrezas del Analista de Datos}
El analista de datos tiene dos funciones principales : analizar los datos en si, y presentar los resultados de forma efectiva

\subsection{Habilidades Analíticas}
Cualidades y características asociadas con la resolución de problemas usando hechos
\begin{description}
    \item[Curiosidad] : Querer aprender algo
    \item[Entender \gls{cntxt}]
    \item[\gls{strctrdthnkng}]
    \item[Diseño de Datos] : Organización de la información por formatos, tipos o almacenamientos
    \item[Estrategia de datos] : Gestión de la gente, procesos y herramientas usadas en el Análisis de Datos
\end{description}

\subsection{Pensamiento Analítico}
Identificar y definir un problema, para luego resolverlo usando datos de forma organizada, paso a paso. Presenta 5 aspectos principales:
\begin{description}
    \item[Visualización] : Representación gráfica de la información
    \item[Ser Estratégico] : Crear planes para resolver un problema dado ciertos datos, manteniendo el enfoque, calidad y utilidad
    \item[Pensamiento Orientado a Problemas] : Identificar problema, y mantenerlo como motor principal del proyecto completo
    \item[Correlación] : Relación entre dos variables. \textit{Correlación ${\neq}$ causalidad}
    \item{Los siguientes aspectos debieran considerarse al mismo tiempo :
    \begin{description}
        \item[Big Picture Thinking] : Visión del panorama completo
        \item[Detail Oriented Picture] : Estudio de las partes en detalle
    \end{description}}
\end{description}

\subsection{Reuniones y Presentaciones}
Los aspectos más importantes de las presentaciones son : definir el propósito, mantener la presentación concisa, mantener algún tipo de flujo lógico, que sea atractivo visualmente, y que sea fácil de entender. Las habilidades de presentación se practican en cada una
\begin{itemize}
    \item {Controlar la emoción y los nervios}
    \item {Comenzar con las ideas generales}
    \item {Utilizar la regla de los cinco segundos}
    \item {La preparación es clave}
    \item {Se recomienda el siguiente orden de diapositivas : agenda o estructura, propósito, datos o análisis, propuesta, llamada a acción}
    \item {Tener en cuenta que la audiencia : no siempre verá los pasos tomados para llegar a la conclusión, tienen la mente ocupada, se distrae fácilmente}
\end{itemize}

\paragraph{Consejos y trucos para presentar datos y resultados}
\begin{itemize}
    \item {Conocer la mejor forma de organizarse. audiencia? propósito?}
    \item {Preparar puntos de conversación, limitar texto en diapositivas}
    \item {Finalizar con recomendaciones}
    \item {Dejar tiempo para preguntas y presentar los datos}
    \item {Pedir opiniones al cierre}
\end{itemize}

\paragraph{Cómo hablar}
\begin{itemize}
    \item {Mantener las oraciones cortas}
    \item {Generar instancias para pausas intencionales}
    \item {Mantener un tono de voz uniforme}
\end{itemize}

\paragraph{Limitar Hábitos Nerviosos}
\begin{itemize}
    \item {Mantenerse quieto, moverse con propósito}
    \item {Practicar buena postura}
    \item {Hacer contacto visual positivo}
\end{itemize}

\paragraph{Anticiparse a Preguntas}
\begin{itemize}
    \item {Entender las expectativas de los \gls{stkhldrs}}
    \item {Asegurarse de entender el objetivo, y lo que los \gls{stkhldrs} desean}
    \item {Si no se entiende, las preguntas no se responderán correctamente}
    \item {Realizar la \textbf{Prueba del colega} : ensayo de presentación junto a alguien que no sepa del tema, para ver qué preguntas, conclusiones y comentarios pueda generar}
    \item {Empezar sin preconcepciones : no hay que asumir que la audiencia conoce la terminología, acrónimos, eventos o información previa y necesaria}
    \item {Trabajar con el equipo para preparar un borrador con respuestas a preguntas anticipadas}
\end{itemize}

\paragraph{Considerar las Limitaciones de los Datos}
\begin{itemize}
    \item {Análisis crítico de las correlaciones}
    \item {Observar el \gls{cntxt}}
    \item {Comprender las fortalezas y debilidades de las herramientas}
\end{itemize}

\paragraph{Lista de Verificación para preparar Q+A}
\textbf{Antes de la presentación}
\begin{itemize}
    \item {Armar y preparar preguntas}
    \item {Conversar sobre presentaciones con colegas, gerente..., y pedir que comenten preguntas que puedan tener}
    \item {Pedir comentarios y opiniones sobre la presentación o documento de análisis}
    \item {Pensar en preguntas complejas o partes poco claras}
    \item {Practicar la presentación por seguridad, por si falta algo}
\end{itemize}
\textbf{Durante la presentación}
\begin{itemize}
    \item {Prepararse para preguntas}
    \item {Abordar cualquier tema que surja}
    \item {Evitar que una sola persona desvíe la presentación}
    \item {Presentar un apéndice con visualizaciones y contenido extra. \textit{apéndice : información que no está presente en la presentación en si, pero que puede ser útil al tenerla al alcance}}
\end{itemize}

\paragraph{Tipos de Objeciones}
\textbf{Acerca de los datos}
\begin{itemize}
    \item {Dónde se obtuvieron?}
    \item {De qué sistema vienen?}
    \item {Qué transformaciones se le hicieron?}
    \item {Qué tan reciente o precisos son?}
\end{itemize}
\textbf{Acerca del Análisis}
\begin{itemize}
    \item {Se puede replicar?}
    \item {Quién entregó sus opiniones al respecto?}
\end{itemize}
\textbf{Acerca de los Resultados}
\begin{itemize}
    \item {Estos resultados existen para otros periodos de tiempo?}
    \item {Se controlaron las diferencias en los datos?}
\end{itemize}

\paragraph{Respuestas para posibles objeciones}
\begin{itemize}
    \item {Comunicar cualquier suposición}
    \item {Explicar por qué el análisis puede diferir de las expectativas}
    \item {Reconocer la validez de la objeción y tomar acción para seguir investigando}
\end{itemize}

\paragraph{Otras prácticas recomendadas}
\begin{itemize}
    \item {Escuchar la pregunta completa, repetirla de ser necesario}
    \item {Entender el contexto de la pregunta}
    \item {Involucrar a toda la audiencia, incluir a otras voces}
    \item {Mantener respuestas cortas y precisas}
\end{itemize}

\begin{table}
    \centering
    \begin{tabular}{|p{4.8cm}|p{4.8cm}|p{4.7cm}}
        \hline
        \multicolumn{3}{|c|}{Prácticas en Reuniones} \\
        \hline
        \textbf{Antes} & \textbf{Durante}  &\textbf{Después} \\
        \hline
        \begin{description}
            \item {Establecer propósitos, metas y resultados deseados}
            \item {Determinar preguntas o solicitudes a abordar}
            \item {Reconocer a los asistentes}
            \item {Preparar una \textbf{agenda} que incluya horas de inicio y finalización, ubicación, acceso remoto, objetivos y material complementario}
            \item {Distribuir la agenda de antemano}
        \end{description} & \begin{description}
            \item {Hacer introducciones si son necesarias y revisar mensajes claves}
            \item {Presentar los datos junto a cualquier observación, interpretación e implicancia}
            \item {Hacer participar a los asistentes en base a su experiencia y puntos de vista}
            \item {Tomar notas}
            \item {Determinar pasos a seguir}
        \end{description} &  \begin{description}
            \item {Distribuir notas, datos y material extra}
            \item {Confirmar pasos a seguir y plazos para medidas adicionales}
            \item {Pedir comentarios y preguntas}
        \end{description} \\
        \hline
    \end{tabular}
\end{table}

\subsection{Comunicación Clara y Efectiva}
Al mantener el diálogo con colegas, \gls{stkhldrs} y miembros del equipo, se preservan las buenas relaciones. \textit{Es importante tomar en cuenta \textbf{quién} es la audiencia, \textbf{qué saben} previamente y \textbf{qué necesitan} saber}

\subsubsection{Identificar antes de Comunicar}
\begin{itemize}
    \item {\textbf{Hechos} y datos concretos}
    \item {\textbf{\gls{cntxt}} y cualquier cosa que facilite la comprensión}
    \item {\textbf{Incógnitas}, conceptos desconocidos o cosas que se pasen por alto}
    \item {\textbf{Temas comunes} para preguntas, como objetivos, recursos y seguridad}
\end{itemize}

\subsubsection{Tips para Comunicación Efectiva}
\begin{itemize}
    \item {Aprender en el camino y realizar preguntas}
    \item {Practicar buenos hábitos de escritura para diálogo por escrito. Leer en voz alta antes de enviar la conversación}
    \item {Responder en un tiempo adecuado}
    \item {Ser claro con las necesidades, al igual que la historia a contar a través de los datos}
\end{itemize}

\subsubsection{Balance de Expectativas y Objetivos}
\begin{itemize}
    \item {Ser razonables y realistas al establecer plazos y metas}
    \item {Identificar desafíos y problemas con anticipación, además de posibles soluciones}
    \item {Resolver conflictos al replantear el problema, determinar el contexto detrás del conflicto, y tener un diálogo con altura de miras}
\end{itemize}

\subsection{\gls{stkhldrs} más Comunes}
\begin{itemize}
    \item {\textbf{Equipo Ejecutivo} : Proporciona liderazgo estratégico y operativo a la empresa. Generalmente están muy ocupados, por lo que se trabaja con el jefe/gerente de proyecto. \textit{Ejemplos : Vicepresidentes, Directores de Marketing, Profesionales}}
    \item {\textbf{Equipo Orientado al Cliente} : Cualquier persona en la organización que tenga algún nivel de interacción con clientes, actuales y potenciales}
    \item {\textbf{Equipo Ciencia de Datos} : 
    \begin{itemize}
        \item {\textbf{Analista de Datos} : Recolección, transformación y organización de datos, con tal de realizar conclusiones, predicciones y decisiones basadas en hechos}
        \item {\textbf{Ingeniero de Datos} : Procesamiento de datos iniciales para crear un proceso viable, al darles un formato útil para el análisis. Generalmente es el encargado de prepararlos para los analistas y científicos}
        \item {\textbf{Científico de Datos} : Creación de modelos y exploración de lo desconocido usando datos sin editar, para encontrar preguntas y problemas nuevos}
        \item {\textbf{Especialistas de Depósitos de Datos} : Desarrollo de procesos y procedimientos para guardar y organizar datos de forma efectiva}
    \end{itemize}}
\end{itemize}

\newpage