\section{Destrezas del Analista de Datos}

\subsection{Habilidades Analíticas}
Cualidades y características asociadas con la resolución de problemas usando hechos
\begin{description}
    \item[Curiosidad] : Querer aprender algo
    \item[Entender \gls{cntxt}]
    \item[\gls{strctrdthnkng}]
    \item[Diseño de Datos] : Organización de la información por formatos, tipos o almacenamientos
    \item[Estrategia de datos] : Gestión de la gente, procesos y herramientas usadas en el Análisis de Datos
\end{description}

\subsection{Pensamiento Analítico}
Identificar y definir un problema, para luego resolverlo usando datos de forma organizada, paso a paso. Presenta 5 aspectos principales:
\begin{description}
    \item[Visualización] : Representación gráfica de la información
    \item[Ser Estratégico] : Crear planes para resolver un problema dado ciertos datos, manteniendo el enfoque, calidad y utilidad
    \item[Pensamiento Orientado a Problemas] : Identificar problema, y mantenerlo como motor principal del proyecto completo
    \item[Correlación] : Relación entre dos variables. \textit{Correlación ${\neq}$ causalidad}
    \item{Los siguientes aspectos debieran considerarse al mismo tiempo :
    \begin{description}
        \item[Big Picture Thinking] : Visión del panorama completo
        \item[Detail Oriented Picture] : Estudio de las partes en detalle
    \end{description}}
\end{description}

\subsection{Reuniones y Presentaciones}
\begin{table}
    \centering
    \begin{tabular}{|p{4.8cm}|p{4.8cm}|p{4.7cm}}
        \hline
        \multicolumn{3}{|c|}{Prácticas en Reuniones} \\
        \hline
        \textbf{Antes} & \textbf{Durante}  &\textbf{Después} \\
        \hline
        \begin{description}
            \item {Establecer propósitos, metas y resultados deseados}
            \item {Determinar preguntas o solicitudes a abordar}
            \item {Reconocer a los asistentes}
            \item {Preparar una \textbf{agenda} que incluya horas de inicio y finalización, ubicación, acceso remoto, objetivos y material complementario}
            \item {Distribuir la agenda de antemano}
        \end{description} & \begin{description}
            \item {Hacer introducciones si son necesarias y revisar mensajes claves}
            \item {Presentar los datos junto a cualquier observación, interpretación e implicancia}
            \item {Hacer participar a los asistentes en base a su experiencia y puntos de vista}
            \item {Tomar notas}
            \item {Determinar pasos a seguir}
        \end{description} &  \begin{description}
            \item {Distribuir notas, datos y material extra}
            \item {Confirmar pasos a seguir y plazos para medidas adicionales}
            \item {Pedir comentarios y preguntas}
        \end{description} \\
        \hline
    \end{tabular}
\end{table}

\subsection{Comunicación Clara y Efectiva}
Al mantener el diálogo con colegas, \gls{stkhldrs} y miembros del equipo, se preservan las buenas relaciones. \textit{Es importante tomar en cuenta \textbf{quién} es la audiencia, \textbf{qué saben} previamente y \textbf{qué necesitan} saber}

\subsubsection{Identificar antes de Comunicar}
\begin{itemize}
    \item {\textbf{Hechos} y datos concretos}
    \item {\textbf{\gls{cntxt}} y cualquier cosa que facilite la comprensión}
    \item {\textbf{Incógnitas}, conceptos desconocidos o cosas que se pasen por alto}
    \item {\textbf{Temas comunes} para preguntas, como objetivos, recursos y seguridad}
\end{itemize}

\subsubsection{Tips para Comunicación Efectiva}
\begin{itemize}
    \item {Aprender en el camino y realizar preguntas}
    \item {Practicar buenos hábitos de escritura para diálogo por escrito. Leer en voz alta antes de enviar la conversación}
    \item {Responder en un tiempo adecuado}
    \item {Ser claro con las necesidades, al igual que la historia a contar a través de los datos}
\end{itemize}

\subsubsection{Balance de Expectativas y Objetivos}
\begin{itemize}
    \item {Ser razonables y realistas al establecer plazos y metas}
    \item {Identificar desafíos y problemas con anticipación, además de posibles soluciones}
    \item {Resolver conflictos al replantear el problema, determinar el contexto detrás del conflicto, y tener un diálogo con altura de miras}
\end{itemize}

\subsection{\gls{stkhldrs} más Comunes}
\begin{itemize}
    \item {\textbf{Equipo Ejecutivo} : Proporciona liderazgo estratégico y operativo a la empresa. Generalmente están muy ocupados, por lo que se trabaja con el jefe/gerente de proyecto. \textit{Ejemplos : Vicepresidentes, Directores de Marketing, Profesionales}}
    \item {\textbf{Equipo Orientado al Cliente} : Cualquier persona en la organización que tenga algún nivel de interacción con clientes, actuales y potenciales}
    \item {\textbf{Equipo Ciencia de Datos} : 
    \begin{itemize}
        \item {\textbf{Analista de Datos} : Recolección, transformación y organización de datos, con tal de realizar conclusiones, predicciones y decisiones basadas en hechos}
        \item {\textbf{Ingeniero de Datos} : Procesamiento de datos iniciales para crear un proceso viable, al darles un formato útil para el análisis. Generalmente es el encargado de prepararlos para los analistas y científicos}
        \item {\textbf{Científico de Datos} : Creación de modelos y exploración de lo desconocido usando datos sin editar, para encontrar preguntas y problemas nuevos}
        \item {\textbf{Especialistas de Depósitos de Datos} : Desarrollo de procesos y procedimientos para guardar y organizar datos de forma efectiva}
    \end{itemize}}
\end{itemize}

\newpage