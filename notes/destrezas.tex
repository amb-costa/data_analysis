\section{Destrezas del Analista de Datos}

\subsection{Habilidades Analíticas}
Cualidades y características asociadas con la resolución de problemas usando hechos
\begin{description}
    \item[Curiosidad] : Querer aprender algo
    \item[Entender \gls{cntxt}] : Comprensión del ambiente en que se genera, produce o utiliza una idea
    \item[\gls{strctrdthnkng}] : Desglosar cosas en pasos o partes pequeñas, y trabajar con ellas de forma lógica y ordenada
    \item[Diseño de Datos] : Organización de la información por formatos, tipos o almacenamientos
    \item[Estrategia de datos] : Gestión de la gente, procesos y herramientas usadas en el Análisis de Datos
\end{description}

\subsection{Pensamiento Analítico}
Identificar y definir un problema, para luego resolverlo usando datos de forma organizada, paso a paso. Presenta 5 aspectos principales:
\begin{description}
    \item[Visualización] : Representación gráfica de la información
    \item[Ser Estratégico] : Crear planes para resolver un problema dado ciertos datos, manteniendo el enfoque, calidad y utilidad
    \item[Pensamiento Orientado a Problemas] : Identificar problema, y mantenerlo como motor principal del proyecto completo
    \item[Correlación] : Relación entre dos variables. \textit{Correlación ${\neq}$ causalidad}
    \item{Los siguientes aspectos debieran considerarse al mismo tiempo :
    \begin{description}
        \item[Big Picture Thinking] : Visión del panorama completo
        \item[Detail Oriented Picture] : Estudio de las partes en detalle
    \end{description}}
\end{description}

\subsection{Comunicación Clara y Efectiva}
Mantención de buenas relaciones con colegas, \gls{stkhldrs} y miembros del equipo. Previo a la comunicación, hay que establecer
\begin{itemize}
    \item{Quién es la audiencia?}
    \item{Qué es lo que ya saben?}
    \item{Qué es lo que necesitan saber?}
    \item{De qué forma puedo comunicarlo efectivamente?}
\end{itemize}

\subsubsection{Tips para comunicación efectiva}
\begin{itemize}
    \item {Aprender en el camino y realizar preguntas}
    \item {Practicar buenos hábitos de escritura, para comunicación por escrito}
    \item {Leer correos en voz alta antes de enviarlos}
    \item {Responder en un tiempo adecuado}
    \item {Ser claro con las necesidades}
    \item {Contar una historia clara: 
    \begin{itemize}
        \item {Comparar los mismos tipos de datos}
        \item {Visualizar con cuidado}
        \item {Dejar de lado gráficos innecesarios}
        \item {Realizar pruebas de importancia estadística}
        \item {Prestar atención al tamaño de la muestra}
    \end{itemize}}
\end{itemize}

\subsubsection{Cómo balancear expectativas y objetivos}
\begin{itemize}
    \item {Establecer una fecha límite razonable y realista}
    \item {Destacar problemas para \gls{stkhldrs} con anticipación}
    \item {Determinar expectativas realistas en toda etapa del proyecto}
    \item {Identificar desafíos y posibles soluciones}
    \item {Resolver conflictos 
    \begin{itemize}
        \item {Replantear el problema}
        \item {Conversar con altura de miras}
        \item {Entender el contexto del conflicto}
    \end{itemize}}
\end{itemize}

\subsection{Reuniones y Presentaciones}
\begin{table}
    \centering
    \begin{tabular}{|p{4.8cm}|p{4.8cm}|p{4.8cm}}
        \hline
        \multicolumn{3}{|c|}{Prácticas en Reuniones} \\
        \hline
        \textbf{Antes} & \textbf{Durante}  &\textbf{Después} \\
        \hline
        \begin{description}
            \item {Establecer propósitos, metas y resultados deseados, junto con preguntas o solicitudes a abordar}
            \item {Reconocer a los asistentes y hacerlos participar en los diferentes puntos de vista y experiencia previa}
            \item {Preparar y distribuir una \textbf{agenda} que incluya horas de inicio y finalización, lugar e información para casos remotos, objetivos y material complementario para revisar de antemano}
        \end{description} & \begin{description}
            \item {Hacer introducciones si son necesarias y revisar mensajes claves}
            \item {Presentar los datos junto con sus observaciones, interpretaciones e implicancias}
            \item {Tomar notas}
            \item {Determinar pasos a seguir}
        \end{description} &  \begin{description}
            \item {Distribuir notas y datos}
            \item {Confirmar pasos a seguir y plazos para medidas adicionales}
            \item {Pedir comentarios}
        \end{description} \\
        \hline
    \end{tabular}
\end{table}



\subsection{Roles y Situaciones en el Estudio de Datos}
\subsubsection{\gls{stkhldrs} más comunes a interactuar}
\begin{itemize}
    \item {\textbf{Equipo Ejecutivo} : Proporciona liderazgo estratégico y operativo a la empresa. Generalmente están muy ocupados, por lo que se trabaja con el jefe/gerente de proyecto. \textit{Ejemplos : Vicepresidentes, Directores de Marketing, Profesionales}}
    \item {\textbf{Equipo Orientado al Cliente} : Cualquier persona en la organización que tenga algún nivel de interacción con clientes, actuales y potenciales}
    \item {\textbf{Equipo Ciencia de Datos} : 
    \begin{description}
        \item[Analista de Datos]{: Recolección, transformación y organización de datos, con tal de realizar conclusiones, predicciones y decisiones basadas en hechos}
        \item[Ingeniero de Datos]{ : Procesa los datos iniciales para crear un proceso viable. Generalmente es el encargado de prepararlos para los analistas y científicos}
        \item[Científico de Datos]{ : Creación de modelos y exploración de lo desconocido usando datos sin editar, para encontrar preguntas y problemas nuevos}
    \end{description}}
    \item{\textbf{Preguntas para mantenerse enfocado} : 
    \begin{itemize}
        \item {Quiénes son los \gls{stkhldrs} primarios y secundarios?}
        \item {Quién administra los datos?}
        \item {Dónde se puede acudir por ayuda?}
        \item {\textbf{Limitaciones de datos} : 
        \begin{itemize}
            \item{Datos incompletos}
            \item{Omisión de datos desalineados, que no tienen métricas estandarizadas}
            \item{Gestión de datos sucios}
        \end{itemize}}
    \end{itemize}}
\end{itemize}

\begin{description}
    \item[\textit{Scope of Work} - Alcance del Trabajo]{ : Un esquema previamente acordado sobre el trabajo a realizar en un proyecto. Algunos contenidos básicos a cualquier alcance son entregables, hitos, cronogramas e informes. \textit{Declaración de Trabajo ${\neq}$ Alcance de Trabajo}:
    \begin{description}
        \item[Declaración]{ : Documento que identifica claramente los productos y servicios que un proveedor o contratista proporcionará a una organización. Incluye objetivos, directrices, programas y costos}
        \item[Alcance]{ : Se basa en proyectos y establece las expectativas y límites de un proyecto. Puede incluirse en la Declaración previa para ayudar a definir los resultados del proyecto}
    \end{description}}
\end{description}

\newpage