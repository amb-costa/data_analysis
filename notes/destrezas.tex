\section{Destrezas del Analista de Datos}
\textbf{Habilidades Analíticas : }Cualidades y características asociadas con la resolución de problemas usando hechos:

\begin{itemize}
    \item {\textbf{Curiosidad : }Querer aprender algo}
    \item {\textbf{Entender Contexto : }Comprensión del ambiente en que se genera, produce o utiliza una idea}
    \item {\textbf{Mentalidad Técnica : }Desglosar cosas en pasos o partes pequeñas, y trabajar con ellas de forma lógica y ordenada}
    \item {\textbf{Diseño de Datos : }Organización de la información por formatos, tipos o almacenamientos}
    \item {\textbf{Estrategia de datos : }Gestión de la gente, procesos y herramientas usadas en el Análisis de Datos}
\end{itemize}

\textbf{Pensamiento Analítico : }Identificar y definir un problema, para luego resolverlo usando datos de forma organizada, paso a paso. Presenta 5 aspectos principales:

\begin{itemize}
    \item {\textbf{Visualización : }Representación gráfica de la información}
    \item {\textbf{Ser Estratégico : }Crear planes para resolver un problema dado ciertos datos, manteniendo el enfoque, calidad y utilidad}
    \item {\textbf{Pensamiento Orientado a Problemas : }Identificar problema, y mantenerlo como motor principal del proyecto completo}
    \item {\textbf{Correlación : }Relación entre dos variables. \textit{Correlación ${\neq}$ causalidad}}
    \item{Los siguientes aspectos debieran considerarse al mismo tiempo :
    \begin{itemize}
        \item {\textbf{Big Picture Thinking : }Visión del panorama completo}
        \item {\textbf{Detail Oriented Picture : }Estudio de las partes en detalle}
    \end{itemize}}
    
\end{itemize}