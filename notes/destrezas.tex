% nta : si bien presentaciones debieran ir en share, se entiende mejor en el contexto de habilidades del analista
% la idea es recolectar destrezas o habilidades blandas de otras secciones

\section{Destrezas del Analista de Datos}
Un analista de datos tiene dos funciones principales: el análisis en si, y presentar los resultados de forma efectiva. Estos se comparten a través de instancias como reuniones, presentaciones y documentos electrónicos. Así, es necesario desarrollar \textbf{habilidades blandas}: cualidades y comportamientos no técnicos, que describen el trabajo de una persona

\subsection{Pensamiento Analítico}
Notar la diferencia entre pensamiento y \textbf{Habilidades Analíticas}: cualidades asociadas con la resolución de problemas usando hechos,  como ser \textbf{curioso}, entender el \gls{cntxt} de la información y practicar \gls{strctrdthnkng}. El \textbf{Pensamiento} es el proceso de identificar y definir un problema, para resolver de forma organizada en base a datos. Los aspectos principales son
\begin{itemize}
    \item{\textbf{Visualización}: Representación gráfica de la información}
    \item{\textbf{Ser Estratégico}: Crear planes para resolver un problema dado ciertos datos, manteniendo el enfoque, calidad y utilidad}
    \item{\textbf{Pensamiento Orientado a Problemas}: Identificar problema, y mantenerlo como motor principal del proyecto completo}
    \item{\textbf{Correlación}: Relación entre dos variables. \textit{Correlación ${\neq}$ causalidad}}
    \item{Los siguientes aspectos debieran considerarse al mismo tiempo:
    \begin{description}
        \item[Big Picture Thinking]: Visión del panorama completo
        \item[Detail Oriented Picture]: Estudio de las partes en detalle
    \end{description}}
\end{itemize}

\subsection{\gls{stkhldrs} más Comunes}
\begin{itemize}
    \item {\textbf{Equipo Ejecutivo}: Proporciona liderazgo estratégico y operativo a la empresa. Generalmente están muy ocupados, por lo que se trabaja con el jefe/gerente de proyecto. \textit{Ejemplos: Vicepresidentes, Directores de Marketing, Profesionales}}
    \item {\textbf{Equipo Orientado al Cliente}: Cualquier persona en la organización que tenga algún nivel de interacción con clientes, actuales y potenciales}
    \item {\textbf{Equipo Ciencia de Datos}: 
    \begin{itemize}
        \item {\textbf{Analista de Datos}: Recolección, transformación y organización de datos, con tal de realizar conclusiones, predicciones y decisiones basadas en hechos}
        \item {\textbf{Ingeniero de Datos}: Procesamiento de datos iniciales para crear un proceso viable, al darles un formato útil para el análisis. Generalmente es el encargado de prepararlos para los analistas y científicos}
        \item {\textbf{Científico de Datos}: Creación de modelos y exploración de lo desconocido usando datos sin editar, para encontrar preguntas y problemas nuevos}
        \item {\textbf{Especialistas de Depósitos de Datos}: Desarrollo de procesos y procedimientos para guardar y organizar datos de forma efectiva}
    \end{itemize}}
\end{itemize}

\subsection{Comunicación Clara y Efectiva}
Al mantener el diálogo con colegas, \gls{stkhldrs} y miembros del equipo, se preservan las buenas relaciones. \textit{Es importante tomar en cuenta \textbf{quién} es la audiencia, \textbf{qué saben} previamente y \textbf{qué necesitan} saber}

Al igual que en otros procesos, hay que identificar \textbf{hechos} y datos concretos, el \gls{cntxt} detrás y posibles conceptos desconocidos o cosas que se pasen por alto. Siempre existen temas comunes para preguntas, como objetivos, recursos y seguridad

La comunicación se aprende en el camino, y algunas de las expectativas refieren a responder en un tiempo adecuado, o ser claro con la necesidad del mensaje. Se necesitan buenos hábitos de escritura al momento de hablar por escrito, y se recomienda leer en voz alta antes de enviar alguna conversación

Al momento de comunicar expectativas y objetivos, es necesario ser razonables y realistas al momento de establecer plazos o metas. Idealmente, se identifican desafíos y problemas con anticipación con el fin de encontrar soluciones. Cuando ocurre algún conflicto, es útil replantear el problema, determinar el contexto detrás del conflicto y tener un diálogo con altura de miras

\subsection{Reuniones y Presentaciones}
Los datos, y sus resultados, se comparten en instancias como reuniones. La presentación se debe mantener concisa, y su creación empieza por definir su \textbf{propósito}. Para que sea fácil de entender, se comienza con ideas generales, y sigue algún tipo de flujo lógico. 

Además de entender el objetivo de la presentación, es importante considerar las expectativas y deseos de los \gls{stkhldrs} para anticipar preguntas. La \textbf{audiencia} se puede distraer fácilmente, tiene la mente ocupada o no siempre verá los pasos tomados para llegar a cierta conclusión. Tampoco hay que asumir que conocen la terminología y los acrónimos, o que conocen de eventos o información previa 

El presentador debe practicar un buen lenguaje corporal, al exhibir una buena postura, mantenerse quieto y moverse con propósito. Su voz debe mantener un tono de voz uniforme, hablar con oraciones cortas y hacer contacto visual positivo. La presentación necesita pausas intencionales para mostrar los datos, junto a puntos de conversación para abordar temas que aparezcan. La idea es involucrar a la audiencia al incluir otras voces como personas expertas en la materia, pero es necesario evitar que una persona desvíe la conversación

Es necesario que la presentación tenga un atractivo visual que capte la atención. Las visuales debieran cumplir con la regla de los cinco segundos, donde la audiencia entiende la idea principal en dicha cantidad de tiempo. Al momento de crear diapositivas, se debe limitar la cantidad de texto, y seguir un orden similar al siguiente: \textit{agenda/estructura, propósito, datos/análisis, propuesta, llamada a acción}. Tambien se puede compartir un \textbf{apéndice}: información que no está en la presentación en si, pero puede ser útil tener al alcance

Es necesario anticipar preguntas a través de las expectativas de los \gls{stkhldrs}. Al momento de la pregunta, se debe escuchar completa y repetir de ser necesario, porque es importante entender el contexto detrás. La respuesta debe ser corta y precisa, pero si no hay una, se debe reconocer para tomar acción e investigar al respecto

Es natural que se reciban objeciones, o que la información presentada sea limitada. Se debe mantener el \gls{cntxt} en mente, y comprender las fortalezas y debilidades de las herramientas usadas al momento del análisis, además de realizar una evaluación crítica de las correlaciones. Las objeciones deben ser reconocidas y validadas, además de tomar acción para seguir investigando. Sin embargo, el análisis puede ser válido y haya que explicar por qué este puede diferir de las expectativas, además de cualquier suposición

Algunas preguntas y objeciones se enfocan en aspectos de los datos (\textit{como su captura, sistema de origen, transformaciones que pudieron sufrir, precisión o actividad reciente}), en el análisis en si (\textit{por ejemplo, si se puede replicar}) o en los resultados obtenidos (\textit{como comparaciones con otros periodos de tiempo})

De cualquier forma, es buena idea preparar un borrador como equipo, con respuestas a preguntas anticipadas. Se puede conversar con colegas y jefes para pedir comentarios o preguntas sobre la presentación o documento, como la \textbf{Prueba del Colega}: un ensayo de presentación junto a alguien que no sepa del tema, para estudiar qué preguntas, conclusiones y comentarios pueda generar. La idea es encontrar preguntas complejas o partes poco claras por seguridad. Además, se recomienda pedir recomendaciones y comentarios al cierre como retroalimentación para futuras presentaciones

\newpage