\section{Destrezas del Analista de Datos}

\subsection{Habilidades Analíticas}
Cualidades y características asociadas con la resolución de problemas usando hechos
\begin{description}
    \item[Curiosidad] : Querer aprender algo
    \item[Entender \gls{cntxt}] : Comprensión del ambiente en que se genera, produce o utiliza una idea
    \item[\gls{strctrdthnkng}] : Desglosar cosas en pasos o partes pequeñas, y trabajar con ellas de forma lógica y ordenada
    \item[Diseño de Datos] : Organización de la información por formatos, tipos o almacenamientos
    \item[Estrategia de datos] : Gestión de la gente, procesos y herramientas usadas en el Análisis de Datos
\end{description}

\subsection{Pensamiento Analítico}
Identificar y definir un problema, para luego resolverlo usando datos de forma organizada, paso a paso. Presenta 5 aspectos principales:
\begin{description}
    \item[Visualización] : Representación gráfica de la información
    \item[Ser Estratégico] : Crear planes para resolver un problema dado ciertos datos, manteniendo el enfoque, calidad y utilidad
    \item[Pensamiento Orientado a Problemas] : Identificar problema, y mantenerlo como motor principal del proyecto completo
    \item[Correlación] : Relación entre dos variables. \textit{Correlación ${\neq}$ causalidad}
    \item{Los siguientes aspectos debieran considerarse al mismo tiempo :
    \begin{description}
        \item[Big Picture Thinking] : Visión del panorama completo
        \item[Detail Oriented Picture] : Estudio de las partes en detalle
    \end{description}}
\end{description}

\subsection{Roles y Situaciones en el Estudio de Datos}
\begin{description}
    \item[\textit{Scope of Work} - Alcance del Trabajo]{ : Un esquema previamente acordado sobre el trabajo a realizar en un proyecto. Algunos contenidos básicos a cualquier alcance son entregables, hitos, cronogramas e informes. \textit{Declaración de Trabajo ${\neq}$ Alcance de Trabajo}:
    \begin{description}
        \item[Declaración]{ : Documento que identifica claramente los productos y servicios que un proveedor o contratista proporcionará a una organización. Incluye obbjetivos, directrices, programas y costos}
        \item[Alcance]{ : Se basa en proyectos y establece las expectativas y límites de un proyecto. Puede incluirse en la Declaración previa para ayudar a definir los resultados del proyecto}
    \end{description}}
\end{description}


\begin{itemize}
    \item {\textbf{Analista de Datos} : Trabaja generalmente con DB, puede trabar creando paneles}
    \item {\textbf{Ingeniero de Datos} : Procesa los datos iniciales para crear un proceso viable. Generalmente es el encargado de prepararlos para los analistas y científicos}
    \item {\textbf{Científico de Datos} : Crea modelos utilizando dichos datos}
\end{itemize}

\newpage