\section{Destrezas del Analista de Datos}

\subsection{Habilidades Analíticas}
Cualidades y características asociadas con la resolución de problemas usando hechos
\begin{description}
    \item[Curiosidad] : Querer aprender algo
    \item[Entender \gls{cntxt}]
    \item[\gls{strctrdthnkng}]
    \item[Diseño de Datos] : Organización de la información por formatos, tipos o almacenamientos
    \item[Estrategia de datos] : Gestión de la gente, procesos y herramientas usadas en el Análisis de Datos
\end{description}

\subsection{Pensamiento Analítico}
Identificar y definir un problema, para luego resolverlo usando datos de forma organizada, paso a paso. Presenta 5 aspectos principales:
\begin{description}
    \item[Visualización] : Representación gráfica de la información
    \item[Ser Estratégico] : Crear planes para resolver un problema dado ciertos datos, manteniendo el enfoque, calidad y utilidad
    \item[Pensamiento Orientado a Problemas] : Identificar problema, y mantenerlo como motor principal del proyecto completo
    \item[Correlación] : Relación entre dos variables. \textit{Correlación ${\neq}$ causalidad}
    \item{Los siguientes aspectos debieran considerarse al mismo tiempo :
    \begin{description}
        \item[Big Picture Thinking] : Visión del panorama completo
        \item[Detail Oriented Picture] : Estudio de las partes en detalle
    \end{description}}
\end{description}

\subsection{Reuniones y Presentaciones}
\begin{table}
    \centering
    \begin{tabular}{|p{4.8cm}|p{4.8cm}|p{4.7cm}}
        \hline
        \multicolumn{3}{|c|}{Prácticas en Reuniones} \\
        \hline
        \textbf{Antes} & \textbf{Durante}  &\textbf{Después} \\
        \hline
        \begin{description}
            \item {Establecer propósitos, metas y resultados deseados}
            \item {Determinar preguntas o solicitudes a abordar}
            \item {Reconocer a los asistentes}
            \item {Preparar una \textbf{agenda} que incluya horas de inicio y finalización, ubicación, acceso remoto, objetivos y material complementario}
            \item {Distribuir la agenda de antemano}
        \end{description} & \begin{description}
            \item {Hacer introducciones si son necesarias y revisar mensajes claves}
            \item {Presentar los datos junto a cualquier observación, interpretación e implicancia}
            \item {Hacer participar a los asistentes en base a su experiencia y puntos de vista}
            \item {Tomar notas}
            \item {Determinar pasos a seguir}
        \end{description} &  \begin{description}
            \item {Distribuir notas, datos y material extra}
            \item {Confirmar pasos a seguir y plazos para medidas adicionales}
            \item {Pedir comentarios y preguntas}
        \end{description} \\
        \hline
    \end{tabular}
\end{table}

\newpage