\makeglossaries

\newglossaryentry{cntxt}{
    name = Contexto,
    sort = {contexto},
    description = {Comprensión del ambiente en que se genera, produce o utiliza una idea, en la que algo ocurre o existe}
}

\newglossaryentry{attrbt}{
    name = Atributo,
    sort = {atributo},
    description = {Característica o cualidad de los datos usados para etiquetar una columna}
}

\newglossaryentry{frmla}{
    name = Fórmula,
    sort = {fórmula},
    description = {Conjunto de instrucciones para un cálculo específico, usando los datos disponibles. \textit{Ejemplos : Promedio, Error, Desviación Estándar}}
}

\newglossaryentry{oprtr}{
    name = Operador,
    sort = {operador},
    description = {Símbolo que nombra el tipo de operación o cálculo matemático a realizar. \textit{Ejemplos : operadores lógicos and, or, not}}
}

\newglossaryentry{algrthm}{
    name = Algoritmo,
    sort = {algoritmo},
    description = {Proceso o conjunto de reglas a seguir para realizar una tarea en específico}
}

\newglossaryentry{xprts}{
    name = Expertos en la Materia, 
    sort = {expertos},
    description = {Profesionales del área de estudio que pueden aportar conocimiento experto al análisis}
}

\newglossaryentry{stkhldrs}{
    name = \textit{Stakeholders},
    sort = {stakeholders},
    description = {\textit{Interesados}, personas que invierten tiempo y recursos en el proyecto, y que les interesa el resultado final. El tipo \textbf{Primario} tiene relación directa con el proyecto y se verá afectado por su resultado, como \textit{directivos y gerentes}. El tipo \textbf{Secundario} también se verá afectado, pero no posee intereses directos como \textit{empresas externas}}
}

\newglossaryentry{prblmdmn}{
    name = \textit{Problem Domain} - Dominio del Problema, 
    sort = {problem}, 
    description = {Área de análisis específico que abarca cada actividad que afecta o es afectada por el problema}
}

\newglossaryentry{strctrdthnkng}{
    name = \textit{Structured Thinking} - Pensamiento Estructurado,
    sort = {structured}, 
    description = {Proceso en el que se reconoce el problema o situación, y se organiza la información disponible para revelar deficiencias, oportunidades y opciones}
}

\newglossaryentry{pplanlss}{
    name = Análisis de Personas, 
    sort = {análisis personas},
    description = {Analisis de datos enfocado en los empleados y su vida laboral, con tal de definir y crear un lugar de trabajo más productivo y empoderador}
}

\newglossaryentry{bsnssanlss}{
    name = Análisis de Negocios,
    sort = {análisis negocios}, 
    description = {Matemática y estadística enfocada a recolectar, analizar e interpretar datos para realizar decisiones de negocio. Se identifican cuatro tipos según la \href{https://online.hbs.edu/blog/post/business-analytics-examples}{Escuela de Negocios de Harvard} :
    \begin{itemize}
        \item {\textbf{Descriptivo} : Identificar tendencias y patrones}
        \item {\textbf{Predictivo} : Predicción de resultados en base a tendencias y patrones}
        \item {\textbf{Diagnóstico} : Evaluación del problema, búsqueda del problema raíz}
        \item {\textbf{Prescriptivo} : Test y pruebas para encontrar el mejor resultado}
    \end{itemize}}
}

\newglossaryentry{cld}{
    name = La Nube,
    sort = {nube},
    description = {Ubicación virtual donde pueden almacenarse datos, en vez de utilizar un \textit{harddrive} de computador}
}

\newglossaryentry{5w}{
    name = \textit{Five Why's},
    sort = {five},
    description = {Práctica para encontrar la raíz de un problema, que consiste en preguntar "por qué?" cinco veces}
}

\newglossaryentry{gpnlss}{
    name = \textit{Gap Analysis} - Análisis de Déficit,
    sort = {analisis deficit},
    description = {Método de evaluación y examinación de un proceso y cómo funciona actualmente, en contraste con la meta y estado ideal}
}

\newglossaryentry{bias}{
    name = Sesgo,
    sort = {sesgo},
    description = {Preferencia, a favor o en contra, subconsciente o no, respecto a una persona, grupo de personas o cosas. Algunos ejemplos incluyen 
    \begin{itemize}
        \item {\textbf{\textit{Sampling Bias} - Sesgo de Muestreo} : cuando una muestra no es representativa de la población}
        \item {\textbf{\textit{Observer Bias} - Sesgo del Observador} : también llamado \textbf{del Investigador}, cuando dos personas tienden a observar cosas de forma distinta}
        \item{\textbf{\textit{Interpretation Bias} - Sesgo de Interpretación} : tendencia a interpretar situaciones de forma positiva o negativa}
        \item {\textbf{\textit{Confirmation Bias} - Sesgo de Confirmación} : tendencia a buscar o interpretar información de tal forma que confirma suposiciones preexistentes}
    \end{itemize}}
}

\newglossaryentry{dtbs}{
    name = Sesgo de Datos,
    sort = {sesgo datos},
    description = {Tipo de error que sistemáticamente distorsiona los resultados hacia cierta dirección}
}

\newglossaryentry{ppltn}{
    name = Población,
    sort = {poblacion},
    description = {Todos los valores posibles en un set de datos}
}

\newglossaryentry{smpl}{
    name = Muestra,
    sort = {muestra},
    description = {Parte de una población, y que es representativa de la misma}
}

\newglossaryentry{datatype}{
    name = \textit{Data Type},
    sort = {data type},
    description = {Descripción del tipo de valor que cierto atributo posee. \textit{Ejemplos : number, str/text ( secuencias de carácteres y puntuación con información textual), bool (verdadero o falso en base a condiciones lógicas)}}
}

\newglossaryentry{ethcs}{
    name = Ética,
    sort = {etica},
    description = {Estándares bien establecidos sobre lo correcto e incorrecto, que establece lo que los humanos deben hacer y cumplir, en términos de derechos, obligaciones, beneficios a la sociedad, igualdad o virtudes en específico}
}

\newglossaryentry{qry}{
    name = \textit{Query} - Consulta,
    sort = {query},
    description = {Petición a una DB para obtener y manipular datos}
}

\newglossaryentry{dtgvrnnc}{
    name = \textit{Data Governance} - Gobernanza de Datos,
    sort = {data governance},
    description = {Proceso para asegurar el manejo formal de los activos de datos de una compañía}
}

\newglossaryentry{csv}{
    name = \textit{Comma-Separated Values},
    sort = {csv},
    description = {Formato en que los valores se separan por comas}
}

\newglossaryentry{rgx}{
    name = \textit{Regex} - Expresión Regular,
    sort = {regex},
    description = {Secuencia de caracteres que especifica un patrón, combinación o formato a seguir por un string}
}

\glsaddallunused