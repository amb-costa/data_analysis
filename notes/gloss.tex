\makeglossaries

\renewcommand*{\glstextformat}[1]{\textcolor{Bittersweet}{#1}}

\newglossaryentry{cntxt}{
    name = Contexto,
    sort = {contexto},
    description = {Comprensión del ambiente en que se genera, produce o utiliza una idea, en la que algo ocurre o existe}
}

\newglossaryentry{attrbt}{
    name = Atributo,
    sort = {atributo},
    description = {Característica o cualidad de los datos usados para etiquetar una columna}
}

\newglossaryentry{frmla}{
    name = Fórmula,
    sort = {fórmula},
    description = {Conjunto de instrucciones para un cálculo específico, usando los datos disponibles. \textit{Ejemplos: Promedio, Error, Desviación Estándar}}
}

\newglossaryentry{fnctn}{
    name = Función,
    sort = {función},
    description = {Grupo de instrucciones que lleva a cabo un cálculo en específico utilizando datos}
}

\newglossaryentry{oprtr}{
    name = Operador,
    sort = {operador},
    description = {Símbolo que nombra el tipo de operación o cálculo matemático a realizar. \textit{Ejemplos: operadores lógicos and, or, not}}
}

\newglossaryentry{algrthm}{
    name = Algoritmo,
    sort = {algoritmo},
    description = {Conjunto de pasos a seguir para completar una tarea. \textit{Conjunto de pasos para que un programa termine una tarea}}
}

\newglossaryentry{xprts}{
    name = Expertos en la Materia, 
    sort = {expertos},
    description = {Profesionales del área de estudio que pueden aportar conocimiento experto al análisis}
}

\newglossaryentry{stkhldrs}{
    name = \textit{Stakeholders},
    sort = {stakeholders},
    description = {\textit{Interesados}. Personas que invierten tiempo y recursos en el proyecto, y que les interesa el resultado final. El tipo \textbf{Primario} tiene relación directa con el proyecto y se verá afectado por su resultado, como \textit{directivos y gerentes}. El tipo \textbf{Secundario} también se verá afectado, pero no posee intereses directos como \textit{empresas externas}}
}

\newglossaryentry{prblmdmn}{
    name = \textit{Problem Domain}, 
    sort = {problem}, 
    description = {\textit{Dominio del Problema}. Área de análisis específico que abarca cada actividad que afecta o es afectada por el problema}
}

\newglossaryentry{strctrdthnkng}{
    name = \textit{Structured Thinking},
    sort = {structured}, 
    description = {\textit{Pensamiento Estructurado}. Proceso en el que se reconoce el problema o situación, y se organiza la información disponible para revelar deficiencias, oportunidades y opciones}
}

\newglossaryentry{pplanlss}{
    name = Análisis de Personas, 
    sort = {análisis personas},
    description = {Analisis de datos enfocado en los empleados y su vida laboral, con tal de definir y crear un lugar de trabajo más productivo y empoderador}
}

\newglossaryentry{bsnssanlss}{
    name = Análisis de Negocios,
    sort = {análisis negocios}, 
    description = {Matemática y estadística enfocada a recolectar, analizar e interpretar datos para realizar decisiones de negocio. Se identifican cuatro tipos según la \href{https://online.hbs.edu/blog/post/business-analytics-examples}{Escuela de Negocios de Harvard} :
    \begin{itemize}
        \item {\textbf{Descriptivo}: Identificar tendencias y patrones}
        \item {\textbf{Predictivo}: Predicción de resultados en base a tendencias y patrones}
        \item {\textbf{Diagnóstico}: Evaluación del problema, búsqueda del problema raíz}
        \item {\textbf{Prescriptivo}: Test y pruebas para encontrar el mejor resultado}
    \end{itemize}}
}

\newglossaryentry{cld}{
    name = La Nube,
    sort = {nube},
    description = {Ubicación virtual donde pueden almacenarse datos, en vez de utilizar un \textit{harddrive} de computador}
}

\newglossaryentry{5w}{
    name = \textit{Five Why's},
    sort = {five},
    description = {Práctica para encontrar la raíz de un problema, que consiste en preguntar "por qué?" cinco veces}
}

\newglossaryentry{gpnlss}{
    name = \textit{Gap Analysis},
    sort = {analisis deficit},
    description = {\textit{Análisis de Déficit}. Método de evaluación y examinación de un proceso y cómo funciona actualmente, en contraste con la meta y estado ideal}
}

\newglossaryentry{str}{
    name = \textit{String},
    sort = {string},
    description = {\textit{Text String o Cadena de Texto}. Secuencia de caracteres y puntuación dentro de una celda con información textual, compuestos mayormente por letras}
}

\newglossaryentry{datatype}{
    name = \textit{Data Type},
    sort = {data type},
    description = {\textit{Tipo de Dato}. Descripción del tipo de valor que cierto atributo posee. \textit{Ejemplos: number (int, float), string, bool (verdadero o falso en base a condiciones lógicas)}}
}

\newglossaryentry{ethcs}{
    name = Ética,
    sort = {etica},
    description = {Estándares bien establecidos sobre lo correcto e incorrecto, que establece lo que los humanos deben hacer y cumplir, en términos de derechos, obligaciones, beneficios a la sociedad, igualdad o virtudes en específico}
}

\newglossaryentry{qry}{
    name = \textit{Query},
    sort = {query},
    description = {\textit{Consulta}. Petición a una DB para obtener y manipular datos}
}

\newglossaryentry{dtgvrnnc}{
    name = \textit{Data Governance},
    sort = {data governance},
    description = {\textit{Gobernanza de Datos}. Proceso para asegurar el manejo formal de los activos de datos de una compañía}
}

\newglossaryentry{csv}{
    name = \textit{Comma-Separated Values},
    sort = {csv},
    description = {Formato en que los valores se separan por comas}
}

\newglossaryentry{rgx}{
    name = \textit{Regex},
    sort = {regex},
    description = {\textit{Expresión Regular}. Secuencia de caracteres que especifica un patrón, combinación o formato a seguir por un string}
}

\newglossaryentry{ntwrkng}{
    name = \textit{Networking},
    sort = {networking},
    description = {Creación de lazos profesionales con colegas, potenciales jefes o empresas de interés}
}

\newglossaryentry{mrgr}{
    name = \textit{Merger},
    sort = {merger},
    description = {\textit{Fusión}. Acuerdo entre dos organizaciones para unirse legalmente entre si}
}

\newglossaryentry{syntx}{
    name = \textit{Syntax},
    sort = {syntax},
    description = {\textit{Sintaxis}. Estructura predeterminada que incluye toda la información requerida, junto con su ubicación correcta}
}

\newglossaryentry{schm}{
    name = \textit{Schema},
    sort = {schema},
    description = {\textit{Esquema}. Descripción de la forma en que algo está organizado}
}

\newglossaryentry{typecast}{
    name = \textit{Typecasting},
    sort = {typecast},
    description = {Conversión de \gls{datatype}}
}

\newglossaryentry{chnglg}{
    name = \textit{Changelog},
    sort = {changelog},
    description = {\textit{Registro de cambios}. Archivo que contiene la lista de modificaciones hecha a un proyecto, en orden cronológico}
}

\newglossaryentry{probemp}{
    name = Problema Empresarial,
    sort = {problema},
    description = {Resultado u objetivos que la empresa busca alcanzar, que considera exitoso. Es la pregunta o problema a solucionar a través del análisis. \textit{También llamado Desafío/Objetivo Empresarial, Estrategia de Negocios}}
}

\newglossaryentry{habbland}{
    name = Habilidades Blandas,
    sort = {habilidades},
    description = {Cualidades y comportamentos no técnicos, que describen el trabajo de una persona}
}

\newglossaryentry{mrgngnnc}{
    name = Márgen de Ganancias,
    sort = {margen},
    description = {Porcentaje que indica cuántos centavos/pesos de ganancia se generaron por cada dólar de venta}
}

\newglossaryentry{alctrab}{
    name = Alcance de Trabajo,
    sort = {alcance},
    description = {Esquema previamente acordado sobre el trabajo a realizar en un proyecto, estableciendo las expectativas y límites. Algunos contenidos básicos son entregables, hitos, cronogramas e informes}
}

\newglossaryentry{decltrab}{
    name = Declaración de Trabajo,
    sort = {declaracion},
    description = {Documento que identifica claramente los productos y servicios que un proveedor o contratista proporciona a una organización. Incluye elementos como objetivos, directrices, programas y costos. \textit{no es lo mismo que el alcance: este puede ser parte de la declaración}}
}

\newglossaryentry{opnsrc}{
    name = \textit{Open Source},
    sort = {open}, 
    description = {Código de libre disponibilidad, que podría ser modificado y compartido por las personas que lo utilizan}
}



\glsaddallunused