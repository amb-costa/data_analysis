% nta
% como hay varios caps que incluyen programación o sus fundamentos
% es mejor unificar todo en un capítulo
% y que este capítulo aparezca antes en el apunte

\section{Fundamentos de Programación}
La \textbf{Programación Computacional} consiste en entregar instrucciones a una computadora para llevar una acción (o un grupo de ellas) a cabo. Un \textbf{Lenguaje de Programación} corresponde a las palabras y símbolos que utilizamos para escribir instrucciones, de forma que un computador las siga. Esto permite \textbf{clarificar pasos del análisis, ahorrar tiempo, y reproducir y compartir el trabajo.}\textit{ej: R, Python, Javascript}
\begin{itemize}
    \item {\textbf{IDE}: aplicación ligera que reúne varias herramientas a usar, en un lugar central}
    \item {\textbf{Instrucciones Condicionales}: if(), else(), elseif(), if(condicion){acción} else if(otra){otra accion} else {accion final}}
    \item {\textbf{Consola}: área donde se entregan comandos al lenguaje. Se borra cuando el lenguaje se cierra, por lo que se necesita usar un editor para guardar comandos necesarios}
    \item {\textbf{\textit{Syntax} - Sintaxis}: estructura predeterminada que incluye toda la información requerida, junto a su ubicación correcta}
    \item {\textbf{\textit{Schema} - Esquema}: descripción de la forma en que algo está organizado, como una tabla relacional}
\end{itemize}

\subsection{Conceptos Clave}
\begin{itemize}
    \item {\textbf{Algoritmo}: conjunto de pasos a seguir para completar una tarea. \textit{Conjunto de pasos para que un programa termine una tarea}}
    \item {\textbf{La Nube}: ubicación virtual donde se pueden almacenar datos, en vez de utilizar un \textit{harddrive} de computador}
    \item {\textbf{Operador}: símbolo que nombra el tipo de operación o cálculo matemático a realizar. \textit{ej: \textbf{aritmético}(+-*/), \textbf{lógico}(AND, OR, NOT), }}
    \item {\textbf{\textit{Datatype} - Tipo de Dato}: Descripción del tipo de valor que cierto atributo posee. \textit{ej: number(int,float), string, bool(TRUE, FALSE)}. Se define el \textbf{\textit{Typecasting}} como la conversión de \textit{datatype}}
    \item {\textbf{\textit{String}[str] - Cadena de Texto}: Secuencia de caracteres y puntuación dentro de una celda con información textual, compuesto en mayor proporción por letras}
    \item {\textbf{Fórmula}: Conjunto de instrucciones para un cálculo específico con datos datos. \textit{ej: promedio, error, desviación estándar}}
    \item {\textbf{Celda}: Posición descrita usando filas o \textbf{registros} y columnas o \textbf{campos}. \textit{ej: A3 columna A fila 3}. Hay dos tipos de referencia para celdas
    \begin{itemize}
        \item {\textbf{Referencia Relativa}: al ser arrastrada, la celda cambia de valor junto con la fórmula donde está presente}
        \item {\textbf{Referencia Absoluta}: el valor de la celda es fijo y no cambia con movimientos en fórmulas o celdas}
    \end{itemize}
    \begin{table}
            \centering
            \begin{tabular}{|p{5cm}|p{4.5cm}|p{4.5cm}|}
                \hline
                \multicolumn{3}{|c|}{Diferencias entre Referencias de Celdas} \\
                \hline
                & \textbf{Referencia Absoluta} & \textbf{Referencia Relativa} \\
                \hline
                \textbf{Ejemplos} & A1, C3, G2:G9 & \$A\$10, C\$2, \$D3 \\
                \hline
                \textbf{Si la celda se arrastra una columna a la derecha, una fila hacia abajo} & A10 $\rightarrow$ B11 & \$A10 $\rightarrow$ \$A11 \\
                \hline
            \end{tabular}
        \end{table}}
\end{itemize}

\subsection{Bases de Datos}
Al igual que al momento de clasificar, las bases de datos \textbf{relacionales} (\textit{Relational Database Management System - RDBMS}) contienen un grupo de tablas que se pueden conectar entre si, en base a sus relaciones o lo que tengan en común. \textbf{Si un campo se usa en dos tablas, se puede usar como conexión}

\begin{itemize}
    \item {\textbf{\textit{Primary Key} - Clave Primaria}: Identificador para una columna en específico donde cada uno de sus valores son únicos. No admite valores nulos o blancos, y no siempre es requerida en una tabla pero si aparece, solo se permite una. También es posible crear una \textbf{Clave Compuesta} a partir de varias columnas de una tabla}
    \item {\textbf{\textit{Foreign Key} - Clave Externa}: Campo (o conjunto de) dentro de una tabla que es llave primaria en otra. Esta referencia permite conectar ambas tablas, y una tabla puede tener múltiples claves externas}
\end{itemize}

Así, las bases de datos \textbf{no relacionales} agrupan datos con todas las variables posibles en conjunto, lo que dificulta clasificar. Generalmente se relaciona con dialectos \textit{NOSQL}

Se aísla la información de una \textit{database} a través de consultas o \textit{\textbf{queries}}. Así, se puede seleccionar, crear, añadir o descargar datos desde y hacia la \textit{database}

\subsubsection{\textit{Spreadsheets vs Databases}}
Considerar que para ambos, \textbf{Campo} corresponde a las columnas, y \textbf{Registro} a las filas. En un registro se contienen múltiples campos, o sea: un set de datos puede tener múltiples valores.

\paragraph{Características en Común}
Ambas herramientas permiten aritmética, fórmulas, y la agregación de datos

\paragraph{Diferencias}
Las hojas de cálculo se generan en un programa, mientras que SQL es un lenguaje de interacción con programas de DB.
Las hojas de cálculo se guardan de forma local, mientras que las consultas SQL se mantienen en la DB.
Las hojas de cálculo brindan acceso a datos que el usuario ingresa. SQL puede obtener información desde fuentes variadas en la misma DB
Las hojas de cálculo son ideales para sets de datos pequeños, SQL permiten trabajar con sets más grandes
Las hojas de cálculo facilitan el trabajo independiente, mientras que SQL mantiene registro de los cambios dentro del equipo
Las hojas de cálculo contienen funcionalidades incorporadas, y SQL es útil en varios programas


\begin{table}
    \centering
    \begin{tabular}{|p{3.5cm}|p{5.5cm}|p{5.5cm}|}
        \hline
        & \textbf{Spreadsheet} & \textbf{Database} \\
        \hline
        \textbf{Ubicación} & Apps de Software & Almacenes de datos accesibles mediante \textit{queries} \\
        \hline
        \textbf{Estructura\break de Datos} & Formato de filas y columnas & Uso de reglas y relaciones \\
        \hline
        \textbf{Organización} & En celdas & En colecciones completas \\
        \hline
        \textbf{Acceso de Datos} & Cantidad limitada & Grandes cantidades \\
        \hline
        \textbf{Ingreso de Datos} & Manual & Escrito y Coherente \\
        \hline
        \textbf{Personas\break trabajando} & Generalmente un usuario a la vez & Múltiples usuarios \\
        \hline
        \textbf{Controlado por} & Usuario & Sistema de gestión de base de datos \\
        \hline
    \end{tabular}
\end{table}

\newpage