% nta
% como hay varios caps que incluyen programación o sus fundamentos
% es mejor unificar todo en un capítulo
% y que este capítulo aparezca antes en el apunte

\section{Fundamentos de Programación}
La \textbf{Programación Computacional} consiste en entregar instrucciones a una computadora para llevar una acción (o un grupo de ellas) a cabo. Un \textbf{Lenguaje de Programación} corresponde a las palabras y símbolos que utilizamos para escribir instrucciones, de forma que un computador las siga. Esto permite \textbf{clarificar pasos del análisis, ahorrar tiempo, y reproducir y compartir el trabajo.}\textit{ej: R, Python, Javascript}
\begin{itemize}
    \item {\textbf{IDE}: aplicación ligera que reúne varias herramientas a usar, en un lugar central}
    \item {\textbf{Instrucciones Condicionales}: if(), else(), elseif(),
            if(condicion){acción} else if(otra){otra accion} else {accion final}}
    \item {\textbf{Consola} : área donde se entregan comandos al lenguaje. Se borra cuando el lenguaje se cierra, por lo que se necesita usar un editor para guardar comandos necesarios}
\end{itemize}

\subsection{Conceptos Clave}
\begin{itemize}
    \item {\textbf{Algoritmo}: conjunto de pasos a seguir para completar una tarea. \textit{Conjunto de pasos para que un programa termine una tarea}}
    \item {\textbf{La Nube}: ubicación virtual donde se pueden almacenar datos, en vez de utilizar un \textit{harddrive} de computador}
    \item {\textbf{Operador}: símbolo que nombra el tipo de operación o cálculo matemático a realizar. \textit{ej: \textbf{aritmético}(+-*/), \textbf{lógico}(AND, OR, NOT), }}
    \item {\textbf{\textit{Datatype} - Tipo de Dato}: Descripción del tipo de valor que cierto atributo posee. \textit{ej: number(int,float), string, bool(TRUE, FALSE)}}
    \item {\textbf{\textit{String}[str] - Cadena de Texto}: secuencia de caracteres y puntuación dentro de una celda con información textual, compuesto en mayor proporción por letras}
\end{itemize}