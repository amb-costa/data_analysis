% nta: ahora el ciclo de vida pasa a secciones, con tal de mantener orden en secciones y subsecciones

\section{\textit{Ask} - Preguntar}
Se establece un \gls{probemp}. Es necesario comprender las expectativas de los \gls{stkhldrs}, junto a sus desafíos o preconcepciones anteriores. Algunas técnicas o métodos a aplicar incluyen
\begin{itemize}
    \item {\gls{strctrdthnkng}}
    \item {\gls{5w}, para determinar la causa principal de un problema}
    \item {\gls{gpnlss}, para encontrar déficits en el proyecto}
\end{itemize}

\subsection{Tipos de Objetivos}
\begin{description}
    \item [Realizar Predicciones]{ : Decisiones informadas para visualizar las cosas a futuro}
    \item [Categorizar Cosas]{ : Asignar información a dos grupos de datos distintos con cualidades comunes}
    \item [Encontrar Algo Inusual]{ : Identificar datos que se salen de la norma}
    \item [Identificar Temas]{ : Agrupar información categorizada en conceptos más amplios}
    \item [Descubrir Conexiones]{ : Encontrar desafíos similares encontrados por otros, para combinar datos y conclusiones}
    \item [Encontrar Patrones]{ : Utilizar datos históricos para entender qué ocurrió en el pasado, y que podría repetirse}
\end{description}


\subsection{Preguntas \textit{SMART}}
Preguntas abiertas (por lo general), que incitan a respuestas útiles y efectivas. \textit{Esto contrasta con las preguntas \textbf{sugestivas} con respuestas en particular, \textbf{imprecisas} que no ofrecen \gls{cntxt}, y \textbf{cerradas} con respuestas breves o de una sola palabra}
\begin{description}
    \item [\textit{Specific}]{ : Simples, enfocadas en una cantidad pequeña de ideas relacionadas entre si. \textbf{Esta pregunta aborda el problema, entrega la respuesta o \gls{cntxt} que necesito?}}
    \item [\textit{Measurable}]{ : Cuantificables y evaluables. \textbf{Esta pregunta da respuestas medibles?}}
    \item [\textit{Action-Oriented}]{ : Promueven el cambio y la acción. \textbf{Esta pregunta entregará información que ayude a diseñar un plan de acción o cambio?}}
    \item [\textit{Relevant}]{ : Tienen importancia en el problema a resolver. \textbf{Esta pregunta se relaciona con un problema en específico?}}
    \item [\textit{Time-Bound}]{ : Son específicas al periodo de tiempo estudiado. \textbf{La pregunta es relevante para el periodo establecido?}}
\end{description}

\newpage