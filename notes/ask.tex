% nta: esta parte podría reeditarse para agregar info de destrezas, en especial las que involucren a stakeholders, reuniones y otros

\subsection{\textit{Ask} - Preguntar}
Se establece un \textbf{Desafío/Objetivo Empresarial}, y las características de lo que se considera exitoso. Para esto es necesario comprender las expectativas de los \gls{stkhldrs}, junto a desafíos o preconcepciones anteriores. Algunas técnicas o métodos a aplicar incluyen
\begin{itemize}
    \item {\gls{strctrdthnkng}}
    \item {\gls{5w}, para determinar la causa principal de un problema}
    \item {\gls{gpnlss}, para encontrar déficits en el proyecto}
\end{itemize}

\subsubsection{Tipos de Problemas}
\begin{itemize}
    \item {\textbf{Realizar Predicciones} : Decisiones informadas para visualizar las cosas a futuro}
    \item {\textbf{Categorizar Cosas} : Asignar información a dos grupos de datos distintos con cualidades comunes}
    \item {\textbf{Encontrar Algo Inusual} : Identificar datos que se salen de la norma}
    \item {\textbf{Identificar Temas} : Agrupar información categorizada en conceptos más amplios}
    \item {\textbf{Descubrir Conexiones} : Encontrar desafíos similares encontrados por otros, para combinar datos y conclusiones}
    \item {\textbf{Encontrar Patrones} : Utilizar datos históricos para entender qué ocurrió en el pasado, y que podría repetirse}
\end{itemize}


\subsubsection{Preguntas \textit{SMART}}
Preguntas altamente efectivas y usualmente abiertas, que incitan a respuestas útiles. \textit{Esto contrasta con las preguntas \textbf{sugestivas} con respuestas en particular,\textbf{imprecisas} que no ofrecen \gls{cntxt}, y \textbf{cerradas} con respuestas breves o de una sola palabra}
\begin{description}
    \item [\textit{Specific}]{ : Simples y enfocadas en una cantidad pequeña de ideas relacionadas entre si. \textbf{Esta pregunta tiene el \gls{cntxt} necesario, aborda el problema o entrega la respuesta que necesito?}}
    \item [\textit{Measurable}]{ : Cuantificables y evaluables. \textbf{Esta pregunta da respuestas medibles?}}
    \item [\textit{Action-Oriented}]{ : Promueven el cambio y la acción. \textbf{Esta pregunta entregará información que ayude a diseñar un plan de acción o cambio?}}
    \item [\textit{Relevant}]{ : Tienen importancia en el problema a resolver. \textbf{Esta pregunta se relaciona con un problema en específico?}}
    \item [\textit{Time-Bound}]{ : Son específicas al periodo de tiempo estudiado. \textbf{La pregunta es relevante para el periodo establecido?}}
\end{description}

\subsubsection{Comunicación Clara y Efectiva}
Mantención de buenas relaciones con colegas, \gls{stkhldrs} y miembros del equipo a través de buenas prácticas de diálogo. \textit{Es importante tomar en cuenta \textbf{quién} es la audiencia, \textbf{qué saben} previamente y \textbf{qué necesitan} saber}

\paragraph{Identificar antes de Comunicar}
\begin{itemize}
    \item {\textbf{Hechos} y datos concretos}
    \item {\textbf{\gls{cntxt}} y cualquier cosa que facilite la comprensión}
    \item {\textbf{Incógnitas}, conceptos desconocidos o cosas que se pasen por alto}
    \item {\textbf{Temas comunes} para preguntas, como objetivos, recursos y seguridad}
\end{itemize}

\paragraph{Tips para Comunicación Efectiva}
\begin{itemize}
    \item {Aprender en el camino y realizar preguntas}
    \item {Practicar buenos hábitos de escritura para diálogo por escrito. Leer en voz alta antes de enviar la conversación}
    \item {Responder en un tiempo adecuado}
    \item {Ser claro con las necesidades, al igual que la historia a contar a través de los datos}
\end{itemize}

\paragraph{Balance de Expectativas y Objetivos}
\begin{itemize}
    \item {Ser razonables y realistas al establecer plazos y metas}
    \item {Identificar desafíos y problemas con anticipación, además de posibles soluciones}
    \item {Resolver conflictos al replantear el problema, determinar el contexto detrás del conflicto, y tener un diálogo con altura de miras}
\end{itemize}

\subsubsection{\gls{stkhldrs} más Comunes}
\begin{itemize}
    \item {\textbf{Equipo Ejecutivo} : Proporciona liderazgo estratégico y operativo a la empresa. Generalmente están muy ocupados, por lo que se trabaja con el jefe/gerente de proyecto. \textit{Ejemplos : Vicepresidentes, Directores de Marketing, Profesionales}}
    \item {\textbf{Equipo Orientado al Cliente} : Cualquier persona en la organización que tenga algún nivel de interacción con clientes, actuales y potenciales}
    \item {\textbf{Equipo Ciencia de Datos} : 
    \begin{description}
        \item[Analista de Datos]{: Recolección, transformación y organización de datos, con tal de realizar conclusiones, predicciones y decisiones basadas en hechos}
        \item[Ingeniero de Datos]{ : Procesa los datos iniciales para crear un proceso viable. Generalmente es el encargado de prepararlos para los analistas y científicos}
        \item[Científico de Datos]{ : Creación de modelos y exploración de lo desconocido usando datos sin editar, para encontrar preguntas y problemas nuevos}
    \end{description}}
\end{itemize}


\newpage