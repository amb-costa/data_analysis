% nta : la idea es convertir cada fase del ciclo en una sección, para efectos del índice
% conceptos básicos de visualización
% planificar + preparar visualizaciones efectivas + dashboards
% intro a tableau
% anticipar y responder preguntas

\section{\textit{Share} - Compartir}
Con los resultados del análisis, se preparan \textbf{\textit{Data Viz} - visualizaciones}, que representarán a los datos de forma gráfica para su presentación, aunque también pueden ser parte del análisis. Cuando la información se encuentra en imágenes, es más fácil de entender. 

La \textbf{\textit{Data Storytelling} - Narrativa de Datos} consiste en la comunicación del significado de un conjunto de datos a través de una narrativa contada con visuales, que se personalizan de acuerdo a la audiencia. La meta es \textbf{involucrar} a la audiencia al capturar y mantener su atención

La \textbf{Composición de Datos} consiste en combinar las partes individuales de una visualización, y mostrarlas todas como un conjunto. Hay dos tipos principales de \textit{Data Viz}:
\begin{itemize}
    \item {\textbf{Visualización Estática} : no cambian con el tiempo a menos que se editen}
    \item {\textbf{Visualización Dinámica} : son interactivas, cambian con el tiempo}
\end{itemize}
Una visual efectiva tiene un \textbf{significado claro}, un uso sofisticado de \textbf{contraste} y una \textbf{ejecución} refinada. Una visualización bien elegida facilita la comprensión de la historia a contar, y dirige la \textbf{atención} a los puntos más importantes : la audiencia debiera reconocer la idea principal en los primeros cinco segundos de verla

Los \textit{Data Viz} se vuelven más accesible con el uso de etiquetas, formatos y de texto (para elementos que no son textuales como fotos), y que los datos sean simples y puedan distinguirse unos de otros

\paragraph{Design Thinking}
Proceso utilizado para resolver problemas complejos desde el foco del usuario. Consiste en cinco fases que representan una descripción general, no es necesario que estén en orden
\begin{itemize}
    \item {\textbf{Empatizar} : con las emociones y necesidades de la audiencia}
    \item {\textbf{Definir} : las necesidades de la audiencia (de la mano con empatizar)}
    \item {\textbf{Idear} : generar ideas y soluciones} 
    \item {\textbf{Prototipar} : paneles, diagramas y otras visualizaciones}
    \item {\textbf{Testear} : obtener comentarios de parte de los \gls{stkhldrs}}
\end{itemize}

\paragraph{Atributos Preatencionales}
Elementos de una visualización de datos que las personas reconocen automáticamente, sin realizar un esfuerzo consciente
\begin{itemize}
    \item {\textbf{Marcas} : objetos visuales básicos como puntos, líneas, formas. Cada marca se puede descomponer en \textbf{tamaño, forma, color y posición} (donde se encuentra la marca específica en el espacio, en relación con una escala u otras marcas)}
    \item {\textbf{Canales} : aspectos o variables que representan características de los datos, las marcas que se usaron para visualizarlos. La efectividad en que la marca comunica el dato depende de tres elementos:
    \begin{itemize}
        \item {\textbf{Exactitud} : utilidad para estimar con exactitud los valores representados}
        \item {\textbf{Destaque} : qué tan fácil es distinguir un dato del otro}
        \item {\textbf{Agrupamiento} : qué tan bien comunica el canal sobre distintos grupos que existen en los datos}
    \end{itemize}}
\end{itemize}

\paragraph{Algunos Principios de Diseño}
\begin{itemize}
    \item {Elegir elemento visual correcto}
    \item {Optimizar proporción dato-trazo. \textit{Enfocarse en la parte del elemento visual que es esencial para comprender el sentido del gráfico}}
    \item {Utilizar orientación de manera eficaz}
    \item {Color y cantidad de cosas}
\end{itemize}

\paragraph{Elementos de Arte para \textit{DataViz}}
\begin{itemize}
    \item {\textbf{Línea} : curva, derecha, gruesa, punteado...}
    \item {\textbf{Forma} : en \textit{DataViz}, siempre 2D}
    \item {\textbf{Color} : \textbf{Matiz}(color), \textbf{Intensidad}(brillante, mate), \textbf{Valor}(claro, oscuro)}
    \item {\textbf{Espacio} : área alrededor y entre objetos}
    \item {\textbf{Movimiento} : sentido de flujo o acción}
    \item {\textbf{Títulos} : línea de palabras escritas en letras grandes en la parte de arriba de la visualización para comunicar qué datos se están presentando}
    \item {\textbf{Subtítulos} : apoya al título añadiendo más contexto y descripción}
    \item {\textbf{Leyenda} : identifica el significado de varios elementos en un \textit{DataViz}, aparte o afuera de los datos}
    \item {\textbf{Etiquetas} : lo mismo que la leyenda, pero se ubica en los datos. Preferir por sobre leyendas, ya que es más fácil de leer. \textit{Por ejemplo en el eje x/y, identifica datos en relación a otros datos}}
    \item {\textbf{Anotaciones} : centra al público en un aspecto concreto de los datos}
\end{itemize}

\paragraph{Principios básicos de Diseño}
\begin{itemize}
    \item {\textbf{Equilibrio} : distribución pareja de la información}
    \item {\textbf{Énfasis} : punto focal en los datos más importantes, para la concentración del público}
    \item {\textbf{Movimiento} : imitar la manera en que las personas leen}
    \item {\textbf{Patrón} : crear o romper}
    \item {\textbf{Repetición} : añade eficacia}
    \item {\textbf{Proporción} : de acuerdo a la importancia}
    \item {\textbf{Ritmo} : sensación de movimiento o flujo}
    \item {\textbf{Variedad} : mantiene el interés}
    \item {\textbf{Unidad} : el \textit{DataViz} debe ser cohesivo}
\end{itemize}

\paragraph{Evitar gráficos confusos y engañosos}
\begin{itemize}
    \item {Cortar el eje y}
    \item {Uso engañoso de eje y doble}
    \item {Limitar artificialmente el alcance de los datos}
    \item {Elecciones problemáticas en cuanto a cómo se combinan o agrupan los datos}
    \item {Usar elementos visuales de "parte-todo" cuando los totales no se suman correctamente}
    \item {Ocultar tendencias en gráficos acumulativos}
    \item {Suavizar artificialmente las tendencias}
\end{itemize}

\subsection{Dos \textit{frameworks} para buenas visualizaciones}
\paragraph{Método McCandles}
\begin{itemize}
    \item {\textbf{Información} : los datos}
    \item {\textbf{Historia} : concepto, que junto a los datos generan el esquema de lo que se muestra}
    \item {\textbf{Meta} : función de los datos, usabilidad y utilidad}
    \item {\textbf{Forma Visual} : metáfora, lo bello y la estructura}
\end{itemize}

\paragraph{Verificación con Trifecta de Kaiser Fung (Junk Charts)}
Conjunto útil de preguntas para criticar lo consumido y qué tan eficaz es
\begin{itemize}
    \item {Cuál es la pregunta práctica? a través de la perspectiva del público que se enfrenta a la visualización}
    \item {Qué dicen los datos?}
    \item {Qué dice el elemento visual?}
\end{itemize}

\subsection{Ejemplos de \textit{DataViz}}
\begin{itemize}
    \item {\textbf{Gráfico de Barras/Columnas} : contraste de tamaños para comparar dos o más valores. Puede ser horizontal (barras) o vertical (columnas)}
    \item {\textbf{Gráfico de Líneas} : para cambios, movimientos o tendencias, donde cada punto representa un dato. \textit{Preferir por sobre barras para cambios pequeños}}
    \item {\textbf{Gráfico Circular} : cuál es la proporción de una parte en relación al conjunto completo}
    \item {\textbf{Gráfico de Área} : seguimiento de cambio en valores en múltiples categorías}
    \item {\textbf{Mapas} : organizan los datos en forma periódica}
    \item {\textbf{Mapa de Calor} : comparación de categorías en un conjunto de datos, como relaciones entre dos variables, utilizando un sistema de color como una \textit{Paleta de Colores Divergente} : muestra dos rangos de calores utilizando intensidad de colores para mostrar la magnitud del número}
    \item {\textbf{Histograma} : gráfico que muestra la frecuencia en que ciertos valores se encuentran en ciertos rangos}
    \item {\textbf{Gráfico de Correlación} : muestra las relaciones o tendencias entre datos. \textit{También llamado Diagrama de Dispersión}}
    \item {\textbf{Diagrama de Densidad} : muestra las tendencias en un intervalo o rango}
    \item {Se tienen los siguientes patrones significativos (referir al árbol de decisiones)
    \begin{itemize}
        \item {\textbf{Cambio} : gráfico de línea/barra}
        \item {\textbf{Agrupación} : gráfico de distribución}
        \item {\textbf{Relatividad/Proporciones} : gráfico circular}
        \item {\textbf{Clasificación} : gráfico de barras}
        \item {\textbf{Correlación} : diagrama de dispersión}
    \end{itemize}}
\end{itemize}

% árbol de decisiones : herramienta de toma de decisiones basadas en preguntas clave, para visualizaciones:
% solo una variable numérica? histograma, diagrama de densidad
% múltiples sets de datos? gráfico de líneas, gráfico circular
% cambios en el tiempo? gráfico de barras
% mostrar relaciones entre datos? diagrama de dispersión, mapa de calor

\subsection{\textit{Data Viz} - Herramientas de Visualización}
\begin{itemize}
    \item {Permiten crear gráficos, tablas, mapas y cuadros}
    \item {Convierten números complejos en historias que las personas puedan entender}
    \item {Ayudan a interesados a sacar conclusiones que permiten tomar decisiones informadas y elaborar estrategias empresariales eficaces}
    \item {\textbf{Métrica} : Tipo de dato singular y cuantificable que puede utilizarse para medir algo
    \begin{itemize}
        \item {\textbf{Meta Métrica} : Meta propuesta por una compañía que puede ser medida y evaluada de acuerdo a métricas}
        \item {\textit{Ejemplo} : Retorno de Inversión = $\frac{Ganancia Neta}{Costo de Inversión}$}
    \end{itemize}}
    \item {\textbf{\textit{Pivot Table} - Tabla Pivote} : Utilizado en el procesamiento de datos, permite resumir, ordenar y reorganizar datos en un DB, para realizar cálculos como promedios o cantidades totales. Se generan \textit{campos calculados}, que llevan a cabo ciertos cálculos en base a los valores de otros campos}
\end{itemize}

\subsubsection{\textit{Report} - Informes}
Recuento estático de datos, entregados de forma periódica a \Gls{stkhldrs}

\subsubsection{\textit{Dashboard} - Panel}
Herramienta que organiza la información entrante desde distintas fuentes, en formatos de \textit{Data Viz}, en una ubicación central, . Esto facilita el monitoreo, análisis y visualización simple de los datos. El ingreso es automático \textit{solo si la estructura de datos es la misma}. Los paneles también pueden incluir \textbf{filtros} para mostrar y ocultar información a conveniencia
\paragraph{Proceso para crear un panel}
    \begin{itemize}
        \item {Identificar a \Gls{stkhldrs} que necesiten ver los datos y cómo los usarán, a través de preguntas efectivas}
        \item {Diseño del panel : encabezados claros para etiquetar información ; descripciones de texto concisas para cada visualización ; mostrar la información más importante en la parte superior}
        \item {Crear prototipos}
        \item {Seleccionar visualizaciones para utilizar en el panel : cambio de valores a lo largo del tiempo con gráficos de línea y barras ; contribución de cada parte al resultado final con gráficos circulares y de torta}
        \item {Crear filtros de ser necesario, identificar patrones mientras los datos originales se mantienen intactos}
    \end{itemize}
\paragraph{Tres categorías comunes de paneles}
    \begin{itemize}
        \item {\textbf{Paneles Estratégicos} : Toma de decisiones para la empresa en periodos más largos}
        \item {\textbf{Paneles Operativos} : Escala temporal de días, semanas o meses, prácticamente a tiempo real}
        \item {\textbf{Paneles Analíticos} : Uso, análisis y predicciones como categoría más técnica, realizadas por \textit{data scientists}}
    \end{itemize}