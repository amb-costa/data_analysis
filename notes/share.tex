% nta : la idea es convertir cada fase del ciclo en una sección, para efectos del índice
% conceptos básicos de visualización
% planificar + preparar visualizaciones efectivas + dashboards
% intro a tableau
% anticipar y responder preguntas

\section{\textit{Share} - Compartir}
Con los resultados del análisis, se preparan \textbf{visualizaciones}, que representarán a los datos de forma gráfica para su presentación. Cuando la información se encuentra en imágenes, es más fácil de entender. La idea es que la visualización refuerce la historia contada por los datos
\begin{itemize}
    \item {Mirar visuales, de forma de entender y sacar conclusiones de sus datos, como forma de análisis}
    \item {La audiencia debiera reconocer el tema principal de la visualización en los primeros 5 segundos de verla}
    \item {Luego, debieran entender y sacar conclusiones}
\end{itemize}

\paragraph{Atributos Preatencionales}
Elementos de una visualización de datos que las personas reconocen automáticamente, sin realizar un esfuerzo consciente
\begin{itemize}
    \item {\textbf{Marcas} : objetos visuales básicos como puntos, líneas, formas. Cada marca se puede descomponer en \textbf{tamaño, forma, color y posición} (donde se encuentra la marca específica en el espacio, en relación con una escala u otras marcas)}
    \item {\textbf{Canales} : aspectos o variables que representan características de los datos, las marcas que se usaron para visualizarlos. La efectividad en que la marca comunica el dato depende de tres elementos:
    \begin{itemize}
        \item {\textbf{Exactitud} : utilidad para estimar con exactitud los valores representados}
        \item {\textbf{Destaque} : qué tan fácil es distinguir un dato del otro}
        \item {\textbf{Agrupamiento} : qué tan bien comunica el canal sobre distintos grupos que existen en los datos}
    \end{itemize}}
\end{itemize}

\paragraph{Algunos Principios de Diseño}
\begin{itemize}
    \item {Elegir elemento visual correcto}
    \item {Optimizar proporción dato-trazo. \textit{Enfocarse en la parte del elemento visual que es esencial para comprender el sentido del gráfico}}
    \item {Utilizar orientación de manera eficaz}
    \item {Color y cantidad de cosas}
\end{itemize}

\paragraph{Evitar gráficos confusos y engañosos}
\begin{itemize}
    \item {Cortar el eje y}
    \item {Uso engañoso de eje y doble}
    \item {Limitar artificialmente el alcance de los datos}
    \item {Elecciones problemáticas en cuanto a cómo se combinan o agrupan los datos}
    \item {Usar elementos visuales de "parte-todo" cuando los totales no se suman correctamente}
    \item {Ocultar tendencias en gráficos acumulativos}
    \item {Suavizar artificialmente las tendencias}
\end{itemize}

\subsection{Dos \textit{frameworks} para buenas visualizaciones}
\paragraph{Método McCandles}
\begin{itemize}
    \item {\textbf{Información} : los datos}
    \item {\textbf{Historia} : concepto, que junto a los datos generan el esquema de lo que se muestra}
    \item {\textbf{Meta} : función de los datos, usabilidad y utilidad}
    \item {\textbf{Forma Visual} : metáfora, lo bello y la estructura}
\end{itemize}

\paragraph{Verificación con Trifecta de Kaiser Fung (Junk Charts)}
Conjunto útil de preguntas para criticar lo consumido y qué tan eficaz es
\begin{itemize}
    \item {Cuál es la pregunta práctica? a través de la perspectiva del público que se enfrenta a la visualización}
    \item {Qué dicen los datos?}
    \item {Qué dice el elemento visual?}
\end{itemize}