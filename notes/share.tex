% nta: la idea es convertir cada fase del ciclo en una sección, para efectos del índice
% conceptos básicos de visualización
% planificar + preparar visualizaciones efectivas + dashboards
% intro a tableau
% anticipar y responder preguntas

\section{\textit{Share} - Compartir}
Con los resultados del análisis, se preparan \textbf{\textit{Data Viz} - visualizaciones}, que representarán a los datos de forma gráfica para su presentación, aunque también pueden ser parte del análisis. Cuando la información se encuentra en imágenes, es más fácil de entender. 

La \textbf{\textit{Data Storytelling} - Narrativa de Datos} consiste en la comunicación del significado de un conjunto de datos a través de una narrativa contada con visuales, que se personalizan de acuerdo a la audiencia. La meta es \textbf{involucrar} a la audiencia al capturar y mantener su atención, por lo que hay que considerar quién es la audiencia y lo que necesitan saber. Para crear esta narrativa de forma eficiente, se necesita
\begin{itemize}
    \item {\textbf{Personajes}: personas afectadas}
    \item {\textbf{Configuración}: qué es lo que ocurre, qué tan seguido, qué tareas. \textit{Antecedentes previos}}
    \item {\textbf{Conflicto o Tema Principal}: punto de tensión}
    \item {\textbf{Revelación}: resolución del problema}
    \item {\textbf{Momento a-ha!}: compartir recomendaciones}
\end{itemize}
Un \textit{framework} estrátegico de presentación parte por entender el \gls{probemp}. Luego se establece la \textbf{hipótesis}: la idea a probar a través de los datos. Mientras antes se establezca la hipótesis, mejor es el entendimiento


% nta: incluir en alguna parte: paste != link != embed
La \textbf{Composición de Datos} consiste en combinar las partes individuales de una visualización, y mostrarlas todas como un conjunto. Para distinguir entre datos \textbf{activos} y \textbf{estáticos}, se considera su antigüedad, cuánto falta para que sean obsoletos o inválidos, y si necesitan actualización para considerarse válidos. Hay dos tipos principales de \textit{Data Viz}:
\begin{itemize}
    \item {\textbf{Visualización Estática}: no cambian con el tiempo a menos que se editen. \textit{ej: screenshots o capturas en presentaciones y paneles}
    \begin{description}
        \item [Pros]{ 
        \begin{itemize}
            \item {Control estricto de una narrativa puntual de los datos y la información}
            \item {Explicar en profundidad análisis complejos a un público más amplio}
        \end{itemize}}
        \item [Contras]{ 
        \begin{itemize}
            \item {Información pierde valor inmediatamente y sigue haciéndolo con el tiempo}
            \item {Las instantáneas no siguen el ritmo de cambio de los datos}
        \end{itemize}}
    \end{description}}
    \item {\textbf{Visualización Dinámica o Activa}: son interactivas, cambian con el tiempo. \textit{ej: paneles, informes con vistas conectadas}
    \begin{description}
        \item[Pros]{
        \begin{itemize}
            \item {Paneles más dinámicos y escalables}
            \item {Datos más actualizados para las personas que los necesiten, cuando lo necesiten}
            \item {Vistas actualizadas de los datos, construir una única fuente de verdad escalable para distintos usos}
            \item {Permiten tomar medidas inmediatas sobre datos que cambian con frecuencia}
            \item {Alivian tiempo y recursos dedicados al análisis}
        \end{itemize}}
        \item[Contras]{
        \begin{itemize}
            \item {Pueden requerir recursos de ingeniería para mantener las canalizaciones}
            \item {Se pierde control de narrativa ya que no hay interpretación de datos: caos de datos}
            \item {Falta de confianza si los datos no se manejan adecuadamente}
        \end{itemize}
        }
    \end{description}}
\end{itemize}
Una visual efectiva tiene un \textbf{significado claro}, un uso sofisticado de \textbf{contraste} y una \textbf{ejecución} refinada. Una visualización bien elegida facilita la comprensión de la historia a contar, y dirige la \textbf{atención} a los puntos más importantes: la audiencia debiera reconocer la idea principal en los primeros cinco segundos de verla

Los \textit{Data Viz} se vuelven más accesible con el uso de etiquetas, formatos y de texto (para elementos que no son textuales como fotos), y que los datos sean simples y puedan distinguirse unos de otros





\subsection{Dos \textit{frameworks} para buenas visualizaciones}
\paragraph{Método McCandles}
\begin{itemize}
    \item {\textbf{Información}: los datos}
    \item {\textbf{Historia}: concepto, que junto a los datos generan el esquema de lo que se muestra}
    \item {\textbf{Meta}: función de los datos, usabilidad y utilidad}
    \item {\textbf{Forma Visual}: metáfora, lo bello y la estructura}
\end{itemize}
Aplicar este método en \textit{Data Viz} consiste en
\begin{itemize}
    \item {Presentar gráficos por su nombre}
    \item {Responder preguntas obvias antes que sean preguntadas}
    \item {Establecer la percepción del gráfico}
    \item {Mostrar los datos que apoyan esa percepción}
    \item {Mencionar la importancia de los datos a la audiencia: presentar el posible impacto de la solución para la empresa, junto a las acciones que los \gls{stkhldrs} podrían tomar}
\end{itemize}

\paragraph{Verificación con Trifecta de Kaiser Fung (Junk Charts)}
Conjunto útil de preguntas para criticar lo consumido y qué tan eficaz es
\begin{itemize}
    \item {Cuál es la pregunta práctica? a través de la perspectiva del público que se enfrenta a la visualización}
    \item {Qué dicen los datos?}
    \item {Qué dice el elemento visual?}
\end{itemize}

\subsection{\textit{Data Viz} - Herramientas de Visualización}
\begin{itemize}
    \item {Permiten crear gráficos, tablas, mapas y cuadros}
    \item {Convierten números complejos en historias que las personas puedan entender}
    \item {Ayudan a interesados a sacar conclusiones que permiten tomar decisiones informadas y elaborar estrategias empresariales eficaces}
    \item {\textbf{Métrica}: Tipo de dato singular y cuantificable que puede utilizarse para medir algo
    \begin{itemize}
        \item {\textbf{Meta Métrica}: Meta propuesta por una compañía que puede ser medida y evaluada de acuerdo a métricas}
        \item {\textit{Ejemplo}: Retorno de Inversión = $\frac{Ganancia Neta}{Costo de Inversión}$}
    \end{itemize}}
    \item {\textbf{\textit{Pivot Table} - Tabla Pivote}: Utilizado en el procesamiento de datos, permite resumir, ordenar y reorganizar datos en un DB, para realizar cálculos como promedios o cantidades totales. Se generan \textit{campos calculados}, que llevan a cabo ciertos cálculos en base a los valores de otros campos}
\end{itemize}

\subsubsection{\textit{Report} - Informes}
Recuento estático de datos, entregados de forma periódica a \Gls{stkhldrs}

\subsubsection{\textit{Dashboard} - Panel}
Herramienta que organiza la información entrante desde distintas fuentes, en formatos de \textit{Data Viz}, en una ubicación central, . Esto facilita el monitoreo, análisis y visualización simple de los datos. El ingreso es automático \textit{solo si la estructura de datos es la misma}. Los paneles también pueden incluir \textbf{filtros} para mostrar y ocultar información a conveniencia
\paragraph{Proceso para crear un panel}
    \begin{itemize}
        \item {Identificar a \Gls{stkhldrs} que necesiten ver los datos y cómo los usarán, a través de preguntas efectivas}
        \item {Diseño del panel: encabezados claros para etiquetar información ; descripciones de texto concisas para cada visualización ; mostrar la información más importante en la parte superior}
        \item {Crear prototipos}
        \item {Seleccionar visualizaciones para utilizar en el panel: cambio de valores a lo largo del tiempo con gráficos de línea y barras ; contribución de cada parte al resultado final con gráficos circulares y de torta}
        \item {Crear filtros de ser necesario, identificar patrones mientras los datos originales se mantienen intactos}
    \end{itemize}
\paragraph{Tres categorías comunes de paneles}
    \begin{itemize}
        \item {\textbf{Paneles Estratégicos}: Toma de decisiones para la empresa en periodos más largos}
        \item {\textbf{Paneles Operativos}: Escala temporal de días, semanas o meses, prácticamente a tiempo real}
        \item {\textbf{Paneles Analíticos}: Uso, análisis y predicciones como categoría más técnica, realizadas por \textit{data scientists}}
    \end{itemize}
\begin{table}
    \centering
    \begin{tabular}{|p{2.9cm}|p{5.4cm}|p{5.5cm}|}
        \hline
        \multicolumn{3}{|c|}{Beneficios de paneles} \\
        \hline
        & \textbf{Para analistas} & \textbf{Para \Gls{stkhldrs}} \\
        \hline
        \small{\textbf{Centralización}} & Compartir una única fuente de datos con los interesados & Trabajar con visión integral de datos, iniciativas y procesos \\
        \hline
        \small{\textbf{Visualización}} & Mostrar y actualizar datos \break entrantes en tiempo real & Detectar patrones y tendencias cambiante más rápidamente \\
        \hline
         \small{\textbf{Percepción}} & Extraer información desde \break diferentes conjuntos de datos & Comprender la historia detrás de los números \\
        \hline
        \small{\textbf{Personalización}} & Crear vistas personalizadas para los interesados & Profundizar en preocupaciones o intereses específicos \\
        \hline
    \end{tabular}
\end{table}

\begin{table}
    \centering
    \begin{tabular}{|p{1.8cm}|p{5.9cm}|p{5.9cm}|}
        \hline
        & \textbf{\textit{Reports}} & \textbf{\textit{Dashboard}} \\
        \hline
        \textbf{Pros} & \begin{description}
            \item {Datos de alto nivel histórico}
            \item {Fácil de diseñar y usar}
            \item {Datos limpios y ordenados}
        \end{description} & \begin{description}
            \item {Automático e interactivo}
            \item {Mejor acceso para \Gls{stkhldrs}}
            \item {Bajo mantenimiento}
        \end{description} \\
        \hline
        \textbf{Contras} & \begin{description}
            \item {Mantenimiento contínuo}
            \item {Estático}
            \item {Menos atractivo visual}
        \end{description} & \begin{description}
            \item {Diseño requiere más trabajo}
            \item {Puede ser confuso}
            \item {Datos sin procesar}
        \end{description} \\
        \hline
    \end{tabular}
\end{table}

\newpage