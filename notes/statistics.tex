% nta : habrán muchos términos y cálculos estadísticos, en especial para las secciones de Procesar y Analizar
% idealmente, tener toda esa información en un documento aparte
% tentativamente serán parte de la sección del kit del analista

\section{Estadística}
\subsection{Conceptos Básicos}
\begin{description}
    \item [Población]{ : Todos los valores posibles en un set de datos}
    \item [Muestra]{ : Parte de una población, que es representativa de la misma}
    \item [Muestreo Aleatorio]{ : Método de selección de muestra donde se eligen valores al azar, de forma que cada valor posible tenga una posibilidad justa de salir seleccionado}
    \item [Margen de Error]{ : Diferencia entre los valores reales y los obtenidos a utilizar. Cantidad máxima que, se espera, los resultados de la muestra difieran de los valores de la población}
    \item [Nivel de Confianza]{ : Grado de confianza en los resultados. Probabilidad de que la muestra represente fielmente al resto de la población. \textit{Si se realiza una encuesta 100 veces, el x\% de las veces saldrán los mismos resultados}}
    \item [Intervalo de Confianza]{ : Rango de valores posibles del resultado de la población respecto al nivel de confianza. \textit{resultado muestra ${\pm}$ margen error}}
    \item [Significancia Estadística]{ : Determinación sobre la posibilidad de que el resultado se base en posibilidades aleatorias. \textit{A mayor significancia, menor es la probabilidad}. También llamado \textbf{Poder Estadístico}} 
    \item [Testeo de Hipótesis]{ : Pruebas para verificar si una encuesta o experimento entrega resultados válidos}
    \item [Tasa de Respuesta Estimada]{ : Porcentaje esperado de respuestas válidas a una encuesta, prueba o estudio}
    \item [Correlación]{ : mención del grado en que dos variables se mueven una en relación con la otra (un patrón) o entre si (positiva, negativa). \textit{No implica \textbf{Causalidad} : un episodio lleva directamente a un resultado específico}}
\end{description}

\paragraph{Planificación de Tamaños de Muestra}
\begin{itemize}
    \item {No utilizar muestras menores a 30}
    \item {95\% es el nivel de confianza más usado, pero un 90\% también puede funcionar}
    \item {80\% o 0.8 es el mínimo porcentaje de poder estadístico para considerar que los resultados son estadísticamente significativos}
    \item {A muestras más grandes, mayor es el nivel de confianza y significancia estadística, y menor es el margen de error}
    \item {Los tamaños de muestra más grandes implican más costos para la empresa}
    \item {Existen calculadoras online para tamaños de muestra, en base a niveles de confianza y significancia que se necesiten}
\end{itemize}

\subsection{Sesgo}
El sesgo es la preferencia, a favor o en contra, subconsciente o no, respecto a una persona, grupo de personas o cosas. Algunos ejemplos incluyen 
    \begin{description}
        \item [\textit{Sampling Bias}]{ o \textbf{Sesgo de Muestreo}, cuando una muestra no es representativa de la población}
        \item [\textit{Observer Bias}]{ o \textbf{Sesgo del Observador/del Investigador}, refiere a cuando dos personas tienden a observar cosas de forma distinta}
        \item [\textit{Interpretation Bias}]{o \textbf{Sesgo de Interpretación}, es la tendencia a interpretar situaciones de forma positiva o negativa}
        \item [\textit{Confirmation Bias}]{ o \textbf{Sesgo de Confirmación}, es la tendencia a buscar o interpretar información de tal forma que confirma suposiciones preexistentes}
        \item [\textit{Data Bias}]{ o \textbf{Sesgo de Datos}, consiste en errores que distorsionan sistemáticamente los resultados hacia cierta dirección}
    \end{description}

\newpage