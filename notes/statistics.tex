% nta: habrán muchos términos y cálculos estadísticos, en especial para las secciones de Procesar y Analizar
% idealmente, tener toda esa información en un documento aparte
% tentativamente serán parte de la sección del kit del analista

\section{Estadística}
\subsection{Conceptos Básicos}
\begin{description}
    \item [Población]{: Todos los valores posibles en un set de datos}
    \item [Muestra]{: Parte de una población, que es representativa de la misma}
    \item [Muestreo Aleatorio]{: Método de selección de muestra donde se eligen valores al azar, de forma que cada valor posible tenga una posibilidad justa de salir seleccionado}
    \item [Margen de Error]{: Diferencia entre los valores reales y los obtenidos a utilizar. Cantidad máxima que, se espera, los resultados de la muestra difieran de los valores de la población}
    \item [Nivel de Confianza]{: Grado de confianza en los resultados. Probabilidad de que la muestra represente fielmente al resto de la población. \textit{Si se realiza una encuesta 100 veces, el x\% de las veces saldrán los mismos resultados}}
    \item [Intervalo de Confianza]{: Rango de valores posibles del resultado de la población respecto al nivel de confianza. \textit{resultado muestra ${\pm}$ margen error}}
    \item [Significancia Estadística]{: Qué tan probable es que los resultados se obtuvieran por posibilidades aleatorias. \textit{A mayor significancia, menor es la probabilidad}. También llamado \textbf{Poder Estadístico}} 
    \item [Testeo de Hipótesis]{: Pruebas para validar los resultados de una encuesta o estudio}
    \item [Tasa de Respuesta Estimada]{: Porcentaje esperado de respuestas válidas a un estudio}
    \item [Correlación]{: mención del grado en que dos variables se mueven una en relación con la otra (un patrón) o entre si (positiva, negativa). \textit{No implica \textbf{Causalidad}: un episodio lleva directamente a un resultado específico}}
\end{description}

\paragraph{Planificación de Tamaños de Muestra}
\begin{itemize}
    \item {No utilizar muestras menores a 30}
    \item {95\% es el nivel de confianza más usado, pero un 90\% también puede funcionar}
    \item {80\% o 0.8 es el mínimo porcentaje de poder estadístico para considerar que los resultados son estadísticamente significativos}
    \item {A muestras más grandes, mayor es el nivel de confianza y significancia estadística, y menor es el margen de error}
    \item {Los tamaños de muestra más grandes implican más costos para la empresa}
    \item {Existen calculadoras online para tamaños de muestra, en base a niveles de confianza y significancia que se necesiten}
\end{itemize}

\subsection{Sesgo}
Preferencia, a favor o en contra, subconsciente o no, respecto a una persona, grupo de personas o cosas. Algunos ejemplos incluyen 
    \begin{description}
        \item [\textit{Sampling Bias} - de Muestreo]{:  la muestra no es representativa de la población}
        \item [\textit{Observer Bias} - del Observador/Investigador]{: dos personas tienden a observar cosas de forma distinta}
        \item [\textit{Interpretation Bias} de Interpretación]{: tendencia a interpretar situaciones de forma positiva o negativa}
        \item [\textit{Confirmation Bias} - de Confirmación]{: tendencia a buscar o interpretar información de forma que confirme suposiciones preexistentes}
        \item [\textit{Data Bias} - de Datos]{: errores que distorsionan sistemáticamente los resultados hacia cierta dirección}
    \end{description}

\subsection{Aplicación en \textit{Data Viz}}
Los resultados matemáticos pueden procesarse en forma de visualizaciones, las que pueden replicarse a mano o generar a través de herramientas de \textit{Data Viz}, las que pueden ofrecer distintas formas de personalizar las visuales
\subsubsection{Elementos de Arte}
\begin{itemize}
    \item {\textbf{Línea}: curva, derecha, gruesa, punteado...}
    \item {\textbf{Forma}: en \textit{DataViz}, siempre 2D}
    \item {\textbf{Color}: \textbf{Matiz}(color), \textbf{Intensidad}(brillante, mate), \textbf{Valor}(claro, oscuro)}
    \item {\textbf{Espacio}: área alrededor y entre objetos}
    \item {\textbf{Movimiento}: sentido de flujo o acción}
    \item {\textbf{Títulos}: texto escrito en letras grandes en la parte superior de la visualización, describe los datos a presentar}
    \item {\textbf{Subtítulos}: apoya al título añadiendo más contexto y descripción}
    \item {\textbf{Leyenda}: identifica el significado de los elementos, aparte o afuera de la visualización}
    \item {\textbf{Etiquetas}: leyenda ubicada dentro de la visualizacióne. Preferir esta opción, es más fácil de leer. \textit{ej:  en el eje x/y, identifica datos en relación a otros datos}}
    \item {\textbf{Anotaciones}: centra al público en un aspecto concreto de los datos}
\end{itemize}

\subsubsection{Ejemplos más Comunes}
\begin{itemize}
    \item {\textbf{Gráfico de Barras/Columnas}: contraste de tamaños para comparar dos o más valores. Puede ser horizontal (barras) o vertical (columnas)}
    \item {\textbf{Gráfico de Líneas}: para cambios, movimientos o tendencias, donde cada punto representa un dato. \textit{Preferir por sobre barras para cambios pequeños}}
    \item {\textbf{Gráfico Circular}: cuál es la proporción de una parte en relación al conjunto completo}
    \item {\textbf{Gráfico de Área}: seguimiento de cambio en valores en múltiples categorías}
    \item {\textbf{Mapas}: organizan los datos en forma periódica}
    \item {\textbf{Mapa de Calor}: comparación de categorías en un conjunto de datos, como relaciones entre dos variables, utilizando un sistema de color como una \textit{paleta de colores divergentes}: muestra dos rangos de calores utilizando intensidad de colores para mostrar la magnitud del número}
    \item {\textbf{Histograma}: gráfico que muestra la frecuencia en que ciertos valores se encuentran en ciertos rangos}
    \item {\textbf{Gráfico de Correlación}: muestra las relaciones o tendencias entre datos. \textit{También llamado Diagrama de Dispersión}}
    \item {\textbf{Diagrama de Densidad}: muestra las tendencias en un inteervalo o rango}
    \item {Se tienen los siguientes patrones significativos (referir al árbol de decisiones)
    \begin{description}
        \item [Cambio]{: gráfico de línea/barra}
        \item [Agrupación]{: gráfico de distribución}
        \item [Relatividad/Proporciones]{: gráfico circular}
        \item [Clasificación]{: grafico de barras}
        \item [Correlación]{: diagrama de dispersión}
    \end{description}}
\end{itemize}

% árbol de decisiones: herramienta de toma de decisiones basadas en preguntas clave, para visualizaciones:
% solo una variable numérica? histograma, diagrama de densidad
% múltiples sets de datos? gráfico de líneas, gráfico circular
% cambios en el tiempo? gráfico de barras
% mostrar relaciones entre datos? diagrama de dispersión, mapa de calor



\paragraph{Evitar gráficos confusos y engañosos}
\begin{itemize}
    \item {Cortar el eje y}
    \item {Uso engañoso de eje y doble}
    \item {Limitar artificialmente el alcance de los datos}
    \item {Elecciones problemáticas en cuanto a cómo se combinan o agrupan los datos}
    \item {Usar elementos visuales de ''parte-todo'' cuando los totales no se suman correctamente}
    \item {Ocultar tendencias en gráficos acumulativos}
    \item {Suavizar artificialmente las tendencias}
    
\end{itemize}

\newpage