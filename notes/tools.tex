\section {Kit de Herramientas del Analista}
% separación de subsección por motivos de orden

\section{\textit{Spreadsheets} - Hojas de Cálculo}
Las hojas de cálculo se utilizan generalmente para conjuntos de datos pequeños, donde los valores se ingresan de forma manual. Brindan funciones incorporadas como creaciones de gráficos y visualizaciones, corrección ortográfica incorporada, entre otras. Pueden conectarse con otras hojas, tablas HTML y archivos \gls{csv}.

\paragraph{Herramientas Comunes}
Toda fórmula se comienza escribiendo '='. Al hacerlo, se despliega un menú que incluye fórmulas, nombres y \gls{str}. Se incluyen ejemplos de \gls{syntx} para \textcolor{blue}{fórmulas} y \textcolor{ForestGreen}{funciones}
\begin{itemize}
    \item {\textbf{Formateo Condicional} : Cambia la estética de las celdas cuando sus valores cumplen con condiciones especificadas}
    \item {\textbf{Llenado Automático} : Presente como un cuadrado en la esquina inferior derecha de la celda, se arrastra y llena celdas vacías con el contenido y referencias de la celda inicial}
    \item {\textbf{Remover Duplicados} : Busca y elimina automáticamente aquellas entradas que podrían estar duplicadas} 
    \item {\textbf{Buscar y Reemplazar} : Herramienta que busca por un término específico y lo reemplaza por otro determinado}
    \item {\textcolor{ForestGreen}{\textbf{SPLIT}} : Función que divide el texto en base a un caracter, y guarda cada fragmento en una celda nueva. El caracter es un \textbf{delimitador} y puede ser un espacio, coma, guiones, entre otros}
    \item {\textcolor{ForestGreen}{\textbf{=CONCATENATE(item1,item2...)}} : Función que une múltiples \gls{str} en uno solo. \textit{ej =CONCATENATE("text",'' '',"string") devuelve "text string"}}
    \item {\textcolor{ForestGreen}{\textbf{=COUNTIF(rango,"valor\_comillas")}} : Cuenta el número de celdas, dentro de un rango determinado que cumplen con una condición. \textit{ej =COUNTIF(A2:A30,'${\geq}$10') devuelve la cantidad de valores que son iguales o mayores a 10}}
    \item {\textcolor{ForestGreen}{\textbf{=LEN(rango)}} : Cuenta el número de caracteres que contiene un \gls{str} dado. \textit{ej =LEN(A21) devuelve el largo del string en la celda A21}}
    \item {\textcolor{ForestGreen}{\textbf{=LEFT}} \& \textcolor{ForestGreen}{\textbf{=RIGHT}}: Extrae un conjunto de caracteres a la izquierda o derecha de un \gls{str} determinado}
    \item {\textcolor{ForestGreen}{\textbf{=MID(rango,punto\_referencia,cant\_char)}} : Extrae un segmento de \gls{str} desde el medio. \textit{ej =MID(C3,4,5) devuelve un string de 5 caracteres que empezó en la posición 4 del string inicial ubicado en la celda C3}}
    \item {\textcolor{ForestGreen}{\textbf{=TRIM(rango)}} : Remueve los espacios iniciales, finales y repetidos de un \gls{str}. \textit{ej =TRIM('' a esto le sobran espacios  '') devuelve ''a esto le sobran espacios''}}
    \item {\textcolor{ForestGreen}{\textbf{=VLOOKUP(valor,''nombre\_ubicacion'',rango,columna,FALSE)}} : \textit{Vertical Lookup}, busca un valor en una columna, para devolver otro valor correspondiente. \textit{ej =VLOOKUP() con FALSE para coincidencias exactas}}
    \item {\textcolor{ForestGreen}{\textbf{=COUNTA}} : Cuenta el total de valores dentro de un rango especificado}
    \item {\textcolor{ForestGreen}{\textbf{=IMPORTRANGE(sheet\_url,range\_str)}} : Importa datos desde una hoja de cálculo a otra, con actualización automática ante cualquier cambio}
    \item {\textcolor{ForestGreen}{\textbf{=QUERY(sheet+range,"SQL CLAUSE")}} : Pseudo instrucción SQL, puede funcionar como asistente para importación de datos}
    \item {\textcolor{ForestGreen}{\textbf{=FILTER(range,cond1,cond2...)}} : Muestra datos que cumplen con una o más condiciones en un rango específico. \textit{A diferencia de \textcolor{ForestGreen}{QUERY}}, no puede combinarse con otras funciones en pos de cálculos más complejos}
    
\end{itemize}

\begin{description}
    \item[Celda]{ : Posición descrita usando filas \textbf{o registros/\textit{records}} y columnas \textbf{o campos/\textit{fields}}. \textit{Ejemplo : columna A fila 3 = A3}. 
    \begin{description}
        \item[Referencia Relativa]{ : El valor de la celda cambia junto con la fórmula donde está presente, al arrastrarla o copiarla en otro lugar}
        \item[Referencia Absoluta]{ : El valor de la celda es fijo y no cambia con movimientos en fórmulas o celdas.\textit{Para realizar cambio entre absoluto y relativo, presionar F4}}
        \begin{table}
            \centering
            \begin{tabular}{|p{5cm}|p{4.5cm}|p{4.5cm}|}
                \hline
                \multicolumn{3}{|c|}{Diferencias entre Referencias de Celdas} \\
                \hline
                & \textbf{Referencia Absoluta} & \textbf{Referencia Relativa} \\
                \hline
                \textbf{Ejemplos} & A1, C3, G2:G9 & \$A\$10, C\$2, \$D3 \\
                \hline
                \textbf{Si la celda se arrastra una columna a la derecha, una fila hacia abajo} & A10 $\rightarrow$ B11 & \$A10 $\rightarrow$ \$A11 \\
                \hline
            \end{tabular}
        \end{table}
    \end{description}}
    \item[Fórmulas y Funciones]{ : 
    \begin{description}
        \item {No todas las funciones requieren operadores. \textit{Ejemplos : SUM, AVERAGE, COUNT, MIN, MAX}}
        \item[Errores Comunes]{ : 
        \begin{itemize}
            \item {\textbf{\#DIV/0!} : Intento de dividir por cero o por una celda vacía}
            \item {\textbf{\#ERROR()} : Fórmula no puede ser interpretada como input. \textit{Parsing error}}
            \item {\textbf{\#N/A} : Los datos a usar en la fórmula no se encuentran en la hoja}
            \item {\textbf{\#NAME?} : No se entiende el nombre de la fórmula o función}
            \item {\textbf{\#NUM!} : Los cálculos de la fórmula o función no pueden realizarse}
            \item {\textbf{\#VALUE!} : Error general para problemas con fórmulas o referencias de celda}
            \item {\textbf{\#REF!} : Una fórmula referencia una celda no válida o eliminada}
        \end{itemize}} 
    \end{description}}
\end{description}




% separando subsecciones por motivos de orden
\subsection{\textit{DB + Queries} - Bases de Datos y Consultas}
\begin{itemize}
    \item {Aislar información específica de una base de datos}
    \item {Facilitar aprendizaje y comprensión de las solicitudes/consultas}
    \item {Seleccionar, crear, agregar o descargar datos desde la Base de Datos para Análisis}
    \item {\textbf{\textit{Primary Key} - Clave Primaria} : Identificador que referencia una columna donde cada uno de sus valores es único
    \begin{itemize}
        \item {Se utiliza para asegurar que los datos en una columna en específico son únicos}
        \item {No admite valores nulos o blancos}
        \item {Solo se permite una llave primaria en una tabla. Además, las tablas no siempre requieren tener una clave primaria}
        \item {Es posible construir una clave primaria a partir de varias columnas de una tabla: \textbf{Clave Compuesta}}
    \end{itemize}}
    \item {\textbf{\textit{Foreign Key} - Clave Externa} : Campo dentro de una tabla que es una llave primaria en otra.
    \begin{itemize}
        \item {Columna (o grupo de) que permite conectar los datos entre dos tablas}
        \item {Como referencia a la clave primaria de otra tabla, permite conectarlas}
        \item {Puede existir más de una clave externa en una tabla}
    \end{itemize}}
    \item {\textbf{Tipos de Base de Datos : }
    \begin{itemize}
        \item {\textbf{Relacional} : Bases de datos que contienen un grupo de tablas que pueden conectarse entre si utilizando sus relaciones o lo que tengan en común. Generalmente se relacionan con \textit{SQL - Structured Query Language} \textbf{Si un campo se usa en dos tablas, esto puede usarse para conectarlas.} \textit{Ejemplos de dialectos SQL: MySQL, PostgreSQL, Big Query}}
        \item {\textbf{No Relacional} : Bases de datos donde todas las variables posibles a analizar, se agrupan conjuntamente. Esto las hace difícil de clasificar. Generalmente relacionadas con dialectos \textit{NOSQL}}
    \end{itemize}}
\end{itemize}

\subsubsection{\textit{Metadata} - Metadatos}
Los metadatos son datos sobre datos, y permiten interpretar los contenidos de una hoja de cálculo o DB. Como parte de \gls{dtgvrnnc}, se almacenan en una ubicación central, convirtiéndose en una fuente única y verídica de información consistente y uniforme, de acceso estandarizado. Así, se asegura que los datos son exactos, precisos, relevantes y actualizados. 

\paragraph{Repositorios de \textit{Metadata}}
Bases de datos creadas específicamente para guardar metadatos, ya sea de forma física o virtual. Permiten realizar agregaciones de datos de forma más fácil y rápida al describir aspectos como
\begin{itemize}
    \item {Estado y Ubicación de los metadatos}
    \item {Estructura de las tablas contenidas}
    \item {Descripción de flujo de datos dentro del repositorio}
    \item {Seguimiento de los usuarios que acceden a los metadatos, junto con el tiempo de acceso}
\end{itemize}


\paragraph{Algunos ejemplos de elementos de \textit{Metadata}}
\begin{itemize}
    \item {Títulos y Descripciones}
    \item {Etiquetas y Categorías}
    \item {Permisos de Acceso y Edición}
    \item {Fecha de Creación y Usuario a cargo}
    \item {Fecha de Modificación y Usuario que realizó el cambio}
\end{itemize}

\paragraph{Tipos más comunes de \textit{Metadata}}
\begin{itemize}
        \item {\textbf{Descriptivo} : Describe los datos, y puede utilizarse para identificarlos a futuro}
        \item {\textbf{Estructural} : Indica cómo un grupo de datos se organiza, y si es que pertenece (o no) a una o más recolecciones de datos}
        \item {\textbf{Administrativo} : Indica la fuente técnica de una propiedad u objeto digital}
\end{itemize}

\subsubsection{Comandos y Sintaxis SQL}
Tomar en cuenta que existen múltiples dialectos SQL y es recomendable consultar la documentación. Sin embargo, la gran mayoría de las instrucciones son comunes, y se promueven buenas prácticas de comandos. Algunos de los dialectos SQL consisten en \textit{MySQL, BigQuery, PostgreSQL, Microsoft SQL, SQLite}
\begin{itemize}
    \item [*]{ : selector universal para todas las columnas}
\end{itemize}

\paragraph{Cláusulas}
Instrucciones básicas de todo \gls{qry} de SQL
\begin{itemize}
    \item {\textbf{SELECT}}
    \item {\textbf{FROM}\textit{ : table\_name}}
    \item {\textbf{WHERE} : filtro para incorporar condiciones tipo \textit{WHEN}, etc}
\end{itemize}

\paragraph{Buenas Prácticas}
\begin{itemize}
    \item {MAYÚSCULAS para iniciadores/cláusulas y funciones}
    \item {minúsculas para nombres de columnas. evitar mezclas}
    \item {'comillas simples' para strings}
    \item {${"}$comillas dobles${"}$ si el string contiene comillas simples/'apóstrofos'}
    \item {${--}$realizar comentarios para revisiones a futuro}
    \item{/* Para comentarios muy extensos, considerar este formato */}
    \item {\textbf{\textit{snake\_case}} como reemplazo de los espacios en los nombres\_de\_columnas}
    \item {\textbf{\textit{CamelCase}} para nombrar tablas. laPrimeraMayúscula no es obligatoria}
    \item {Uso de sangrías para mantener cada línea en menos de 100 caracteres}
    \item {Considerar el uso de editores de texto SQL, que mejoran la legibilidad a través de ediciones como colores y sangrías, además de aceptar \gls{rgx}}
\end{itemize}

\paragraph{Excepciones}
\begin{itemize}
    \item {Generalmente el uso de mayúsculas/minúsculas no afecta al código. BigQuery si distingue entre mayúsculas/minúsculas, a diferencia de otros dialectos}
    \item {Lo mismo para comillas simples o dobles, pero las buenas prácticas garantizan buen funcionamiento a través de todos los dialectos}
    \item {MySQL acepta \#etiqueta para comentarios}
\end{itemize}

\subsection{\textit{Spreadsheets vs Databases}}
Considerar que para ambos, \textbf{Campo} corresponde a las columnas, y \textbf{Registro} a las filas. En un registro se contienen múltiples campos, o sea: un set de datos puede tener múltiples valores.
\begin{table}
    \centering
    \begin{tabular}{|p{3.5cm}|p{5.5cm}|p{5.5cm}|}
        \hline
        & \textbf{Spreadsheet} & \textbf{Database} \\
        \hline
        \textbf{Ubicación} & Apps de Software & Almacenes de datos accesibles mediante \textit{queries} \\
        \hline
        \textbf{Estructura\break de Datos} & Formato de filas y columnas & Uso de reglas y relaciones \\
        \hline
        \textbf{Organización} & En celdas & En colecciones completas \\
        \hline
        \textbf{Acceso de Datos} & Cantidad limitada & Grandes cantidades \\
        \hline
        \textbf{Ingreso de Datos} & Manual & Escrito y Coherente \\
        \hline
        \textbf{Personas\break trabajando} & Generalmente un usuario a la vez & Múltiples usuarios \\
        \hline
        \textbf{Controlado por} & Usuario & Sistema de gestión de base de datos \\
        \hline
    \end{tabular}
\end{table}

\subsection{\textit{Data Viz} - Herramientas de Visualización}
\begin{itemize}
    \item {Permiten crear gráficos, tablas, mapas y cuadros}
    \item {Convierten números complejos en historias que las personas puedan entender}
    \item {Ayudan a interesados a sacar conclusiones que permiten tomar decisiones informadas y elaborar estrategias empresariales eficaces}
    \item {\textbf{Métrica} : Tipo de dato singular y cuantificable que puede utilizarse para medir algo
    \begin{itemize}
        \item {\textbf{Meta Métrica} : Meta propuesta por una compañía que puede ser medida y evaluada de acuerdo a métricas}
        \item {\textit{Ejemplo} : Retorno de Inversión = $\frac{Ganancia Neta}{Costo de Inversión}$}
    \end{itemize}}
    \item {\textbf{\textit{Pivot Table} - Tabla Pivote} : Utilizado en el procesamiento de datos, permite resumir, ordenar y reorganizar datos en un DB, para realizar cálculos como promedios o cantidades totales}
\end{itemize}

\subsubsection{\textit{Report} - Informes}
Recuento estático de datos, entregados de forma periódica a \Gls{stkhldrs}

\subsubsection{\textit{Dashboard} - Panel}
Monitoreo constante y en vivo de datos entrantes, con una ubicación central que los agrupa. El ingreso de datos es automático \textit{solo si la estructura de datos es la misma}
\paragraph{Proceso para crear un panel}
    \begin{itemize}
        \item {Identificar a \Gls{stkhldrs} que necesiten ver los datos y cómo los usarán, a través de preguntas efectivas}
        \item {Diseño del panel : encabezados claros para etiquetar información ; descripciones de texto concisas para cada visualización ; mostrar la información más importante en la parte superior}
        \item {Crear prototipos}
        \item {Seleccionar visualizaciones para utilizar en el panel : cambio de valores a lo largo del tiempo con gráficos de línea y barras ; contribución de cada parte al resultado final con gráficos circulares y de torta}
        \item {Crear filtros de ser necesario, identificar patrones mientras los datos originales se mantienen intactos}
    \end{itemize}
\paragraph{Tres categorías comunes de paneles}
    \begin{itemize}
        \item {\textbf{Paneles Estratégicos} : Toma de decisiones para la empresa en periodos más largos}
        \item {\textbf{Paneles Operativos} : Escala temporal de días, semanas o meses, prácticamente a tiempo real}
        \item {\textbf{Paneles Analíticos} : Uso, análisis y predicciones como categoría más técnica, realizadas por \textit{data scientists}}
    \end{itemize}

\begin{table}
    \centering
    \begin{tabular}{|p{2.9cm}|p{5.4cm}|p{5.5cm}|}
        \hline
        \multicolumn{3}{|c|}{Beneficios de paneles} \\
        \hline
        & \textbf{Para analistas} & \textbf{Para \Gls{stkhldrs}} \\
        \hline
        \small{\textbf{Centralización}} & Compartir una única fuente de datos con los interesados & Trabajar con visión integral de datos, iniciativas y procesos \\
        \hline
        \small{\textbf{Visualización}} & Mostrar y actualizar datos \break entrantes en tiempo real & Detectar patrones y tendencias cambiante más rápidamente \\
        \hline
         \small{\textbf{Percepción}} & Extraer información desde \break diferentes conjuntos de datos & Comprender la historia detrás de los números \\
        \hline
        \small{\textbf{Personalización}} & Crear vistas personalizadas para los interesados & Profundizar en preocupaciones o intereses específicos \\
        \hline
    \end{tabular}
\end{table}

\begin{table}
    \centering
    \begin{tabular}{|p{1.8cm}|p{5.9cm}|p{5.9cm}|}
        \hline
        & \textbf{\textit{Reports}} & \textbf{\textit{Dashboard}} \\
        \hline
        \textbf{Pros} & \begin{description}
            \item {Datos de alto nivel histórico}
            \item {Fácil de diseñar y usar}
            \item {Datos limpios y ordenados}
        \end{description} & \begin{description}
            \item {Automático e interactivo}
            \item {Mejor acceso para \Gls{stkhldrs}}
            \item {Bajo mantenimiento}
        \end{description} \\
        \hline
        \textbf{Contras} & \begin{description}
            \item {Mantenimiento contínuo}
            \item {Estático}
            \item {Menos atractivo visual}
        \end{description} & \begin{description}
            \item {Diseño requiere más trabajo}
            \item {Puede ser confuso}
            \item {Datos sin procesar}
        \end{description} \\
        \hline
    \end{tabular}
\end{table}
    

\subsection{Ejemplos y Otras Herramientas}
Algunos ejemplos consisten en :
\begin{itemize}
        \item {\textbf{Tableau} : \textit{arrastrar + soltar} para crear gráficos interactivos en paneles y hojas de trabajo}
        \item {\textbf{Looker} : Conexión directa con bases de datos}
\end{itemize}
Pueden usarse otras herramientas para análisis estadístico, visualizaciones y otras tareas
\begin{itemize}
    \item {R}
    \item {Python}
\end{itemize}

\newpage