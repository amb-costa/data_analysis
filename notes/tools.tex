% nta : esta sección desaparece eventualmente
% cada parte del kit necesita ser una sección de por si
% entonces también solucionar la subsección de DB

\section {Kit de Herramientas del Analista}
% separando subsecciones por motivos de orden
\subsection{\textit{DB + Queries} - Bases de Datos y Consultas}
\begin{itemize}
    \item {Aislar información específica de una base de datos}
    \item {Facilitar aprendizaje y comprensión de las solicitudes/consultas}
    \item {Seleccionar, crear, agregar o descargar datos desde la Base de Datos para Análisis}
    \item {\textbf{\textit{Primary Key} - Clave Primaria} : Identificador que referencia una columna donde cada uno de sus valores es único
    \begin{itemize}
        \item {Se utiliza para asegurar que los datos en una columna en específico son únicos}
        \item {No admite valores nulos o blancos}
        \item {Solo se permite una llave primaria en una tabla. Además, las tablas no siempre requieren tener una clave primaria}
        \item {Es posible construir una clave primaria a partir de varias columnas de una tabla: \textbf{Clave Compuesta}}
    \end{itemize}}
    \item {\textbf{\textit{Foreign Key} - Clave Externa} : Campo dentro de una tabla que es una llave primaria en otra.
    \begin{itemize}
        \item {Columna (o grupo de) que permite conectar los datos entre dos tablas}
        \item {Como referencia a la clave primaria de otra tabla, permite conectarlas}
        \item {Puede existir más de una clave externa en una tabla}
    \end{itemize}}
    \item {\textbf{Tipos de Base de Datos : }
    \begin{itemize}
        \item {\textbf{Relacional} : Bases de datos que contienen un grupo de tablas que pueden conectarse entre si utilizando sus relaciones o lo que tengan en común. Generalmente se relacionan con \textit{SQL - Structured Query Language} \textbf{Si un campo se usa en dos tablas, esto puede usarse para conectarlas.} \textit{Ejemplos de dialectos SQL: MySQL, PostgreSQL, Big Query}}
        \item {\textbf{No Relacional} : Bases de datos donde todas las variables posibles a analizar, se agrupan conjuntamente. Esto las hace difícil de clasificar. Generalmente relacionadas con dialectos \textit{NOSQL}}
    \end{itemize}}
\end{itemize}

\subsubsection{\textit{Metadata} - Metadatos}
Los metadatos son datos sobre datos, y permiten interpretar los contenidos de una hoja de cálculo o DB. Como parte de \gls{dtgvrnnc}, se almacenan en una ubicación central, convirtiéndose en una fuente única y verídica de información consistente y uniforme, de acceso estandarizado. Así, se asegura que los datos son exactos, precisos, relevantes y actualizados. 

\paragraph{Repositorios de \textit{Metadata}}
Bases de datos creadas específicamente para guardar metadatos, ya sea de forma física o virtual. Permiten realizar agregaciones de datos de forma más fácil y rápida al describir aspectos como
\begin{itemize}
    \item {Estado y Ubicación de los metadatos}
    \item {Estructura de las tablas contenidas}
    \item {Descripción de flujo de datos dentro del repositorio}
    \item {Seguimiento de los usuarios que acceden a los metadatos, junto con el tiempo de acceso}
\end{itemize}


\paragraph{Algunos ejemplos de elementos de \textit{Metadata}}
\begin{itemize}
    \item {Títulos y Descripciones}
    \item {Etiquetas y Categorías}
    \item {Permisos de Acceso y Edición}
    \item {Fecha de Creación y Usuario a cargo}
    \item {Fecha de Modificación y Usuario que realizó el cambio}
\end{itemize}

\paragraph{Tipos más comunes de \textit{Metadata}}
\begin{itemize}
        \item {\textbf{Descriptivo} : Describe los datos, y puede utilizarse para identificarlos a futuro}
        \item {\textbf{Estructural} : Indica cómo un grupo de datos se organiza, y si es que pertenece (o no) a una o más recolecciones de datos}
        \item {\textbf{Administrativo} : Indica la fuente técnica de una propiedad u objeto digital}
\end{itemize}

\subsubsection{Comandos y Sintaxis SQL}
Tomar en cuenta que existen múltiples dialectos SQL y es recomendable consultar la documentación. Sin embargo, la gran mayoría de las instrucciones son comunes, y se promueven buenas prácticas de comandos. Algunos de los dialectos SQL consisten en \textit{MySQL, BigQuery, PostgreSQL, Microsoft SQL, SQLite}
\begin{itemize}
    \item [*]{ : selector universal para todas las columnas}
\end{itemize}

\paragraph{Cláusulas}
Instrucciones básicas de todo \gls{qry} de SQL
\begin{itemize}
    \item {\textbf{SELECT}}
    \item {\textbf{FROM}\textit{ : table\_name}}
    \item {\textbf{WHERE} : filtro para incorporar condiciones tipo \textit{WHEN}, etc}
\end{itemize}

\paragraph{Buenas Prácticas}
\begin{itemize}
    \item {MAYÚSCULAS para iniciadores/cláusulas y funciones}
    \item {minúsculas para nombres de columnas. evitar mezclas}
    \item {'comillas simples' para strings}
    \item {${"}$comillas dobles${"}$ si el string contiene comillas simples/'apóstrofos'}
    \item {${--}$realizar comentarios para revisiones a futuro}
    \item{/* Para comentarios muy extensos, considerar este formato */}
    \item {\textbf{\textit{snake\_case}} como reemplazo de los espacios en los nombres\_de\_columnas}
    \item {\textbf{\textit{CamelCase}} para nombrar tablas. laPrimeraMayúscula no es obligatoria}
    \item {Uso de sangrías para mantener cada línea en menos de 100 caracteres}
    \item {Considerar el uso de editores de texto SQL, que mejoran la legibilidad a través de ediciones como colores y sangrías, además de aceptar \gls{rgx}}
\end{itemize}

\paragraph{Excepciones}
\begin{itemize}
    \item {Generalmente el uso de mayúsculas/minúsculas no afecta al código. BigQuery si distingue entre mayúsculas/minúsculas, a diferencia de otros dialectos}
    \item {Lo mismo para comillas simples o dobles, pero las buenas prácticas garantizan buen funcionamiento a través de todos los dialectos}
    \item {MySQL acepta \#etiqueta para comentarios}
\end{itemize}

\subsection{\textit{Spreadsheets vs Databases}}
Considerar que para ambos, \textbf{Campo} corresponde a las columnas, y \textbf{Registro} a las filas. En un registro se contienen múltiples campos, o sea: un set de datos puede tener múltiples valores.

\paragraph{Características en Común}
Ambas herramientas permiten aritmética, fórmulas, y la agregación de datos

\paragraph{Diferencias}
Las hojas de cálculo se generan en un programa, mientras que SQL es un lenguaje de interacción con programas de DB.
Las hojas de cálculo se guardan de forma local, mientras que las consultas SQL se mantienen en la DB.
Las hojas de cálculo brindan acceso a datos que el usuario ingresa. SQL puede obtener información desde fuentes variadas en la misma DB
Las hojas de cálculo son ideales para sets de datos pequeños, SQL permiten trabajar con sets más grandes
Las hojas de cálculo facilitan el trabajo independiente, mientras que SQL mantiene registro de los cambios dentro del equipo
Las hojas de cálculo contienen funcionalidades incorporadas, y SQL es útil en varios programas


\begin{table}
    \centering
    \begin{tabular}{|p{3.5cm}|p{5.5cm}|p{5.5cm}|}
        \hline
        & \textbf{Spreadsheet} & \textbf{Database} \\
        \hline
        \textbf{Ubicación} & Apps de Software & Almacenes de datos accesibles mediante \textit{queries} \\
        \hline
        \textbf{Estructura\break de Datos} & Formato de filas y columnas & Uso de reglas y relaciones \\
        \hline
        \textbf{Organización} & En celdas & En colecciones completas \\
        \hline
        \textbf{Acceso de Datos} & Cantidad limitada & Grandes cantidades \\
        \hline
        \textbf{Ingreso de Datos} & Manual & Escrito y Coherente \\
        \hline
        \textbf{Personas\break trabajando} & Generalmente un usuario a la vez & Múltiples usuarios \\
        \hline
        \textbf{Controlado por} & Usuario & Sistema de gestión de base de datos \\
        \hline
    \end{tabular}
\end{table}

\begin{table}
    \centering
    \begin{tabular}{|p{2.9cm}|p{5.4cm}|p{5.5cm}|}
        \hline
        \multicolumn{3}{|c|}{Beneficios de paneles} \\
        \hline
        & \textbf{Para analistas} & \textbf{Para \Gls{stkhldrs}} \\
        \hline
        \small{\textbf{Centralización}} & Compartir una única fuente de datos con los interesados & Trabajar con visión integral de datos, iniciativas y procesos \\
        \hline
        \small{\textbf{Visualización}} & Mostrar y actualizar datos \break entrantes en tiempo real & Detectar patrones y tendencias cambiante más rápidamente \\
        \hline
         \small{\textbf{Percepción}} & Extraer información desde \break diferentes conjuntos de datos & Comprender la historia detrás de los números \\
        \hline
        \small{\textbf{Personalización}} & Crear vistas personalizadas para los interesados & Profundizar en preocupaciones o intereses específicos \\
        \hline
    \end{tabular}
\end{table}

\begin{table}
    \centering
    \begin{tabular}{|p{1.8cm}|p{5.9cm}|p{5.9cm}|}
        \hline
        & \textbf{\textit{Reports}} & \textbf{\textit{Dashboard}} \\
        \hline
        \textbf{Pros} & \begin{description}
            \item {Datos de alto nivel histórico}
            \item {Fácil de diseñar y usar}
            \item {Datos limpios y ordenados}
        \end{description} & \begin{description}
            \item {Automático e interactivo}
            \item {Mejor acceso para \Gls{stkhldrs}}
            \item {Bajo mantenimiento}
        \end{description} \\
        \hline
        \textbf{Contras} & \begin{description}
            \item {Mantenimiento contínuo}
            \item {Estático}
            \item {Menos atractivo visual}
        \end{description} & \begin{description}
            \item {Diseño requiere más trabajo}
            \item {Puede ser confuso}
            \item {Datos sin procesar}
        \end{description} \\
        \hline
    \end{tabular}
\end{table}

\newpage