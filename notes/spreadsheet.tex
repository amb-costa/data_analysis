% separación de subsección por motivos de orden

\subsection{\textit{Spreadsheets} - Hojas de Cálculo}
Las hojas de cálculo pueden conectarse con otras hojas, tablas HTML y archivos \gls{csv}.

\paragraph{Herramientas Comunes}
Considerar que toda fórmula se comienza escribiendo '='. Al hacerlo, se despliega un menú que incluye fórmulas, nombres y \gls{str}. Esta lista incluye ejemplos de \gls{syntx} para fórmulas y funciones
\begin{itemize}
    \item {\textbf{Formateo Condicional} : Cambia la estética de las celdas cuando sus valores cumplen con condiciones especificadas}
    \item {\textbf{Llenado Automático} : Presente como un cuadrado en la esquina inferior derecha de la celda, se arrastra y llena celdas vacías con el contenido y referencias de la celda inicial}
    \item {\textbf{Remover Duplicados} : Busca y elimina automáticamente aquellas entradas que podrían estar duplicadas} 
    \item {\textbf{SPLIT} : Función que divide el texto en base a un caracter, y guarda cada fragmento en una celda nueva. El caracter es un \textbf{delimitador} y puede ser un espacio, coma, guiones, entre otros}
    \item {\textbf{=CONCATENATE(item1,item2...)} : Función que une múltiples \gls{str} en uno solo. \textit{=CONCATENATE("text","string") devuelve "textstring"}}
    \item {\textbf{=COUNTIF(rango,"valor\_comillas")} : Función que devuelve el número de celdas que cumplen con un valor específico dentro de un rango determinado. \textit{=COUNTIF(A2:A30,'${\geq}$10') devuelve la cantidad de valores que son iguales o mayores a 10}}
    \item {\textbf{=LEN(rango)} : Función que entrega el largo de un \gls{str} al contar el número de caracteres que contiene. \textit{=LEN(A21) devuelve el largo del string en la celda A21}}
    \item {\textbf{=LEFT} : Función que devuelve un conjunto de caracteres a la izquierda de un \gls{str} determinado}
    \item {\textbf{=RIGHT} : Función que devuelve un conjunto de caracteres a la derecha de un \gls{str} determinado}
    \item {\textbf{=MID(rango,punto\_referencia,cant\_char)} : Función que devuelve un segmento desde el medio de un \gls{str}. \textit{=MID(C3,4,5) devuelve un string de 5 caracteres que empezó en la posición 4 del string inicial ubicado en la celda C3}}
    \item {\textbf{=TRIM(rango)} : Función que remueve los espacios iniciales, finales y repetidos de un \gls{str}. \textit{=TRIM('' a esto le sobran espacios  '') devuelve ''a esto le sobran espacios''}}    
\end{itemize}

\begin{description}
    \item[Celda]{ : Posición descrita usando filas \textbf{o registros/\textit{records}} y columnas \textbf{o campos/\textit{fields}}. \textit{Ejemplo : columna A fila 3 = A3}. 
    \begin{description}
        \item[Referencia Relativa]{ : El valor de la celda cambia junto con la fórmula donde está presente, al arrastrarla o copiarla en otro lugar}
        \item[Referencia Absoluta]{ : El valor de la celda es fijo y no cambia con movimientos en fórmulas o celdas.\textit{Para realizar cambio entre absoluto y relativo, presionar F4}}
        \begin{table}
            \centering
            \begin{tabular}{|p{5cm}|p{4.5cm}|p{4.5cm}|}
                \hline
                \multicolumn{3}{|c|}{Diferencias entre Referencias de Celdas} \\
                \hline
                & \textbf{Referencia Absoluta} & \textbf{Referencia Relativa} \\
                \hline
                \textbf{Ejemplos} & A1, C3, G2:G9 & \$A\$10, C\$2, \$D3 \\
                \hline
                \textbf{Si la celda se arrastra una columna a la derecha, una fila hacia abajo} & A10 $\rightarrow$ B11 & \$A10 $\rightarrow$ \$A11 \\
                \hline
            \end{tabular}
        \end{table}
    \end{description}}
    \item[Fórmulas y Funciones]{ : 
    \begin{description}
        \item {No todas las funciones requieren operadores. \textit{Ejemplos : SUM, AVERAGE, COUNT, MIN, MAX}}
        \item[Errores Comunes]{ : 
        \begin{itemize}
            \item {\textbf{\#DIV/0!} : Intento de dividir por cero o por una celda vacía}
            \item {\textbf{\#ERROR()} : Fórmula no puede ser interpretada como input. \textit{Parsing error}}
            \item {\textbf{\#N/A} : Los datos a usar en la fórmula no se encuentran en la hoja}
            \item {\textbf{\#NAME?} : No se entiende el nombre de la fórmula o función}
            \item {\textbf{\#NUM!} : Los cálculos de la fórmula o función no pueden realizarse}
            \item {\textbf{\#VALUE!} : Error general para problemas con fórmulas o referencias de celda}
            \item {\textbf{\#REF!} : Una fórmula referencia una celda no válida o eliminada}
        \end{itemize}} 
    \end{description}}
\end{description}



